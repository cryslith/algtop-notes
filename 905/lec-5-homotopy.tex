\section{Homotopy, star-shaped regions}
The idea of homotopy is the central concept of all of algebraic topology, to put it bluntly. We have a functor $ H_\ast:\mathbf{Top}\to\mathbf{Ab}$, but it's too crude to distinguish between topological spaces. The virtue of this definition is that it's computable, although it can't completely distinguish between two spaces.
\begin{definition}
Let $f_0,f_1:X\to Y$ be two maps. A homotopy from $f_0$ to $f_1$ is a map $h:X\times I\to Y$ such that $h(x,0)=f_0(x)$ and $f(x,1)=f_1(x)$. We say that $f_0$ and $f_1$ are homotopic, and this is written $f_0\sim f_1$ because this is an equivalence relation on maps (transitivity follows from the gluing lemma).
\end{definition}
We denote by $[X,Y]$ the set $\mathbf{Top}(X,Y)/\sim$. On Monday, we'll prove the following result.
\begin{theorem}
If $f_0\sim f_1$, then $ H_\ast(f_0)= H_\ast(f_1)$. In other words, homology cannot distinguish between homotopic things.
\end{theorem}
Suppose I have two maps $f_0,f_1:X\to Y$ and a map $g:Y\to Z$, with a homotopy $h:f_0\sim f_1$. Composing $h$ with $g$ gives a homotopy between $g\circ f_0$ and $g\circ f_1$. Precomposing also works. Namely, if $g:W\to X$ is a map and $f_0,f_1:X\to Y$ are homotopic, then $f_0g\sim f_1g$. Does this let us compose homotopy classes? That is, can we complete:
\begin{equation*}
\xymatrix{\mathbf{Top}(Y,Z)\times\mathbf{Top}(X,Y)\ar[d]\ar[r] & \mathbf{Top}(X,Z)\ar[d]\\
[Y,Z]\times[X,Y]\ar@{-->}[r] & [X,Z]}
\end{equation*}
The answer is yes. If $g_0\sim g_1:Y\to Z$ and $f_0\sim f_1: X\to Y$, then $g_0f_0\sim g_0f_1$, but this is just homotopic to $g_1f_1$. This therefore gives us a new category, called the homotopy category.
\begin{definition}
The homotopy category of topological spaces is $\mathrm{Ho}(\mathbf{Top})$ whose objects are topological spaces and $\mathrm{Ho}(\mathbf{Top})(X,Y)=[X,Y]=\mathbf{Top}(X,Y)/\sim$.
\end{definition}
This is an interesting category because it has \textit{terrible} categorical properties. The theorem that we'll prove on Monday says that the homology functor $ H_\ast:\mathbf{Top}\to\mathbf{Ab}$ factors as $\mathbf{Top}\to\mathrm{Ho}(\mathbf{Top})\to\mathbf{Ab}$.
\begin{definition}
A map $f:X\to Y$ is a homotopy equivalence if $[f]\in[X,Y]$ is an isomorphism in $\htop$. In other words, there is $g:Y\to X$ such that $fg\sim 1_Y$ and $gf\sim 1_X$.
\end{definition}
\begin{example}
Homotopy equivalence doesn't preserve compactness. For example, the inclusion $S^{n-1}\subseteq \mathbf{R}^n-\{0\}$. The homotopy inverse $p:\mathbf{R}^n-\{0\}\to S^{n-1}$ can be obtained by dividing a (always nonzero!) vector by its length. Clearly $p\circ i=1_{S^{n-1}}$. We have to find a homotopy $i\circ p\sim 1_{\mathbf{R}^n-\{0\}}$. Namely, we want a map $(\mathbf{R}^n-\{0\})\times I\to \mathbf{R}^n-\{0\}$; but this is straightforward, namely, take $(v,t)\mapsto tv+(1-t)\frac{v}{||v||}$.
\end{example}
Let's set up the stuff about star-shaped region.
\begin{definition}
A star-shaped region is a subspace $X$ of $\mathbf{R}^n$ for some $n$ such that $0\in X$, such that for all $x\in X$, and for all $t\in[0,1]$, $tx\in X$. 
\end{definition}
\begin{example}
Any convex region containing the origin is star shaped.
\end{example}

The first thing to realize is that the inclusion $\{0\}\to X$, where $X$ is a star-shaped region, is a homotopy equivalence, precisely like the above example!
\begin{definition}
A space $X$ is contractible if the map $X\to\ast$ is a homotopy equivalence.
\end{definition}
In other words, the star-shaped region is contractible. The claim is the following.
\begin{theorem}
The augmentation map $\epsilon: H_\ast(X)\to \mathbf{Z}$ is an isomorphism, i.e., $ H_0(X)\cong\mathbf{Z}$ and $ H_i(X)\cong 0$ for $i>0$.
\end{theorem}
The strategy of proof is that $\epsilon$ is induced by sending $X\to \ast$. We'll look at the chain map $S_\ast(x)\to\mathbf{Z}\to S_\ast(X)$. We'll show that this composite induces the same map in homology as the identity map, which means that the identity map factors through $\mathbf{Z}$, so we're done.

We want to set up a homotopy in the category of chain complexes.
\begin{definition}
Let $C_\bullet,D_\bullet$ be chain complexes, and $f_0,f_1:C_\bullet\to D_\bullet$ be chain maps. A chain homotopy $h:f_0\sim f_1$ is a collection of homomorphisms $h:C_n\to D_{n+1}$ such that $\partial h+h\partial=f_1-f_0$.
\end{definition}
It's a really weird relation, but we'll understand why this is a good thing.
\begin{equation*}
\xymatrix{C_{n+2}\ar[r]^{f_1-f_0}\ar[d]^\partial & D_{n+2}\ar[d]^\partial\\
C_{n+1}\ar@{-->}[ur]^h\ar[r]^{f_1-f_0}\ar[d]^\partial & D_{n+2}\ar[d]^\partial\\
C_n\ar@{-->}[ur]^h\ar[r]_{f_1-f_0} & D_n}
\end{equation*}
\begin{lemma}
If $f_0,f_1:C_\bullet\to D_\bullet$ are chain homotopic, then $f_{0,\ast}=f_{1,\ast}: H(C)\to H(D)$.
\end{lemma}
\begin{proof}
This is the same thing as saying that $(f_1-f_0)_\ast=0$. Let $c\in Z_n(C_\bullet)(C)$, so that $\partial c=0$. Well, $(f_1-f_0)_\ast c=(\partial h+h\partial)c=\partial hc+h\partial c=\partial hc$, so it's a boundary. Therefore, $f_{1,\ast}$ and $f_{0,\ast}$ have the same homology class (because when you take homology classes, you work modulo boundaries), which means we're done.
\end{proof}
\begin{proof}
We have the map $S_\ast(X)\xrightarrow{\epsilon}\mathbf{Z}\xrightarrow{\eta}S_\ast(X)$ where the latter map sends $1$ to the constant map at the origin. The claim is that $\eta\epsilon\sim 1:S_\ast(X)\to S_\ast(X)$. Let's just mention that $\eta\epsilon(\sum a_ix_i)=(\sum a_i)c_0$ where $c_0$ is the zero-simplex that is constant at the origin. We want a homotopy $h:S_q(X)\to S_{q+1}(X)$, but it suffices to construct $\Sin_q(X)\to \Sin_{q+1}(X)$. For $\sigma\in\Sin_q(X)$, define $h:\Sin_q(X)\to\Sin_{q+1}(X)$ as follows:
$$h\sigma(t_0,\cdots,t_{q+1})=(1-t_0)\sigma\left(\frac{(t_0,\cdots,t_{q+1})}{1-t_0}\right)$$
I give up livetexing the proof; just read Hatcher.
\end{proof}
% Continue indenting like this.
