% I removed the definition of a manifold and moved it in-line for better flow.
% I think it's best to use /emph in definitions to show what's being defined.
% Put periods even in display math!
% For spacing reasons, should use \colon instead of : for functions, e.g. f \colon A \to B.
% Since ker is lowercase, im probably should be also.

\section{Introduction, simplices}\label{905}
In 18.905, which is the first half of this book, we will cover the following topics:
\begin{enumerate}
    \item Singular homology
    \item CW-complexes
    \item Basics of category theory
    \item Homological algebra
    \item The K\"{u}nneth theorem
    \item UCT, cohomology
    \item Cup and cap products, and
    \item Poincar\'{e} duality.
\end{enumerate}
Below are some examples of commonly encountered topological spaces.
\begin{itemize}
    \item The most basic is \emph{$n$-dimensional Euclidean space}, $\mathbf{R}^n$.
    \item The \emph{$n$-sphere} $S^n=\{x\in \mathbf{R}^{n+1}:|x|=1\}$ is topologized as a subspace of $\mathbf{R}^{n+1}$.
    \item Identifying antipodal points in $S^n$ gives \emph{real projective space} $\mathbf{RP}^n=S^n / (x\sim -x)$, i.e. the space of lines through the origin in $\mathbf{R}^{n+1}$.
    \item Call an ordered collection of $k$ orthonormal vectors an \emph{orthonormal $k$-frame}. The space of orthonormal $k$-frames in $\mathbf{R}^n$ forms the \emph{Stiefel manifold} $V_k(\mathbf{R}^n)$, which is topologized as a subspace of $(S^{n-1})^k$.
    \item Let $x\sim y$ if $x$ and $y$ are $k$-frames with the same span. The \emph{Grassmannian} is the quotient $\mathrm{Gr}_k(\mathbf{R}^n)=V_k(\mathbf{R}^n)/\sim$. In particular, $\mathbf{Gr}_1(\mathbf{R}^n) = \mathbf{RP}^{n-1}$.
\end{itemize}
The above are all \emph{manifolds}, which are Hausdorff spaces locally homeomorphic to Euclidean space. Aside from $\mathbf{R}^n$ itself, the preceding examples are also compact. Such spaces exhibit a hidden symmetry, which is the culmination of 18.905: Poincar\'{e} duality.

As the name suggests, the central aim of algebraic topology is the usage of algebraic tools to study topological spaces. A common technique is to probe topological spaces via maps to them. In different ways, this approach gives rise to singular homology and homotopy groups. We now detail the former; the latter takes stage in 18.906.
\begin{definition}
For $n\geq 0$, the \emph{standard $n$-simplex} $\Delta^n$ is the convex hull of the standard basis $\{e_0,\cdots,e_n\}$ in $\mathbf{R}^{n+1}$. More explicitly,
$$\Delta^n = \left\{\sum t_i e_i : \sum t_i = 1, t_i\geq 0\right\}\subseteq\mathbf{R}^{n+1}.$$
The $t_i$ are called barycentric coordinates.
\end{definition}
We will write $i$ in lieu of $e_i$ to refer to the vertices of $\Delta^n$. The standard simplices are related by face inclusions $d^i\colon \Delta^{n-1} \to \Delta^{n}$ for $0\leq i \leq n$, where $d^i$ misses the vertex $i$. \todo{Insert pic here.}
\begin{definition}
Let $X$ be any topological space. A \emph{singular $n$-simplex} in $X$ is a continuous map $\Delta^n\to X$. Denote by $\mathrm{Sin}_n(X)$ the set of all $n$-simplices in $X$.
    
    This seems like a rather bold construction to make, as $\mathrm{Sin}_n(X)$ is huge. Nonetheless, we will soon make it even larger.
\end{definition}
(See drawing for a torus\todo{Image me}. The direction of the simplex is like an orientation given by ordering of the indices of $\Delta^n$. The standard notation is $\sigma:\Delta^n\to X$.)\todo{Depict this whole thing as an image, and remove the sentence when done}

For $0\leq i \leq n$, precomposition by the face inclusion $d^i$ produces a map $d_i\colon \Sin_n(X)\to\Sin_{n-1}(X)$ sending $\sigma\mapsto\sigma\circ d^i$, which is the $i$th face of $\sigma$. This allows us to make sense of the ``boundary'' of a simplex, and we are particularly interested in simplices for which that boundary vanishes.

For example, if $\sigma$ is a 1-simplex that goes around the hole in a torus $T$, then $d_1\sigma = d_0\sigma$. To express that the boundary vanishes, we want to write $d_0\sigma - d_1\sigma=0$, but this difference is no longer a simplex. To accommodate such formal sums, we will enlarge $\mathrm{Sin}_n(X)$ further by considering the free abelian group it generates.
\begin{definition}
The abelian group $S_n(X)$ of \emph{singular $n$-chains} in $X$ is the free abelian group generated by $n$-simplices
$$S_n(X) = \mathbf{Z}\Sin_n(X).$$
    Its elements are finite linear combinations, i.e. formal sums $\sum_{i\in\text{finite set}}a_i\sigma_i$\todo{idk what was meant by ``display'' in the comments} where $a_i\in\mathbf{Z}$. If $n<0$, say that $S_n(X)=0$. Now, define
$$\partial\colon \Sin_n(X)\to S_{n-1}(X),$$
$$\partial\sigma = \sum_{i=0}^n(-1)^i d_i\sigma.$$
This extends to a homomorphism $\partial \colon S_n(X) \to S_{n-1}(X)$ by additivity.
\end{definition}
We use this homomorphism to obtain something more tractable than the entirety of $S_n(X)$. First we restrict our attention to chains with vanishing boundary.
\begin{definition}
An \emph{$n$-cycle} in $X$ is an $n$-chain $c$ with $\partial c = 0$. Denote $Z_n(X) = \ker(S_n(X)\xrightarrow{\partial}S_{n-1}(X))$.
\end{definition}
For example, with $\sigma$ on the torus described before, $\sigma\in Z_1(X)$ since $\partial \sigma = d_0\sigma - d_1\sigma = 0$.
\begin{theorem}
Any boundary is a cycle, i.e., $B_n(X) := \mathrm{Im}(\partial:S_{n+1}(X)\to S_n(X))\subseteq Z_n(X)$.
\end{theorem}
\begin{proof}
    Homework!
\end{proof}
With the preceding, we are prepared to define singular homology.
\begin{definition}
The \emph{$n$th singular homology group} of $X$ is:
    $$ H_n(X) = Z_n(X)/B_n(X) = \frac{\ker(\partial:S_n(X)\to S_{n-1}(X))}{\mathrm{im}(\partial:S_{n+1}(X)\to S_n(X))}.$$
\end{definition}
Both $Z_n(X)$ and $B_n(X)$ are free abelian groups because they are subgroups of the free abelian group $S_n(X)$, but the quotient $H_n(X)$ isn't necessarily free. While $Z_n(X)$ and $B_n(X)$ are uncountably generated, $H_n(X)$ is finitely generated for the spaces we are interested in. If $T$ is the torus for example, then $H_1(T) \cong \mathbf{Z} \oplus \mathbf{Z}$ and $\sigma$ as described previously is one of the two generators.
