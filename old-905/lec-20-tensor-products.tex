\section{$\bigotimes$}
Welcome to algebraic topology! This is family weekend, so welcome. Today'll be more about algebra, and there'll be very little topology, I'm afraid. Today'll be about tensor products. I got your permission to talk about modules over a commutative ring. We're always going to let $R$ be a commutative ring (they're going to be simple; for example, $\QQ,\FF_p,\Z,\Z/n\Z,\cdots,\text{PIDs}$).

I want to tell you that the category of $R$-modules is what's called a ``categorical ring'', where the addition corresponds to the direct sum, the zero element is the zero module, $1$ is $R$ itself, and multiplication is where you put a circle around a multiplication symbol.

The reason we do this is because of bilinear maps. Let me recall the definition of a bilinear map.
\begin{definition}
If I have $M,N,P$ are $R$-modules, then a bilinear (or if you want to be annoying, $R$-bilinear) map is a map $\beta:M\times N\to P$ such that $\beta(x+x^\prime,y)=\beta(x,y)+\beta(x^\prime,y)$ and $\beta(x,y+y^\prime)=\beta(x,y)+\beta(x,y^\prime)$, and such that $\beta(rx,y)=r\beta(x,y)$ and $\beta(x,ry)=r\beta(x,y)$.
\end{definition}
\begin{example}
$\RR^n\times\RR^n\to\RR$ given by the dot product is a $\RR$-bilinear map. The cross product $\RR^3\times\RR^3\to\RR$ is $\RR$-bilinear. More generally, if $R$ is a ring then the multiplication $R\times R\to R$ is $R$-bilinear, and the multiplication on an $R$-module $M$ given by $R\times M\to M$ is $R$-bilinear. This enters into topology because the map $ H_n(X;R)\times H_n(Y;R)\xrightarrow{\times} H_{m+n}(X\times Y;R)$ is $R$-bilinear.
\end{example}
Wouldn't it be great to reduce stuff about bilinear maps to linear maps? We're going to do this by means of the universal property.
\begin{definition}
Let $M,N$ be $R$-modules. A \emph{tensor product} of $M,N$ is a $R$-module $P$ and a bilinear map $M\times N\xrightarrow{\beta_0}P$ such that for every bilinear map $M\times N\xrightarrow{\beta}Q$ there is a unique factorization.
\begin{equation*}
\xymatrix{M\times N\ar[r]^{\beta_0}\ar[dr]^\beta & P\ar@{-->}[d]^f\\
 & Q}
\end{equation*}
through an $R$-module homomorphism $f$. It's easy to check that $f\circ\beta_0$ is bilinear.
\end{definition}
So $\beta_0$ is universal bilinear map out of $M\times N$. Instead of $\beta_0$ we're going to write $M\times N\xrightarrow{\otimes}P$. This means that $\beta(x,y)=f(x\otimes y)$ in the above diagram. There are lots of things to say about this. When you have something that is defined via a universal property, you first have to check that it exists!
\begin{construction}
I want to construct an $R$-bilinear map out of $M\times N$. I guess I should say it like this. Let $\beta:M\times N\to Q$ be any $R$-bilinear map. This $\beta$ isn't linear. Maybe we should first extend it to a linear map. Consider $R\langle M\times N\rangle$, the free $R$-module generated by $M\times N$. Well, $\beta$ is a map of sets, so there's a unique $R$-linear homomorphism $\overline{\beta}:R\langle M\times N\rangle\to Q$. Then I get a factorization:
\begin{equation*}
\xymatrix{M\times N\ar[rr]^\beta\ar[dr]^{[-]} & & Q\\
& R\langle M\times N\rangle\ar[ur]^{\overline{\beta}} &}
\end{equation*}
The map $[-]$ isn't bilinear. So we should quotient $R\langle M\times N\rangle$ by a submodule $S$ of relations. More precisely, $S$ is the sub $R$-module generated by the relations needed to map $[-]$ a $R$-bilinear map, namely:
\begin{enumerate}
\item $[(x+x^\prime,y)]-[(x,y)]-[(x^\prime-y)]$.
\item $[(x,y+y^\prime)]-[(x,y)]-[(x,y^\prime)]$.
\item $[(rx,y)]-r[(x,y)]$.
\item $[(x,ry)]-r[(x,y)]$
\end{enumerate}
for all $x,x^\prime\in M$ and $y,y^\prime\in N$. Now, this map $[-]$ is bilinear - we've quotiented out by all things that made it false! Now the map $R\langle M\times N\rangle\to Q$ factors through via $R\langle M\times N\rangle\to R\langle M\times N\rangle/S\xrightarrow{f} Q$ because the map $\overline{\beta}$ is linear, and $f$ is unique because the $\overline{\beta}$ is unique, so there's at most one factorization. We just checked that there was one, so we're done. We'll also write the composition $M\times N\xrightarrow{[-]}R\langle M\times N\rangle\to R\langle M\times N\rangle/S$ as $\otimes$.
\end{construction}
You're never going to use this construction to compute anything. If you find yourself using this construction, stop and think about what you're doing.
\begin{remark}
Note that the image of $(m,n)$ in $R\langle M\times N\rangle/S$ generates $R\langle M\times N\rangle/S$ as an $R$-module. The $R$-module $R\langle M\times N\rangle/S$ contains elements of the form $x\otimes y$ with $x\in M$ and $y\in N$ because they generate $R\langle M\times N\rangle$, and $R\langle M\times N\rangle/S$ is a quotient of that.

These $x\otimes y$ are called ``decomposable tensors''. (I've heard them called pure tensors.)
\end{remark}
What are the properties of $R\langle M\times N\rangle/S=:P$?
\begin{enumerate}
\item How many maps are there that make the following diagram commute?
\begin{equation*}
\xymatrix{& P\ar@{-->}[dd]\\
M\otimes N\ar[ur]^\otimes\ar[dr]_\otimes & \\
& P}
\end{equation*}
By the uniqueness statement, there's only one map, namely the identity!
\item Suppose that we have two tensor products of $M$ and $N$, say $P$ and $P^\prime$. We have
\begin{equation*}
\xymatrix{& P\ar@{-->}[dd]^b\\
M\otimes N\ar[ur]^\otimes\ar[dr]_\otimes & \\
& P^\prime\ar@{-->}[uu]_{b^\prime}}
\end{equation*}
And $b,b^\prime$ are unique. If you compose $b$ and $b^\prime$, you'll see that you get the identity of $P$ and $P^\prime$, depending on how you compose the maps. More precisely, you have:
\begin{equation*}
\xymatrix{& P\ar@{-->}[d]^b\\
M\otimes N\ar[ur]^\otimes\ar[dr]_\otimes & P^\prime\ar[d]^{b^\prime}\\
& P^\prime}
\end{equation*}
and
\begin{equation*}
\xymatrix{& P^\prime\ar@{-->}[d]^{b^\prime}\\
M\otimes N\ar[ur]^\otimes\ar[dr]_\otimes & P\ar[d]^{b}\\
& P^\prime}
\end{equation*}
Thus $bb^\prime=1$ and $b^\prime b=1$. So $b,b^\prime$ are isomorphisms, i.e., $P\cong P^\prime$. We say that there is a canonical\footnote{This means god given, but here it means that it's naturally constructed.} isomorphism between any two constructions of a tensor products. The universal property defines the object up to canonical isomorphism. This is a general principle.

We can thus write the tensor product as if it just depended on just $M$ and $N$. We write $M\otimes N$. A general element is a finite sum $\sum_i x_i\otimes y_i$. To be really honest, we'll write $M\otimes_R N$. If $R$ is understood, we'll omit it. I'll usually forget to add the $\otimes_R$, and simply write $\otimes$.
\item Functoriality. If I have homomorphisms $M\times N\xrightarrow{f\times g}M^\prime\times N^\prime$. I have:
\begin{equation*}
\xymatrix{M\times N\ar[d]^{f\times g}\ar[r]^\otimes\ar[dr] & M\otimes N\ar@{-->}[d]\\
M^\prime\times N^\prime\ar[r]^\otimes & M^\prime\otimes N^\prime}
\end{equation*}
The dotted map exists because the diagonal map is $R$-bilinear. We write the map $M\otimes N\to M^\prime \otimes N^\prime$ as $f\otimes g$. We need to check stuff though.
\begin{equation*}
\xymatrix{M\times N\ar[d]^{f\times g}\ar[r]^\otimes\ar[dr] & M\otimes N\ar@{-->}[d]^{f\otimes g}\\
M^\prime\times N^\prime\ar[r]^\otimes\ar[d]^{f^\prime\times g^\prime} & M^\prime\otimes N^\prime\ar@{-->}[d]^{f\otimes g}\\
M^{\prime\prime}\times N^{\prime\prime}\ar[r]^\otimes & M^{\prime\prime}\otimes N^{\prime\prime}}
\end{equation*}
And the composite matches up, i.e., $(f^\prime\otimes g^\prime)(f\otimes g)=(f^\prime f)\otimes g^\prime g$.
\item I said that this was gonna be a categorical ring, so we need to check this. Well, $R\otimes_R M$ should be isomorphic to $M$. Let's think about this for a minute. I just need to check the universal property. Suppose I have an $R$-bilinear map $\beta:R\times M\to P$. We already have a universal $R$-bilinear map $\varphi:R\times M\to M$. I have to construct a universal factorization $f:M\to P$. Just let $f(x)=\beta(1,x)$. It's $R$-bilinear. We can check that this diagram commutes now because $f(\varphi(r,x))=f(rx)=\beta(1,rx)=r\beta(1,x)=\beta(r,x)$. Well, this map $R\times M\to M$ is surjective, so there's at most one factorization. So we're done. There are other checks that are extremely boring, but they're part of the toolkit.

I need to check that $L\otimes(M\otimes N)\cong (L\otimes M)\otimes N$ that's compatible with $L\times (M\times N)\cong (L\times M)\times N$. There's a canonical isomorphism. I don't know how to not say that this is trivial. Also, we need to check that $M\otimes N\cong N\otimes M$. (Just do this yourself. It's really easy.)
\item What happens with $M\otimes\left(\bigoplus_{\alpha\in A}N_\alpha\right)$? It might be a finite direct sum, or maybe an uncountable collection. How does this relate to $\bigoplus_{\alpha\in A}(M\otimes N_\alpha)$? Let's construct a map $\displaystyle\bigoplus_{\alpha\in A}(M\otimes N_\alpha)\to M\otimes\left(\bigoplus_{\alpha\in A}N_\alpha\right)$. We just need to define maps $M\otimes N_\alpha\to M\otimes\left(\bigoplus_{\alpha\in A}N_\alpha\right)$ because direct sums are coproducts. Let this map be $1\otimes\text{in}_\alpha$ where $\mathrm{in}_\alpha:N_\alpha\to \bigoplus_{\alpha\in A}N_\alpha$. These give you a map $f:\bigoplus_{\alpha\in A}(M\otimes N_\alpha)\to M\otimes\left(\bigoplus_{\alpha\in A}N_\alpha\right)$

What about a map the other way? This is a bit trickier. An element of $M\otimes\left(\bigoplus_{\alpha\in A}N_\alpha\right)$ is $x\otimes(y_\alpha)_{\alpha\in A}$, where you note that $y_\alpha=0$ for all but finitely many $\alpha\in A$. Define $g:M\otimes\left(\bigoplus_{\alpha\in A}N_\alpha\right)\to \bigoplus_{\alpha\in A}(M\otimes N_\alpha)$ via $x\otimes(y_\alpha)_{\alpha\in A}\mapsto (x\otimes y_\alpha)_{\alpha\in A}$. It's up to you to check that these are inverses and that you can extend to a general nondecomposable tensor by linearity.
\end{enumerate}
We have not done any computations yet. I guess I should end with the statement that $S_\ast(X;M):=S_\ast(X)\otimes_R M$ if $M$ is an $R$-module. We'll discuss on Monday the question we raised last time, namely:
\begin{question}
How is $ H_\ast(X;M)$ related to $ H_\ast(X)= H_\ast(X;\Z)$? This is a reasonable question.
\end{question}
