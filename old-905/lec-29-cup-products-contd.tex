\section{Cup product, continued}
We can construct an explicit map $S^p(X)\otimes S^q(Y)\xrightarrow{\times} S^{p+q}(Y)$ via:
\begin{equation*}
(f\times g)(\sigma)=f(\sigma_1\circ\alpha_p)g(\sigma_2\circ\omega_q)
\end{equation*}
where $\alpha_p:\Delta^p\to\Delta^{p+q}$ takes $k\mapsto k$ where $k\in[p]$, and $\omega_q:\Delta^q\to\Delta^{p+q}$ sends $\ell\mapsto \ell+p$ where $\ell\in[q]$.

I would like to modify this definition a bit, and correct what I said. I think it's more natural to put a sign so that:
\begin{equation*}
(f\times g)(\sigma)=(-1)^{pq}f(\sigma_1\circ\alpha_p)g(\sigma_2\circ\omega_q)
\end{equation*}
You'll see why we're adding the sign soon. The claim is that $\times$ is a chain map, so we get a map $ H(S^p(X)\otimes S^q(Y))\to H_{p+q}(X\times Y)$. There is a map $ H_p(X)\otimes H_q(Y)\xrightarrow{\mu} H(S^p(X)\otimes S^q(Y))$ defined as follows. Given chain complexes $C_\bullet$ and $D_\bullet$, define $\mu: H(C_\bullet)\otimes H(D_\bullet)\to H(C_\bullet\otimes D_\bullet)$. We've done this before, but let me just say: $\mu:[x]\otimes [y]\mapsto [x\otimes y]$. For some reason, he said that we'll not call this $\mu$. But I don't want to delete the $\mu$ so I'll just leave it there.

We get the cup product as follows. Given $\Delta:X\to X\times X$, we get $\cup: H_p(X)\otimes H_q(X)\to H_{p+q}(X\times X)\xrightarrow{\Delta^\ast} H_{p+q}(X)$. We proved that there was a class $1\in H_0(X)=\Map(\pi_0(X),R)\ni (\alpha\mapsto 1)$. This acts as a unit for the cup product. Also, the cup product is strictly associative.
\begin{definition}
Let $R$ be a commutative ring. A \emph{graded $R$-algebra} is a graded $R$-module $\cdots,A_{-1},A_0, A_1,A_2,\cdots$ (integer graded sequence) (some people take a direct sum, but I find no reason for doing that) with maps $A_p\otimes_R A_q\to A_{p+q}$ and a map $R\to A_0$ (that determines the unit), that makes the following diagram commute.
\begin{equation*}
\xymatrix{
	A_p\otimes_R (A_q\otimes_R A_r)\ar[r]\ar[d] & A_p\otimes_R A_{q+r}\ar[d]\\
	(A_{p+q})\otimes_R A_r \ar[r] & A_{p+q+r}
}
\end{equation*}
\end{definition}
\begin{definition}
A graded $R$-algebra $A$ is said to be (graded) commutative id the following diagram commutes:
\begin{equation*}
\xymatrix{
	x\otimes y\ar@{|->}[rr] & & (-1)^{|x|\cdot|y|}y\otimes x\\
	A\otimes A\ar[rr]^{\tau}\ar[dr] & & A\otimes A\ar[dl]\\
	 & A & 
}
\end{equation*}
\end{definition}
We claim that $ H_\ast(X)$ forms a graded commutative ring under the cup product. This is nontrivial. On the cochain level, this is clearly not graded commutative. We're going to have to work hard -- in fact, so hard that you're going to do some of it for homework.

We'll do a chain level construction. I say there's a map $\alpha:S_n(X\times Y)\to \bigoplus_{p+q=n}S_p(X)\otimes S_n(Y)$. First I'll tell you what happens to $n$-simplices $\sigma:\Delta^n\to X\times Y$. Let $\sigma_1:\Delta^n\to X\times Y\to X$, and similarly for $\sigma_2$. Define $S_n(X\times Y)\xrightarrow{\alpha} \bigoplus_{p+q=n}S_p(X)\otimes S_n(Y)$ by sending:
\begin{equation*}
\sigma\mapsto\sum_{p+q=n}(\sigma_1\circ\alpha_p)\otimes(\sigma_2\otimes\omega_q)
\end{equation*}
We claim that this is a chain map, and that this induces the cross product on cochains.
\begin{remark}
This map $\alpha$ is called the Alexander-Whitney map.
\end{remark}
If I have two chain complexes, I want to consider a map $\mu:\Hom(C_\bullet,R)\otimes \Hom(D_\bullet,R)\to \Hom(C_\bullet\otimes D_\bullet,R)$ given by $f\otimes g\mapsto(x\otimes y\mapsto (-1)^{pq}f(x)g(y))$ where $|f|=|x|=p$ and $|g|=|y|=q$. I haven't quite done the right thing here, have I? I have to write:
\begin{equation*}
f\otimes g\mapsto\begin{cases}
(x\otimes y\mapsto (-1)^{pq}f(x)g(y)) & |x|=|f|=p, |y|=|g|=q\\
0 & \text{else}
\end{cases}
\end{equation*}
You should check that this is a chain map.

Recall that we have: $S^p(X)\otimes S^q(Y)=\Hom(S_p(X),R)\otimes_R \Hom(S_q(Y),R)$. The map $\mu$ we constructed just now gives a map $\Hom(S_p(X),R)\otimes_R \Hom(S_q(Y),R)\xrightarrow{\mu}\Hom(S_p(X)\otimes S_q(Y),R)\xrightarrow{\alpha}\Hom(S_{p+q}(X\times Y),R)=S^{p+q}(X\times Y)$. This is exactly the cross product $\times:S^p(X)\otimes S^q(Y)\to S^{p+q}(X\times Y)$.

Now, $\alpha$ is a natural transformation. Acyclic models comes into play. For homework, you're going to check that the following diagram commutes.
\begin{equation*}
\xymatrix{S_\ast(X\times Y)\ar[r]^{T_\ast}\ar[d]_{\alpha_{X,Y}} & S_\ast(Y\times X)\ar[d]^{\alpha_{Y,X}}\\
S_\ast(X)\otimes_R S_\ast(Y)\ar[r]^{\tau} & S_\ast(Y)\otimes_R S_\ast(X)}
\end{equation*}
Acyclic models helps us prove things like this.

All of this implies that $ H_\ast(X;R)$ is graded commutative. It's a theorem that you can't find a commutative multiplication. This is where Steenrod operations come from. They're called cohomology operations, and we'll talk more about this in 18.906.

My goal is to compute the cohomology of some space.
\begin{prop}
$ H^\ast(X)\otimes H^\ast(Y)\xrightarrow{\times} H^\ast(X\times Y)$ is a $R$-algebra homomorphism.
\end{prop}
If $A$ and $B$ are graded $R$-algebras, then $(A\otimes B)_n=\bigoplus_{p+q=n}A_p\otimes_R B_q$, and $(a\otimes b)(a^\prime\otimes b^\prime)=(-1)^{|a^\prime|\cdot|b|}aa^\prime\otimes bb^\prime$. This tensor product is graded commutative if $A$ and $B$ are.
\begin{proof}
I have $\Delta_X:X\to X\times X$ and $\Delta_Y:Y\to Y\times Y$. I also have $\Delta_{X\times Y}:X\times Y\to X\times Y\times X\times Y$, which factors as $(1\times T\times 1)\circ(\Delta_X\times \Delta_Y)$. Let $\alpha_1,\alpha_2\in H^\ast(X)$ and $\beta_1,\beta_2\in H^\ast(Y)$. Then $\alpha_1\times \beta_1,\alpha_2\times\beta_2\in H^\ast(X\times Y)$. I want to calculate what $(\alpha_1\times\beta_1)\cup(\alpha_2\times\beta_2)$ is. Let's see:
\begin{align*}
(\alpha_1\times\beta_1)\cup(\alpha_2\times\beta_2) & = \Delta_{X\times Y}^\ast(\alpha_1\times\beta_1\times\alpha_2\times\beta_2)\\
& = (\Delta_X\times\Delta_Y)^\ast(1\times T\times 1)^\ast(\alpha_1\times\beta_1\times\alpha_2\times\beta_2)\\
& = (\Delta_X\times\Delta_Y)^\ast(\alpha_1\times T^\ast(\beta_1\times\alpha_2)\times\beta_2)\\
& = (-1)^{|\alpha_2|\cdot|\beta_1|}(\Delta_X\times\Delta_Y)^\ast(\alpha_1\times\alpha_2\times\beta_1\times\beta_2)
\end{align*}
Now, I have a diagram:
\begin{equation*}
\xymatrix{
	 H^\ast(X\times Y) & \ar[l]^{\times_{X\times Y}} H^\ast(X)\otimes_R H^\ast(Y)\\
	 H^\ast(X\times X\times Y\times Y)\ar[u]^{(\Delta_X\times\Delta_Y)^\ast} & H^\ast(X\times X)\otimes H^\ast(Y\times Y)\ar[l]_{\times_{X\times X,Y\times Y}}\ar[u]^{\Delta_X^\times\otimes\Delta_Y^\ast}
}
\end{equation*}
This diagram commutes because the cross product is natural. This is exactly what commutativity of the diagram means. This means that:
\begin{align*}
(\alpha_1\times\beta_1)\cup(\alpha_2\times\beta_2) & = (-1)^{|\alpha_2|\cdot|\beta_1|}(\Delta_X\times\Delta_Y)^\ast(\alpha_1\times\alpha_2\times\beta_1\times\beta_2)\\
& = (-1)^{|\alpha_2|\cdot|\beta_1|}(\alpha_1\cup\alpha_2)\times(\beta_1\cup\beta_2)
\end{align*}
That's exactly what we wanted.
\end{proof}
\begin{example}
How about $ H^\ast(S^p)$? Let $p>0$. This is: $ H^k(S_p) = \begin{cases}\Z & k=0,p\\ 0 & \text{else}\end{cases}$. Say that $\sigma_p$ generates $ H^p(S^p)$. We now have $ H^\ast(S^p)\otimes H^\ast(S^q)\to H^\ast(S^p\times S^q)$. From the K\"unneth theorem, we know that $ H^k(S^p\times S^q)=\begin{cases}\Z & k=0,p,q,p+q\\ 0 & \text{else}\end{cases}$. This structure $ H^\ast(S^p)\otimes H^\ast(S^q)\cong\Z[\sigma_p,\sigma_q]/(\sigma_p^2,\sigma_q^2)\cong H^\ast(S^p\times S^q)$ where $\sigma_p\sigma_q=(-1)^{|p|\cdot|q|}\sigma_q\sigma_p$ where by $\Z[\sigma_p,\sigma_q]$ I mean the free $\Z$-algebra on $\sigma_p,\sigma_q$. When $p=q=1$, this gives an exterior algebra.
\end{example}
This is to be contrasted with $X=S^p\vee S^q\vee S^{p+q}$. This has the same homology as the product of two spheres. But in cohomology, the product of the generators must be zero. There is a diagram:
\begin{equation*}
\xymatrix{
	X\ar[r] & S^p\vee S^q\\
	S^p\ar@{^(->}[u]
}
\end{equation*}
On cohomology, notice that $\sigma_p\sigma_q=0$ in $ H^\ast(S^p\vee S^q)$ because there's no $(p+q)$-dimensional cohomology. Therefore we find that $S^p\vee S^q\vee S^{p+q}\not\simeq S^p\times S^q$.

It's very interesting to think about what the attaching map is:
\begin{equation*}
\xymatrix{
	S^{p+q-1}\ar[r]\ar[d] & S^p\vee S^q\ar[d]\\
	D^{p+q}\ar[r] & S^p\times S^q
}
\end{equation*}
On Wednesday we'll have a relaxed talk about surfaces. BTW I might not TeX that.
