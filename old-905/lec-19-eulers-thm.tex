\section{``Euler's theorem'', and ``homology approximation'' - CTC Wall. I) Singular homology, II) CW complexes, III) ``homological algebra''}
\begin{theorem}[``Euler'']
Let $X$ be a space which admits the structure of a finite CW complex. The sum $\sum_{h=0}^\infty (-1)^k\#(k\text{-cells})$ (generalizes $V-E+F$) is independent of that structure.
\end{theorem}
\begin{proof}
Pick a CW-structure. We have $0\to C_n\to\cdots\to C_2\to C_1\to C_0\to 0$. We also have a sexseq $0\to Z_k\to C_k\to B_{k-1}\to 0$, and another one $0\to B_k\to Z_k\to H_k\to 0$. Let's use them and facts about rank that I talked about on Monday to compute what this alternating sum is. The Euler sum is the same as:
\begin{align*}
\sum_{h=0}^\infty (-1)^k\#(k\text{-cells}) & = \sum^\infty_{k=0}(-1)^k\rank(C_k)\\
& = \sum^\infty_{k=0}(-1)^k\rank(Z_k)+\sum^\infty_{k=0}(-1)^k\rank(B_{k-1})\\
& = \sum^\infty_{k=0}(-1)^k(\rank( H_k)+\rank(B_k)+\rank(B_{k-1}))
\end{align*}
The terms $\rank B_k+\rank B_{k+1}$ telescope because it's an alternating sum, and hence vanish. The sum is $\sum^\infty_{k=0}(-1)^k\rank( H_k)$. But $ H_k(X)= H_k^\text{sing}(X)$ is an invariant of the space, independent of the CW-structure.
\end{proof}
Given $ H_k(X)$, $X$ a finite type CW-complex, what's a lower bound on the number of $k$-cells? Let's see. $ H_k(X)$ is finitely generated because $C_k(X)\subseteq Z_k(X)$ is, and it surjects onto $ H_k(X)$. Thus $ H_k(X)=\bigoplus^{t(k)}_{i=1}\Z/n_i(k)\Z\oplus \Z^{r(k)}$ where the $n_1(k)|\cdots|n_{t(k)}(k)$ are the torsion indices.

The minimal chain complex with $ H_k=\Z$ and $ H_q=0$ for $q\neq k$ is just the chain complex with $0$ everywhere else except for $\Z$ in the $k$th degree. The minimal chain complex with $ H_k=\Z/n\Z$ and $ H_q=0$ for $q\neq k$ is just the chain complex with $0$ everywhere else except for $\Z\xrightarrow{n}\Z$ in dimension $k+1$ to $k$. These things are called elementary chain complexes.

A lower bound on the minimal number of $k$-cells is $r(k)+t(k)+t(k-1)$ where the last term comes for the ``torsion generator in dimension $k-1$'' (didn't catch that).
\begin{theorem}[Wall]
Let $X$ be a simply connected CW-complex of finite type. Then there exists a CW complex $Y$ with $r(k)+t(k)+t(k-1)$ $k$-cells, for all $k$, and a homotopy equivalence $Y\to X$.
\end{theorem}
I'm not going to prove this theorem. You can read Wall's theorem. You really can't ask for more. Oh, also here's a theorem.
\begin{theorem}
Let $X$ be connected and pointed $\ast\in X$. Then $\pi_1(X,\ast)\to H_1(X,\ast)$ exists, called the Hurewicz homomorphism, and it factors as $\pi_1(X,\ast)\to \pi_1(X,\ast)^{ab}\to H_1(X,\ast)$. The last map is an isomorphism.
\end{theorem}
Some examples of Wall's theorem:
\begin{example}
We know that $S^k$ has $\widetilde{ H}_q(X)=\Z$ when $q=k$ and $0$ else. Can you construct a space with $\widetilde{ H}_q(X)=\Z/n\Z$ when $q=k$ and $0$ else? We need to construct a space with the elementary chain complex with $0$ everywhere else except for $\Z\xrightarrow{n}\Z$ in dimension $k+1$ to $k$. You need to have one $0$-cell, do nothing until you get to dimension $k$, which is when you add a $k$-cell, and then use the attaching map $S^k\to S^k$ of degree $n$, i.e.:
\begin{equation*}
\xymatrix{S^k\ar[r]^{\text{degree }n}\ar[d] & S^k\ar[d]\\
D^{k+1}\ar[r] & X}
\end{equation*}
For example, when $k=1$ and $n=2$, you have $\RP^2$. This is called a ``Moore space''.
\end{example}
I brought up doing this in more generality with generators and relations, and Professor Miller built up on that:
\begin{example}
For more general abelian groups, you have a free abelian group $F_0$ sitting in a sexseq $0\to F_1\to F_0\to M\to 0$ (this is an example of a \emph{resolution of $M$}, which is what I'm going to start talking about). Then $F_1$ is also free. Pick some $k>0$. You get a space whose homology is $F_0$, namely $\bigvee_\alpha S^k$, and a space whose homology is $F_1$, namely $\coprod S^k$. You can construct a map $\coprod S^k\to \bigvee_\alpha S^k$ such that the map $\alpha:F_1\to F_0$ is what's induced on homology. Then you get:
\begin{equation*}
\xymatrix{\coprod S^k\ar[r]^{\text{gives }\alpha}\ar[d] & \bigvee_\alpha S^k\ar[d]\\
\coprod D^{k+1}\ar[r] & X}
\end{equation*}
Such an $X$ is called a Moore space, and has homology $M$ in dimension $k$ and zero everywhere else. You can't make this into a functor, i.e., this can't be made into a functor $\mathbf{Ab}\to\mathbf{Top}$.
\end{example}
\subsection{Homological algebra}
You can put coefficients into homology. Let $M$ an abelian group. You can talk about homology with coefficients in $M$. For example, $M=\Z,\QQ,\Z/n\Z,\cdots$. The $\Z/n\Z$ case when $n$ is prime is pretty important because it's then a field.

Given $X$, you get a singular simplicial set $\Sin_\ast(X)$. Then we took the free abelian group $S_\ast$ generated by $\Sin_\ast(X)$. I.e., $S_n=\Z[\Sin_n(X)]=\bigoplus_{\Sin_n(X)}\Z$. But I could replace $\Z$ with anything I wanted, and do the \emph{exact} same construction. I can just as well as put any abelian group here.

Define the ``singular chain complex with coefficients in $M$'' as $S_n(X;M)=\bigoplus_{\Sin_n(X)}M$. There's a boundary map $d:S_n(X;M)\to S_{n-1}(X;M)$. Then the homology $ H(S_\ast(X;M))=: H_\ast(X;M)$. You can verify all the Eilenberg-Steenrod axioms yourself, except for one, namely the dimension axiom - $ H_k(\ast;M)=\begin{cases}M & k=0 \\ 0 & k\neq 0\end{cases}$.

If you think about it, you'll realize that this whole unit in CW-complexes didn't use anything except for the Eilenberg-Steenrod axioms. This shows, by the way, that if you get some weird homology theory satisfying the Eilenberg-Steenrod axioms you get all the same results as if you used what we constructed before.

As an experiment, let's compute $ H_\ast(\RP^n;\Z/2\Z)$. The cellular chain complex is $0\to \Z/2\Z\to\cdots\to\Z/2\Z\to\Z/2\Z\to 0$ where the maps are alternately multiplying by $2$ and $0$. But in this case, all the maps are $0$ because $2=0$! So $ H_k(\RP^n;\Z/2\Z)=\begin{cases}\Z/2\Z & 0\leq k\leq n \\ 0 & \text{else}\end{cases}$. How about $ H_\ast(\RP^n;\QQ)$? Or $ H_\ast(\RP^n;\Z[\frac{1}{p}])$ where $\Z[\frac{1}{p}]\subseteq \QQ$? What about $\Z_{(p)}\subseteq \QQ$ where you've localized at $p$?

Anyway, if I consider $ H_\ast(\RP^n;\Z[\frac{1}{2}])$, then the cellular chain complex simplifies, but in a different way. You have $0\to \Z[\frac{1}{2}] \to\cdots\to \Z[\frac{1}{2}] \to \Z[\frac{1}{2}] \to 0$. Multiplication by $2$, however, is an isomorphism. So, $ H_k(\RP^n;\Z[\frac{1}{2}])=\begin{cases}\Z[\frac{1}{2}] & q=0,n,\, q \text{ odd} \\ 0 & \text{else}\end{cases}$. You get a much simpler result. From this point of view, even projective spaces look like a point, and odd projective spaces look like a sphere!

It's a little awkward to go through this thing. I'd like to understand:
\begin{question}
How is $ H_\ast(X;M)$ related to $ H_\ast(X)= H_\ast(X;\Z)$? This is a reasonable question.
\end{question}
The answer is called the ``universal coefficient theorem''. I'll spend a few days developing what we need to talk about this.

I want to talk about tensor products. Let me take a poll. Do you know tensor products? Working with a commutative ring instead of $\Z$? Actually, all of these examples $M=\Z,\QQ,\Z/n\Z,\Z[\frac{1}{p}]\subseteq\QQ\supseteq\Z_{(p)},\cdots$ are rings. The boundary map $d:S_n(X;M)\to S_{n-1}(X;M)$ is a module homomorphism if $M=R$ is a (\emph{always commutative}) ring.

This means that if $R$ is a commutative ring, then $ H_\ast(X;R)$ is an $R$-module. If $R$ is a ring and $M$ is an $R$-module, then $ H_\ast(X;M)$ is an $R$-module. Just look at what you have here. The $\bigoplus_{\Sin_n(X)}M$ is an $R$-module, and $d$ is an $R$-module homomorphism.

I'll admit, this is a little bit scary, because commutative rings are pretty complicated in general. I won't talk about some weirdo rings, though. I'll develop this more on Friday. Let me pass out homework.
