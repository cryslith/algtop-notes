\section{Finish off the proof of $\cHH^p$ excision, topological manifolds, fundamental classes}
\subsection{The end of the proof}
Let's finish off the proof from last time. Suppose $A,B$ are closed in normal $X$. \underline{Excision for $\cHH^p$}:
\begin{equation*}
\xymatrix{
	\varinjlim_{(W,Y)\in\mathcal{U}_A\times\mathcal{U}_B}H^p(W\cup Y,Y)\ar[rrr]^{\cong,\text{ ordinary excision}}\ar[d]^{\cong,\text{ cofinality/surjectivity}} & & & \varinjlim_{\mathcal{U}_A\times\mathcal{U}_B}H^p(W,W\cap Y)\ar[d]^{\cong,\text{ cofinality, see below}}\\
	\varinjlim_{(U,V)\in\mathcal{U}_{A\cup B,B}}H^p(U,V)\ar[rrr]\ar@{=}[d] & & & \varinjlim_{(U,V)\in\mathcal{U}_{A,A\cap B}}H^p(U,V)\ar@{=}[d]\\
	\cHH^p(A\cup B,B)\ar[rrr] & & & \cHH^p(A,A\cap B)
}
\end{equation*}

$\mathcal{U}_A\times\mathcal{U}_B\to\mathcal{U}_{A,A\cap B}$ is cofinal since: start with $(U,V)\supseteq(A,A\cap B)$. Using normality, separate $B\cap(X-V)\subseteq T$ and $A\subseteq S$. Take $W=U\cap S$ and $Y=V\cup T$. Then $A\subseteq W\subseteq U$ and $A\cap B\subseteq W\cap Y=S\cap V\subseteq V$.

This means that $\cHH^p$ satisfies excision, hence Mayer-Vietoris. Let's put this in the drawer for now.
\subsection{Topological manifolds + Poincar\'{e} duality}
yayyyyyyyyyyyyy finally
\subsubsection{Fundamental class and orientation local system}
\begin{definition}
A \emph{topological manifold} is a Hausdorff space $M$ such that for every $x\in M$, there exists a neighborhood $U\ni x$ that is homeomorphic to some Euclidean space $\RR^n$. It's called an $n$-manifold if all $U$ are homeomorphic to $\RR^n$ for the \emph{same} $n$.
\end{definition}
\begin{example}
$\RR^n$, duh. $\emptyset$ is an $n$-manifold for every $n$. The sphere $S^n$. The Grassmannian $\mathrm{Gr}_k(\RR^n)$, introduced in the beginning of the course. I don't know exactly what the dimension of this is, but you can figure it out. Also, $V_k(\RR^n)$, and surfaces.
\end{example}
These things are the most interesting things to look at.
\begin{warning}
We assume the following.
\begin{enumerate}
\item There exists a countable basis.
\item There exists a good cover, i.e., all nonempty intersections are Euclidean as well (always true for differentiable manifolds because you can take geodesic neighborhoods, and in particular for the manifolds we listed above).
\end{enumerate}
\end{warning}
This is the context in which duality works.
\begin{definition}
Let $X$ be any space, and let $a\in X$. The local homology of $X$ at $a$ is the homology $H_\ast(X,X-a)$. We're always working over a commutative ring.
\end{definition}
For example, $H_q(\RR^n,\RR^n-0)=\begin{cases}\text{free of rank }1 & q=n \\ 0 & q\neq n\end{cases}$. This means that local homology is picking out the characteristic feature of Euclidean space. Therefore we also have $H_q(M,M-a)=\begin{cases}\text{free of rank }1 & q=n \\ 0 & q\neq n\end{cases}$ for $n$-manifolds.
\begin{notation}
Let $j_a:(M,\emptyset)\to (M,M-a)$ be the inclusion.
\end{notation}
\begin{definition}
A fundamental class for $M$ (an $n$-manifold) is $[M]\in H_n(M)$ such that for every $a\in M$, the image of $[M]$ under $j_{a,\ast}:H_n(M)\to H_n(M,M-a)$ is a generator of $H_n(M,M-a)$.
\end{definition}
This is somehow trying to say that this class $[M]$ covers the whole manifold.
\begin{example}
When does a space have a fundamental class?
\begin{center}
\begin{tabular}{c|c c c } 
 \hline
  & $\RR^2$ & $\RP^2$ & $T^2$ \\ 
  \hline
 $R=\Z$ & no! & no! & yes! you did this for homework \\
 $R=\Z/2\Z$ & no! & yes! & yes!
\end{tabular}
\end{center}
Something about orientability and compactness seem to be involved.
\end{example}
What do we have? 
\begin{definition}
$o_M=\coprod_{a\in M}H_n(M,M-a)$ as a set. This has a map $p:o_M\to M$.
\end{definition}
\begin{construction}
This can be topologized in Euclidean neighborhoods. Let $U\cong\RR^n$ be an Euclidean neighborhood of $a$. I can always arrange so that $a$ corresponds to $0$. We have the open disk sitting inside the closed disk: $\widetilde{D^n}\subseteq D^n\subseteq \RR^n$ that corresponds to some open $V\subseteq \overline{V}\subseteq U$. Let $x\in V$. I have a diagram:
\begin{equation*}
\xymatrix{
	H_n(M,M-\overline{V})\ar[d] & & H_n(U,U-\overline{V})\ar[ll]^{\text{ excision of }M-U}_{\cong}\ar[d]^\cong & H_n(\RR^n,\RR^n-D^n)\ar@{=}[l]\ar[d]^{\cong,\text{ homotopy equivalence}}\\
	H_n(M,M-x) & & H_n(U,U-x)\ar[ll] & H_n(\RR^n,\RR^n-0)\ar@{=}[l]
}
\end{equation*}
Hence $H_n(M,M-\overline{V})\cong H_n(M,M-x)$. Thus I can collect points in $o_M$ together when they come from the same class in $H_n(M,M-\overline{V})$, so they form ``sheets''.

I have a map $V\times H_n(M,M-\overline{V})\to o_M|_{V}=p^{-1}(V)$ by sending $(x,c)\mapsto (j_x)_\ast(c)\in H_n(M,M-x)$, and this map is bijective (that's what comes from excision). This LHS has a nice topology by letting $H_n(M,M-\overline{V})$ be discrete. I'm topologizing $o_M$ as the weakest topology these generate.
\end{construction}
``Have I been sufficiently obscure enough? This is not supposed to be a complicated point''. This $o_M\to M$ is called the \emph{orientation local system}, and is a covering space.
\begin{definition}
A continuous map $p:E\to B$ is a covering space if:
\begin{enumerate}
\item $p^{-1}(b)$ is discrete for all $b\in B$.
\item For every $b$ there's a neighborhood $V$ and a map $p^{-1}(V)\to p^{-1}(b)$ such that $p^{-1}(V)\xrightarrow{\cong}V\times p^{-1}(b)$ is a homeomorphism.
\end{enumerate}
\end{definition}
That's exactly the way we topologized $o_M$. There's more structure though because $H_n(M,M-\overline{V})$ is an $R$-module!
\begin{definition}
A local system (of $R$-modules) $p:E\to B$ is a covering space together with structure maps $E\times_B E:=\{(e,e^\prime)|pe=pe^\prime\}\xrightarrow{+} E$ and $z:B\to E$ such that:
\begin{equation*}
\xymatrix{
	E\times_B E\ar[rr]^+\ar[dr] & & E\ar[dl] & R\times E\ar[l]\\
	 & B\ar@{=}[r]\ar[ur] & B\ar[u]^{z} & 
}
\end{equation*}
making $p^{-1}(b)$ a $R$-module.
\end{definition}
We have $H_n(M)\xrightarrow{j_x}H_n(M,M-x)$, which gives a \emph{section} of $o_M$. If I have a covering space $p:E\to B$, a section is a continuous map $s:B\to E$ such that $ps=1_B$. Write $\Gamma(E)$ to be the set of sections. If $E$ is a local system, this is an $R$-module. Hence $H_n(M)\xrightarrow{j_x}H_n(M,M-x)$ gives a map $j:H_n(M)\to \Gamma(o_M)$. This is pretty cool because it's telling you about this high-dimensional homology of $M$ into something ``discrete''.
\begin{theorem}
If $M$ is compact then $j:H_n(M)\to\Gamma(o_M)$ is an isomorphism, and $H_q(M)=0$ for $q>n$.
\end{theorem}
This is case of Poincar\'{e} duality actually because $\Gamma(o_M)$ is somewhat like zero-dimensional cohomology. If this is trivial, like it is for a torus, so if the manifold is connected, then $\Gamma(o_M)$ is just $R$.
