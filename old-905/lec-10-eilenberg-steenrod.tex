\section{Excision, and the Eilenberg-Steenrod axioms}
We have homotopy invariance and the lexseq of a pair. We claimed that $ H_\ast(X,A)$ ``depends only on $X-A$''. You have to be careful about this. 
\begin{definition}
A triple $(X,A,U)$ where $U\subseteq A\subseteq X$ is \emph{excisive} if $\overline{U}\subseteq\mathrm{Int}(A)$. This is a point-set definition. From an excisive triple you can get a pair $(X-U,A-U)\subseteq (X,A)$, and this is called an excision.
\end{definition}
\begin{theorem}
An excision induces an isomorphism in homology, i.e., $ H_\ast(X-U,A-U)\cong H_\ast(X,A)$. We might prove this on Wednesday.
\end{theorem}
What are some consequences? We'll compute $ H_\ast(S^n)$ and $ H^\ast(D^n,S^{n-1})$. Here's the result. We'll use the homeomorphisms $D^n\simeq \Delta^n$ and $S^{n-1}\simeq\partial\Delta^n$. Let's write $S^0=\{0,1\}$, and $\iota_n:\Delta^n\to\Delta^n$.
\begin{theorem}
	\begin{enumerate}
	\item \begin{equation*}
	 H_q(S^n)=\begin{cases}\Z = \langle[c^0_\ast]\rangle & q=0,n>0\\ \Z\oplus\Z = \langle[c^0_1],[\partial\iota_1]\rangle & q=n=0 \\ \Z = \langle[\partial\iota_{n+1}]\rangle & q=n>0 \\ 0 & \text{else} \end{cases}
	\end{equation*}

	\item \begin{equation*}
	 H_q(D^n,S^{n-1}) = \begin{cases}
	\Z=\langle [\iota_n]\rangle & q=n\\
	0 & \text{else}
	\end{cases}
	\end{equation*}
	\end{enumerate}
\end{theorem}
If $n=0$, then we say that $S^{-1}=\emptyset$. What are the generators of these groups?
\begin{proof}
We'll use the lexseq, homotopy invariance, and excision. We have the lexseq:
\begin{equation*}
\xymatrix{ & & \ar[dll]^\partial\\
 H_q(S^{n-1})\ar[r] & H_q(D^n)\ar[r] & H_q(D^n,S^{n-1})\ar[dll]^\partial\\
 H_{1-1}(S^{n-1})\ar[r] & H_{q-1}(D^n)\ar[r] & H_{q-1}(D^n,S^{n-1})\ar[dll]^\partial\\
 & & &}
\end{equation*}
But we know that $D^n$ is contractible, so $ H_q(D^n)=\begin{cases}\Z & q=n\\ 0 & \text{else}\end{cases}$. This means that $\partial: H_q(D^n,S^{n-1})\cong H_{q-1}(S^{n-1})$ for $q>1$, but when $q=1$, we get $0\to H_1(D^n,S^{n-1})\xrightarrow{\partial} H_0(S^{n-1})\to H_0(D^n) \to H_0(D^n,S^{n-1})\to 0$.

Let's think about the case $n>1$. Then $ H_0(S^{n-1})=\mathbf{Z}=\langle[c^0_\ast]\rangle$, and $ H_0(D^n)=\Z=\langle[c^0_\ast]\rangle$, so you have an isomorphism, which means that $ H_0(D^n,S^{n-1})=0$ and $ H_1(D^n,S^{n-1})=0$. Now, let's go to the case $n=1$. Then $ H_0(S^0)=\Z\oplus\Z=\langle [c^0_\ast],[\partial\iota_1]\rangle$ and $ H_0(D^0)=\Z$. But $[\partial\iota_1]$ goes to zero, and $[c^0_\ast]$ goes to the generator of $\Z= H_0(D^0)$. This means that $ H_1(D^1,S^0)$ is generated by $\langle[\iota_1]\rangle\cong\Z$ because the map $ H_1(D^n,S^{n-1})\to H_0(S^{n-1})$ is the boundary map, so $[\iota_1]\mapsto [\partial\iota_1]$.

Excision will come into play through the following statement.
\begin{prop}
If $n>1,q>1$, then $ H_q(D^n,S^{n-1})\to H_q(D^n/S^{n-1},\ast)\cong H_q(S^n,\ast)\cong H_q(S^n)$, because $D^n/S^{n-1}\simeq S^n$. The claim is that this collapse map is an isomorphism.
\end{prop}
This provides the inductive step. Let's assume we've proved the proposition. Then $ H_q(D^n,S^{n-1})\cong H_{q-1}(S^{n-1})$. The proposition says that $ H_q(S^n)\cong H_{q-1}(S^{n-1})$. But we also have the boundary map $\partial: H_{q+1}(D^n,S^{n-1})\cong H_q(S^n)$, i.e., $ H_{q+1}(D^n,S^{n-1})\cong H_q(D^n,S^{n-1})$.

Now I want to prove the proposition. 
\begin{proof}[Proof of proposition]
We want to compare $ H_q(D^n,S^{n-1})$ and $ H_q(D^n/S^{n-1},\ast)$. We'll use excision to do this. We have $D^n=\{ x\in\mathbf{R}^{n+1} | |x|\leq 1 \}$. Let $A=\{ x|1/3\leq |x|\leq 1 \}$ and $U=\{ x|2/3<|x|\leq 1 \}$, and $\{ x||x|=1 \} =S^{n-1}\subseteq U$. We need a preliminary step. $ H_q(D^n,S^{n-1})\to H_q(D^n,A)$. We claim that this an isomorphism, but this is true because of the lexseq and the 5-lemma. By excision, $ H_q(D^n,A)\cong H_q(D^n-U,A-U)$. We can collapse the $(n-1)$-sphere, and $(D^n/S^{n-1}-U/S^{n-1},A/S^{n-1}-U/S^{n-1})=(D^n-U,A-U)$ because you're collapsing something from something that's already been collapsed! Now, we claim that $ H_q(D^n/S^{n-1}-U/S^{n-1},A/S^{n-1}-U/S^{n-1})\cong H_q(D^n/S^{n-1},A/S^{n-1})$, which is true by excision. THE FOLLOWING PART IS MOST DEFINITELY NOT RIGHT\footnote{My own comment: this proof can be finished by noticing that $A\simeq S^{n-1}$ via $\mathbf{v}\mapsto\frac{\mathbf{v}}{||\mathbf{v}||}$, and that $D^n/S^{n-1}\simeq S^n$.}. But also, $ H_q(D^n/S^{n-1},\text{disk})= H_q(D^n/S^{n-1},A/S^{n-1})$. Because the disk is contractible, using the lexseq and the 5-lemma completes the proof of the proposition.
\end{proof}
\end{proof}
``This really turns me on, because I love homology.'' Why should you care about homology?
\begin{corollary}
If $m\neq n$, then $S^m\not\simeq S^n$ because they have different homology groups. 
\end{corollary}
\begin{corollary}
If $m\neq n$, then $\mathbf{R}^m\not\cong \mathbf{R}^n$, because they're not homeomorphic.
\end{corollary}
\begin{proof}
Let $m,n>0$. Assume we have a homeomorphism $f:\mathbf{R}^m\to \mathbf{R}^n$. This restricts to $\mathbf{R}^m-\{0\}\to \mathbf{R}^n-\{0\}$, but each of these are homotopy equivalent to spheres, but we can't get a homotopy equivalence between two spheres of different dimension by the above corollary.
\end{proof}
\subsection{Eilenberg-Steenrod axioms}
\begin{definition}
A homology theory (on $\mathbf{Top}$) is:
\begin{itemize}
\item a functor $h_n:\mathbf{Top}_2\to\mathbf{Ab}$ for all $n$. We'll write $h_n(X)=h_n(X,\emptyset)$
\item natural transformations $\partial:h_n(X,A)\to h_{n-1}(A)$.
\end{itemize}
such that:
\begin{itemize}
\item if $f_0,f_1:(X,A)\to (Y,B)$ are homotopic, then $f{0,\ast}\simeq f_{1,\ast}:h_n(X,A)\to h_n(Y,B)$.
\item an excision induces isomorphisms.
\item a lexseq:
\begin{equation*}
\cdots\to h_{q+1}(X,A)\xrightarrow{\partial}h_q(A)\to h_q(X)\to h_q(X,A)\xrightarrow{\partial}\cdots
\end{equation*}
\item (the dimension axiom): $h_n(\ast)$ is nonzero only in dimension zero. This is like the parallel postulate.
\end{itemize}
\end{definition}
\begin{example}
Ordinary singular homology satisfies these. 
\end{example}
\begin{theorem}[Brouwer fixed-point theorem]
If $f:D^n\to D^n$, then there is some point $x\in D^n$ such that $f(x)=x$.
\end{theorem}
\begin{proof}
Suppose not. Then you can draw a ray from $x$ to $f(x)$ to the boundary $S^{n-1}$, intersecting at a point $g(x)$. Left to you to check that $g$ is continuous. If $x$ was on the boundary, then $x=g(x)$. This is inconsistent by our computation because otherwise the identity on $ H_{n-1}(S^{n-1})$ would be zero, contradiction!
\end{proof}
