\section{Serre classes}\label{section-serre-classes}
\begin{definition}\label{serre-class}
    A class $\cC$ of abelian groups is a \emph{Serre class} if:
    \begin{enumerate}
	\setcounter{enumi}{0}
	\item $0\in \cC$.
	\item if I have a short exact sequence $0\to A\to B\to C\to 0$, then
	    $A\& C\in \cC$ if and only if $B\in\cC$.
    \end{enumerate}
\end{definition}
Some consequences of this definition: a Serre class is closed under
isomorphisms (easy). A Serre class is closed under subobjects and quotients,
because there is a short exact equence
$$0\to A\hookrightarrow B \to B/A\to 0.$$
Consider an exact sequence $A\to B\to C$ (not necessarily a \emph{short} exact
sequence). If $A,C\in \cC$, then $B\in \cC$ because we have a short exact
sequence:
$$
\xymatrix{
    & & & \coker i\ar[d]\ar[r] & 0\\
    & A\ar[r]^i\ar[d] & B\ar[r]^p\ar[ur] & C & 0\\
    0\ar[r] & \ker p\ar[ur] & & & 
}
$$
Some examples are in order.
\begin{example}
    \begin{enumerate}
	\item $\cC = \{0\}$, and $\cC$ the class of all abelian groups.
	\item Let $\cC$ be the class of all torsion abelian groups. We need to
	    check that $\cC$ satisfies the second condition of Definition
	    \ref{}. Consider a short exact sequence
	    $$0\to A\xrightarrow{i} B\xrightarrow{p} C\to 0.$$
	    We need to show that $B$ is torsion if $A$ and $C$ are torsion. To
	    see this, let $b\in B$. Then $p(b)$ is killed by some integer $n$,
	    so there exists $a\in A$ such that $i(a) = nb$. SInce $A$ is
	    torsion, it follows that $b$ is torsion, too.
	\item Let $\cP$ be a set of primes. Define:
	    $$
	    \cC_{\cP} = \{ A : \text{if }p\not\in\cP,\text{ then }p:A\xrightarrow{\simeq} A\text{, i.e., }A \text{ is a } \Z[1/p]\text{-module}\}
	    $$
	    Let $\Z_{(\cP)} = \Z[1/p: p\not\in\cP]\subseteq\QQ$.
	    
	    For instance, if $\cP$ is the set of all primes, then $\cC_\cP$ is
	    the Serre class of all abelian groups. If $\cP$ is the set of all
	    primes other than $\ell$, then $\cC_\cP$ is the Serre class
	    consisting of all $\Z[1/\ell]$-modules. If $\cP = \{\ell\}$, then
	    $\cC_{\{\ell\}} =: \cC_\ell$ is the Serre class of all
	    $\Z_{(\ell)}$-modules. If $\cP = \emptyset$, then $\cC_\emptyset$
	    is all rational vector spaces.
	\item If $\cC$ and $\cC^\prime$ are Serre classes, then so is $\cC\cap
	    \cC^\prime$. For instance, $\cC_\text{tors} \cap \cC_\text{fg}$ is
	    the Serre class $\cC_\text{finite}$. Likewise, $\cC_p \cap
	    \cC_\text{tors}$ is the Serre class of all $p$-torsion abelian
	    groups.
    \end{enumerate}
\end{example}
Here are some straightforward consequences of the definition:
\begin{enumerate}
    \item If $C_\bullet$ is a chain complex, and $C_n\in \cC$, then
	$H_n(C_\bullet)\in\cC$.
    \item Suppose $F_\ast A$ is a filtration on an abelian group. If
	$A\in\cC$, then $\gr_nA\in\cC$ for all $n$. If $F_\ast A$ is finite
	and $\gr_n A\in\cC$ for all $n$, then $A\in\cC$.
    \item Suppose we have a spectral sequence $\{E_r\}$. If $E^2_{s,t}\in \cC$,
	then $E^r_{s,t}\in \cC$ for $r\geq 2$. It follows that if $\{E^r\}$ is
	a right half-plane spectral sequence, then $E^{s+1}_{s,t}\fib
	E^{s+2}_{s,t}\fib\cdots\fib E^\infty_{s,t}\in\cC$.

        Thus, if the spectral sequence comes from a filtered complex (which is
	bounded below, such that for all $n$ there exists an $s$ such that $F_s
	H_n(C) = H_n(C)$, i.e., the homology of the filtration stabilizes),
	then $E^\infty_{s,t} = \gr_s H_{s+t}(C)$. This means that if the
	$E^2_{s,t}\in\cC$ for all $s+t = n$, then $H_n(C)\in\cC$.
\end{enumerate}
To apply this to the Serre spectral sequence, we need an additional axiom for
Definition \ref{serre-class}:
\begin{enumerate}
	\setcounter{enumi}{1}
    \item if $A,B\in\cC$, then so are $A\otimes B$ and $\Tor_1(A,B)$.
\end{enumerate}
All of the examples given above satisfy this additional axiom.
\begin{terminology}
    $f:A\to B$ is said to be a $\cC$-epimorphism if $\coker f\in \cC$, a
    $\cC$-monomorphism if $\ker f\in\cC$, and a $\cC$-isomorphism if it is a
    $\cC$-epimorphism and a $\cC$-monomorphism.
\end{terminology}
\begin{prop}
    Let $\pi:E\to B$ be a fibration and $B$ path connected, such that the fiber
    $F = \pi^{-1}(\ast)$ is path connected. Suppose $\pi_1(B)$ acts trivially
    on $H_\ast(F)$.

    Let $\cC$ be a Serre class satisfiying Axiom 2. Let $s\geq 3$, and assume
    that $H_n(E)\in\cC$ where $1\leq n<s-1$ and $H_t(B)\in \cC$ for $1\leq
    t<s$. Then $H_t(F)\in\cC$ for $1\leq t<s-1$.
\end{prop}
\begin{proof}
    We will do the case $s=3$, for starters.
    We're gonna want to relate the low-dimension homology of these groups.
    What can I say?
    We know that $H_0(E) = \Z$ since it's connected.
    I have $H_1(E)\to H_1(B)$, via $\pi$.
    This is one of the edge homomorphisms, and thus it surjects (no possibility for a differential coming in).
    I now have a map $H_1(F)\to H_1(E)$.
    But I have a possible $d^2:H_2(B)\to H_1(F)$, which is a transgression that gives:
    $$
    H_2(B)\xrightarrow{\partial} H_1(F)\to H_1(E)\to H_1(B)\to 0
    $$
    
    Let me take a step back and say something general.
    You might be interested in knowing when something in $H_n(F)$ maps to zero in $H_n(E)$.
    I.e., what's the kernel of $H_n(F)\to H_n(E)$.
    The sseq gives an obstruction to being an isomorphism.
    The only way that something can be killed by $H_n(F)\to H_n(E)$ is described by:
    $$
    \ker(H_n(F)\to H_n(E)) = \bigcup\left(\img \text{ of }d^r\text{ hitting }E^r_{0,n}\right)
    $$
    You can also say what the cokernel is:
    it's whatever's left in $E^\infty_{s,t}$ with $s+t = n$.
    These obstruct $H_n(F)\to H_n(E)$ from being surjective.
    
    In the same way, I can do this for the base.
    If I have a class in $H_n(E)$, that maps to $H_n(B)$, the question is: what's the image?
    Well, the only obstruction is the possibility is that the element in $H_n(B)$ supports a nonzero differential.
    Thus:
    $$
    \img(H_n(E)\xrightarrow{\pi_\ast} H_n(B)) = \bigcap\left(\ker(d^r:E^r_{r,0}\to\cdots)\right)
    $$
    Again, you can think of the sseq as giving obstructions.
    And also, the obstruction to that map being a monomorphism that might occur in lower filtration along the same total degree line.

    Back to our argument.
    We had the low-dimensional exact sequence:
    $$
    H_2(B)\xrightarrow{\partial} H_1(F)\to H_1(E)\to H_1(B)\to 0
    $$
    Here $p=3$, so we have $H_2(B)\in\cC$ and $H_1(E)\in\cC$.
    Thus $H_1(F)\in\cC$.
    That's the only thing to check when $p=3$.

    Let's do one more case of this induction.
    What does this say?
    Now I'll do $p=4$.
    We're interested in knowing if $E^2_{0,3}\in\cC$.
    There are now two possible differentials!
    I have $H_2(F) = E^2_{0,2}\fib E^3_{0,2}$.
    This quotient comes from $d^2:E^2_{2,1}\to E^2_{0,2}$.
    Now, $d^3:E^3_{3,0}\to E^3_{0,2}$ which gives a surjection $E^3_{0,2}\fib E^4_{0,2}\simeq E^\infty_{0,2}\hookrightarrow H_2(E)$.
    Now, our assumptions were that $E^2_{2,1},E^3_{3,0},H_2(E)\in\cC$.
    Thus $E^3_{0,2}\in\cC$ and so $E^2_{0,2} = H_2(F)\in\cC$.
    Ta-da!
\end{proof}
We're close to doing actual calculations, but I have to talk about the multiplicative structure on the Serre sseq first.
