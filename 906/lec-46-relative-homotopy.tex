\section{Relative homotopy groups}
%Pset 2 has question 8; there's still one more to go. Hood has office hours today 12 -- 1 in 2-390, and I have office hours today tomorrow 4-5 in 2-478.
\subsection{Spheres and homotopy groups}
The functor $\Omega$ (sending a space to its based loop space) admits a left adjoint.
To see this, recall that $\Omega X = X^{S^1}_\ast$, so that
$$\Top_\ast(W,\Omega X) = \Top_\ast(S^1\wedge W,X).$$
\begin{definition}
    The \emph{reduced suspension} $\Sigma W$ is $S^1\wedge W$.
\end{definition}
If $A\subseteq X$, then
$$X/A\wedge Y/B = (X\times Y)/((A\times Y)\cup_{A\times B}(X\times B)).$$
Since $S^1 = I/\partial I$, this tells us that $\Sigma X = S^1\wedge X$ can be identified with
$I\times X/(\partial I \times X\cup I\times \ast)$: in other words, we collapse the top and bottom of a cylinder to a point,
as well as the line along a basepoint.

The same argument says that $\Sigma^n X$ (defined inductively as $\Sigma(\Sigma^{n-1} X)$)
is the left adjoint of the $n$-fold loop space functor $X\mapsto \Omega^n X$.
In other words, $\Sigma^n X = (S^1)^{\wedge n}\wedge X$.
We claim that $S^1\wedge S^n \simeq S^{n+1}$.
To see this, note that
$$S^1\wedge S^n = I/\partial I\wedge I^n\wedge \partial I^n = (I\times I^n)/(\partial I\times I^n\cup I\times \partial I^n).$$
The denominator is exactly $\partial I^{n+1}$, so $S^1\wedge S^n\simeq S^{n+k}$.
It's now easy to see that $S^k\wedge S^n\simeq S^{k+n}$.
\begin{definition}
    The \emph{$n$th homotopy group} of $X$ is $\pi_n X = \pi_0(\Omega^n X)$.
\end{definition}
This is, as we noted in the previous section, $[S^0,\Omega^n X]_\ast = [S^n, X]_\ast = [(I^n,\partial I^n),(X,\ast)]$.

\subsection{The homotopy category}
Define the \emph{homotopy category of spaces} $\Ho(\Top)$ to be the category
whose objects are spaces, and whose hom-sets are given by taking $\pi_0$ of the mapping space.
To check that this is indeed a category, we need to check that if $f_0,f_1:X\to Y$ and $g:Y\to Z$, then $gf_0\simeq gf_1$ ---
but this is clear.
Similarly, we'd need to check that $f_0h\simeq f_1h$ for any $h:W\to X$.
We can also think about the homotopy category of pointed spaces (and pointed homotopies) $\Ho(\Top_\ast)$; this is the category
we have been spending most of our time in.
Both $\Ho(\Top)$ and $\Ho(\Top_\ast)$ have products and coproducts, but very few other limits or colimits.
From a category-theoretic standpoint, these are absolutely terrible.

Let $W$ be a pointed space.
We would like the assignment $X\mapsto X^W_\ast$ to be a homotopy functor.
It clearly defines a functor $\Top_\ast\to\Top_\ast$, so this desire is equivalent to providing a dotted arrow in the
following diagram:
\begin{equation*}
    \xymatrix{
	\Top_\ast\ar[d]\ar[r]^{X\mapsto X^W_\ast} & \Top_\ast\ar[d]\\
	\Ho(\Top_\ast)\ar@{-->}[r] & \Ho(\Top_\ast).
    }
\end{equation*}
Before we can prove this, we will check that a homotopy $f_0\sim f_1:X\to Y$ is the same as a map $I_+\wedge X\to Y$.
There is a nullhomotopy if the basepoint of $I$ is one of the endpoints, so a homotopy is the same as a map
$I\times X/I\times\ast \to Y$. The source is just $I_+\wedge X$, as desired.

A homotopy $f_0 \simeq f_1:X\to Y$ begets a map $(I_+\wedge X)^W\to Y^W_\ast$.
For the assignment $X\mapsto X^W_\ast$ to be a homotopy functor, we need a natural transformation $I_+\wedge X^W_\ast\to Y^W_\ast$, so this map is not quite what's necessary.
Instead, we can attempt to construct a map $I_+\wedge X^W_\ast\to (I_+\wedge X)^W_\ast$.

We can construct a general map $A\wedge X^W_\ast\to (A\wedge X)^W_\ast$: 
there is a map $A\wedge X^W_\ast\to A^W_\ast\wedge X^W_\ast$, given by sending $a\mapsto c_a$;
then the exponential law gives a homotopy $A^W_\ast\wedge X^W_\ast\to (A\wedge X)^W_\ast$.
This, in turn, gives a map $I_+\wedge X^W_\ast\to (I_+\wedge X)^W_\ast\to Y^W_\ast$,
thus making $X\mapsto X^W_\ast$ a homotopy functor.

Motivated by our discussion of homotopy fibers, we can study composites which ``behave'' like short exact sequences.
\begin{definition}
    A \emph{fiber sequence} in $\Ho(\Top_\ast)$ is a composite $X\to Y\to Z$ that is
    isomorphic, in $\Ho(\Top_\ast)$, to some composite $Ff\xrightarrow{p} E\xrightarrow{f}B$;
    in other words, there exist (possibly zig-zags of) maps that are homotopy equivalences, that make the following
    diagram commute:
    \begin{equation*}
	\xymatrix{
	    X\ar[r]\ar[d] & Y\ar[r]\ar[d] & Z\ar[d]\\
	    Ff\ar[r]_p & E\ar[r]_f & B.
	    }
    \end{equation*}
\end{definition}
Let us remark here that if $A^\prime \xar{\sim} A$ is a homotopy equivalence, and $A\to B \to C$ is a fiber sequence, so
is the composite $A^\prime\xar{\sim} A \to B\to C$.

\begin{exercise}\label{loopslimit}
    Prove the following statements.
    \begin{itemize}
	\item $\Omega$ takes fiber sequences to fiber sequences.
	\item $\Omega Ff\simeq F\Omega f$. Check this!
    \end{itemize}
\end{exercise}

We've seen examples of fiber sequences in our elaborate study of the Barratt-Puppe sequence.
\begin{example}
Recall our diagram:
\begin{equation*}
    \xymatrix{
	\cdots\ar[r] & Fp_4\ar[r] & Fp_3 \ar[r] & Fp_2\ar[r] & Fp_1\ar[r]^{p_2} & Ff\ar[r]^{p_1} & X\ar[r]^{f} & Y\\
    \cdots\ar[r] & \Omega Fp_1\ar[r]|{\overline{\Omega p_2}}\ar[u]_{\simeq} & \Omega Ff\ar[u]_{\simeq}\ar[ur]|{i(p_2)}\ar[r]|{\overline{\Omega p}} & \Omega X\ar[r]|{\overline{\Omega f}}\ar[u]_{\simeq}\ar[ur]|{i(p_1)} & \Omega Y\ar[u]_\simeq \ar[ur]|{i(f)} & &\\
	\Omega^2 X\ar[u]_{\simeq}\ar[r]_{\Omega f} & \Omega Y\ar[u]_{\simeq}\ar[ur]_{\overline{\Omega i(f)}} & & &
    }
\end{equation*}
The composite $Ff\to X\xar{f} Y$ is canonically a fiber sequence.
The above diagram shows that $\Omega Y\to F\xrightarrow{p}X$ is another fiber sequence: it is isomorphic to
$Fp\to F\to X$ in $\Ho(\Top_\ast)$.
Similarly, the composite $\Omega X\xrightarrow{\overline{\Omega f}}\Omega Y\to F$ is another fiber sequence;
this implies that $\Omega X\xrightarrow{\Omega f}\Omega Y\to F$ is also an example of a fiber sequence
(because these two fiber sequences differ by an automorphism of $\Omega X$)

Applying $\Omega$ again, we get $\Omega F\xrightarrow{\Omega p} \Omega X\xrightarrow{\Omega f} \Omega Y$.
Since this is a looping of a fiber sequence, and taking loops takes fiber sequences to fiber sequences (Exercise \ref{loopslimit}), this is another fiber sequence. 
Looping again gives another fiber sequence $\Omega^2 Y\xrightarrow{\Omega i} \Omega F\xrightarrow{\Omega p}\Omega X$.
(For the category-theoretically--minded folks, this is an unstable version of a triangulated category.)
\end{example}

\subsection{The long exact sequence of a fiber sequence}
As discussed at the end of \S \ref{secbarrattpuppe}, applying $\pi_0 = [S^0,-]_\ast$ to the
Barratt-Puppe sequence associated to a map $f:X\to Y$ gives a long exact sequence:
\begin{equation*}
    \xymatrix{
	& \cdots\ar[r] & \pi_2 Y\ar[dll]\\
	\pi_1 F\ar[r] & \pi_1 X\ar[r] & \pi_1 Y\ar[dll]\\
	\pi_0 F\ar[r] & \pi_0 X. & 
    }
\end{equation*}
of pointed sets.
The space $\Omega^2 X$ is an \emph{abelian} group object in $\Ho(\Top)$
(in other words, the multiplication on $\Omega^2 X$ is commutative up to homotopy).
This implies $\pi_1(X)$ is a group, and that $\pi_k(X)$ is abelian for $k\geq 2$;
hence, in our diagram above, all maps (except on $\pi_0$) are group homomorphisms.

Consider the case when $X\to Y$ is the inclusion $i:A\hookrightarrow X$ of a subspace.
In this case,
$$Fi=\{(a,\omega)\in A\times X^I_\ast|\omega(1) = a\};$$
this is just the collection of all paths that begin at $\ast\in A$ and end in $A$.
This motivates the definition of \emph{relative homotopy groups}:
\begin{definition}
    Define: 
    $$\pi_n(X,A,\ast) = \pi_n(X,A) := \pi_{n-1}Fi = [(I^n,\partial I^n,(\partial I^n\times I)\cup (I^{n-1}\times 0)),(X,A,\ast)].$$
\end{definition}
We have a sequence of inclusions
$$\partial I^n\times I\cup I^{n-1}\times 0 \subset \partial I^n \subset I^n.$$
One can check that
$$\pi_{n-1}Fi = [(I^n,\partial I^n,(\partial I^n\times I)\cup (I^{n-1}\times 0)),(X,A,\ast)].$$
This gives a long exact sequence on homotopy, analogous to the long exact sequence in relative homology:
\begin{equation}\label{lexseqhomotopy}
    \xymatrix{
	& \cdots\ar[r] & \pi_2 (X,A)\ar[dll]\\
	\pi_1 A\ar[r] & \pi_1 X\ar[r] & \pi_1 (X,A)\ar[dll]\\
	\pi_0 A\ar[r] & \pi_0 X & 
    }
\end{equation}
