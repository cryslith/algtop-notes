\section{The Whitney sum formula}
%Let me just repeat what I was rushing through at the end on Wednesday.
%I wrote slightly incorrect things.
%Let's look at $\cc^n$.
%Of course, $U(n)$ acts on this, and preserves the inner product.
%Recall that $Fr(\cc^n)$ is the space of orthonormal bases of $\cc^n$.
%Thus there's an action of $U(n)$ on this, and this is simply transitive, i.e., it's free and transitive.
%Some would say that this is a ``torsor for $U(n)$''.
%There's a basepoint in here, namely the standard basis, but you may forget about that.
%
%Now I can make various constructions.
%For instance, think about $\Fl(\cc^n)$, which is the space of ordered orthonormal sets of $n$ lines in $\cc^n$.
%There's lots of isotropy now.
%The isotropy group of the $U(n)$ action on the lines through the standard basis is now $T_n=U(1)^n$, i.e., diagonal matrices with roots of unity on the diagonal.
%This is called the \emph{(complete) flag manifold}.
%
%Another thing you could do as well -- you can forget the ordering.
%This doesn't really have a name, I guess, but what I'm doing is considering $\Fl(\cc^n)/\Sigma_n$, which is the space of unordered orthonormal lines in $\cc^n$.
%This is a homogeneous space, too, and it's $U(n)$ quotiented by permutation matrices where $1$ is any complex number of norm $1$.
%For instance, if $N_{U(n)}(T^n) = \Sigma_n\cdot T^n$ is the normalizer, then $\Fl(\cc^n)/\Sigma_n = U(n)/N(T^n)$.
%I can consider the cofiber sequence $T^n\hookrightarrow \Sigma_n\cdot T^n \to W_n$, and the $W_n$ is called the Weyl group.
%Something nice here is that
%\begin{equation*}
%    \xymatrix{
%	T^n\ar@{^(->}[r] & \Sigma_n\cdot T^n\ar[r] & W_n\\
%	& \Sigma_n\ar@{^(->}[u]\ar[ur]^\simeq & 
%    }
%\end{equation*}
%To get projective space I consider the space of splittings of $\cc^n$ into a line and its orthogonal complement -- but that's just giving a line because the metric determines the orthogonal complement.
%
%Consider a complex $n$-plane bundle $\xi^n\downarrow X$, with a metric.
%Then $\Fr(\xi^n)\to X$ is a principal $U(n)$-bundle over $X$.
%I can form $E(\xi) = \Fr(\xi)\times_{U(n)}\cc^n$, and $\PP(\xi) = \Fr(\xi)\times_{U(n)} U(n)/(U(1)\times U(n-1))$.
%Note that $U(n)/(U(1)\times U(n-1)) = \CP^{n-1}$.
%I can also consider $\Fl(\xi) = \Fr(\xi)\times_{U(n} U(n)/T^n$.
%
%We found that $H^\ast(X) \hookrightarrow H^\ast(\Fl(\xi))$.
%This is the splitting principle.
%
%Let's look at the universal case $\xi_n\downarrow BU(n)$.
%So, $\Fl(\xi_n) = EU(n)\times_{U(n)} U(n)/T^n$, so this is $BT^n$.
%It maps down to $BU(n)$, and the map is the one induced by the inclusion $T^n\hookrightarrow U(n)$.
%So, for any ring, we find that $H^\ast(BU(n))\hookrightarrow H^\ast(BT^n)$.
%We know that 
%$$
%H^\ast(BT^n) = \Z[t_1,\cdots,t_n]
%$$
%where $t_i = e(\lambda_i)$ and $|t_i| = 2$, where $\lambda_i = pr_i^\ast\lambda$ and $\lambda\downarrow \CP^\infty$ is the universal line bundle.
As we saw in the previous section, there is an injection
$H^\ast(BU(n))\hookrightarrow H^\ast(BT^n)$. What is the image of this map?

The symmetric group sits inside of $U(n)$, so it acts by conjugation on $U(n)$.
This action stabilizes this subgroup $T^n$. By naturality, $\Sigma_n$ acts on
the classifying space $BT^n$. Since $\Sigma_n$ acts by conjugation on $U(n)$,
it acts on $BU(n)$ in a way that is homotopic to the identity (Lemma
\ref{conjugation-homotopic}). However, each element $\sigma\in \Sigma_n$ simply
permutes the factors in $BT^n = (\CP^\infty)^n$; we conclude that
$H^\ast(BU(n);R)$ actually sits inside the invariants
$H^\ast(BT^n;R)^{\Sigma_n}$.

Recall the following theorem from algebra:
\begin{theorem}
    Let $\Sigma_n$ act on the polynomial algebra $R[t_1,\cdots,t_n]$ by
    permuting the generators. Then
    $$R[t_1,\cdots,t_n]^{\Sigma_n} = R[\sigma_1^{(n)},\cdots,\sigma_n^{(n)}],$$
    where the $\sigma_i$ are the {elementary symmetric polynomials}, defined
    via
    $$\prod^n_{i=1}(x-t_i) = \sum^n_{j=0} \sigma_i^{(n)}x^{n-i}.$$
\end{theorem}
For instance,
$$\sigma_1^{(n)} = -\sum t_i,\ \sigma_n^{(n)} = (-1)^n\prod t_i.$$
If we impose a grading on $R[t_1,\cdots,t_n]$ such that $|t_i| = 2$, then
$|\sigma_i^{(n)}| = 2i$. It follows from our discussion in \S
\ref{homology-bun} that the ring $H^\ast(BT^n)^{\Sigma_n}$ has the same size as
$H^\ast(BU(n))$.
%they the same Poincar\'{e} series. You could imagine that maybe it embeds as a
%lattice inside the integers?

Consider an injection of finitely generated abelian groups $M\hookrightarrow
N$, with quotient $Q$. Suppose that, after tensoring with any field, the map
$M\to N$ an isomorphism. If $Q\otimes k = 0$, then $Q = 0$.
Indeed, if $Q\otimes \QQ = 0$ then $Q$ is torsion. Similarly, if $Q\otimes
\FF_p = 0$, then $Q$ has no $p$-component. In particular, $M\simeq N$. Applying
this to the map $H^\ast(BU(n)\to H^\ast(BT^n)^{\Sigma_n}$, we find that
$$
H^\ast(BU(n);R) \xrightarrow{\simeq} H^\ast(BT^n;R)^{\Sigma_n} =
R[\sigma_1^{(n)},\cdots,\sigma_n^{(n)}].
$$
What happens as $n$ varies? There is a map $R[t_1,\cdots,t_n] \to
R[t_1,\cdots,t_{n-1}]$ given by sending $t_n\mapsto 0$ and $t_i\mapsto t_i$ for
$i\neq n$. Of course, we cannot say that this map is equivariant with respect
to the action of $\Sigma_n$. However, it \emph{is} equivariant with respect to
the action of $\Sigma_{n-1}$ on $R[t_1,\cdots,t_n]$ via the inclusion of
$\Sigma_{n-1}\hookrightarrow \Sigma_n$ as the stabilizer of
$n\in\{1,\cdots,n\}$. Therefore, the $\Sigma_n$-invariants sit inside the
$\Sigma_{n-1}$-invariants, giving a map
$$
R[t_1,\cdots,t_n]^{\Sigma_n} \to R[t_1,\cdots,t_n]^{\Sigma_{n-1}} \to
R[t_1,\cdots,t_{n-1}]^{\Sigma_{n-1}}.
$$
We also find that for $i<n$, we have $\sigma_i^{(n)} \mapsto \sigma_i^{(n-1)}$
and $\sigma_n^{(n)} \mapsto 0$.
\subsection{Where do the Chern classes go?}\label{euler-multiplicativity}
To answer this question, we will need to understand the multiplicativity of the
Chern class. We begin with a discussion about the Euler class. Suppose $\xi^p\downarrow X,\eta^q\downarrow Y$ are oriented real
vector bundles; then, we can consider the bundle $\xi\times\eta\downarrow X\times Y$, which is
another oriented real vector bundle. The orientation is given by picking
oriented bases for $\xi$ and $\eta$.
%There are lot of choices, but I'll put the first one first and the second one
%second.
We claim that
$$e(\xi\times\eta) = e(\xi)\times e(\eta) \in H^{p+q}(X\times Y).$$
Since $D(\xi\times\eta)$ is homeomorphic to $D(\xi)\times D(\eta)$, and
$S(\xi\times\eta) = D(\xi)\times S(\eta)\cup S(\xi)\times D(\eta)$, we learn
from the relative K\"unneth formula that
$$H^\ast(D(\xi\times\eta),S(\xi\times\eta)) \leftarrow
H^\ast(D(\xi),S(\xi))\otimes H^\ast(D(\eta),S(\eta)).$$
It follows that
$$
u_{\xi\times \eta} = u_\xi\times u_\eta\in H^{p+q}(\Th(\xi)\times\Th(\eta));
$$
this proves the desired result since the Euler class is the image of the Thom
class under the map $H^n(\Th(\xi))\to H^n(D(\xi)) \simeq H^n(B)$.

Consider the diagonal map $\Delta:X\to X\times X$. The cross product in
cohomology then pulls back to the cup product, and the direct product of fiber
bundles pulls back to the Whitney sum. It follows that
$$e(\xi\oplus\eta) = e(\xi)\cup e(\eta).$$

If $\xi^n\downarrow X$ is an $n$-dimensional complex vector bundle, then we
defined\footnote{There's a slight technical snag here: a complex bundle doesn't
have an orientation. However, its underlying oriented real vector bundle does.}
$$
c_n(\xi) = (-1)^n e(\xi_\RR).
$$
We need to describe the image of $c_n(\xi_n)$ under the map $H^{2n}(BU(n)) \to
H^{2n}(BT^n)^{\Sigma_n}$.

Let $f:BT^n\to BU(n)$ denote the map induced by the inclusion of the maximal
torus. Then, by construction, we have a splitting
$$f^\ast\xi_n = \lambda_1\oplus\cdots\oplus \lambda_n.$$
Thus,
$$(-1)^ne(\xi)\mapsto (-1)^n e(\lambda_1\oplus\cdots\oplus \lambda_n) = (-1)^n
e(\lambda_1)\cup\cdots\cup e(\lambda_n).$$
The discussion above implies that $f^\ast$ sends the right hand side to
$(-1)^nt_1\cdots t_n = \sigma_n^{(n)}$. In other words, the top Chern class
maps to $\sigma_n^{(n)}$ under the map $f^\ast$.

Our discussion in the previous sections gives a commuting diagram:
\begin{equation*}
    \xymatrix{
	H^\ast(BU(n)) \ar[r] \ar[d] & H^\ast(BT^n)^{\Sigma_n}\ar[d]\\
	H^\ast(BU(n-1)) \ar[r] & H^\ast(BT^{n-1})^{\Sigma_{n-1}}
    }
\end{equation*}
Arguing inductively, we find that going from the top left corner to the bottom
left corner to the bottom right corner sends
$$c_i\mapsto c_i\mapsto \sigma_i^{(n-1)}\text{ for }i<n.$$
Likewise, going from the top left corner to the top right corner to the bottom
right corner sends
$$c_i\mapsto \sigma_i^{(n)}\mapsto \sigma_i^{(n-1)}\text{ for }i<n.$$
We conclude that the map $f^\ast$ sends $c_i^{(i)}\mapsto \sigma_i^{(i)}$.
\subsection{Proving the Whitney sum formula}
By our discussion above, the Whitney sum formula of Theorem \ref{chern-classes}
reduces to proving the following identity:
\begin{equation}\label{symmpol}
    \sigma^{(p+q)}_k = \sum_{i+j=k}\sigma_i^{(p)}\cdot\sigma_j^{(q)}
\end{equation}
inside $\Z[t_1,\cdots,t_p,t_{p+1},\cdots,t_{p+q}]$. Here, $\sigma_i^{(p)}$ is
thought of as a polynomial in $t_1,\cdots,t_p$, while $\sigma_i^{(q)}$ is
thought of as a polynomial in $t_{p+1},\cdots,t_{p+q}$. To derive Equation
\eqref{symmpol}, simply compare coefficients in the following:
\begin{align*}
    \sum_{k=0}^{p+q} \sigma_k^{(p+q)}x^{p+q-k} & = \prod^{p+q}_{i=1}(x-t_i)\\
    & = \prod^p_{i=1}(x-t_i)\cdot\prod^{p+q}_{j=p+1}(x-t_j)\\
    & = \left(\sum^p_{i=0}\sigma^{(p)}_i
    x^{p-i}\right)\left(\sum^q_{j=0}\sigma_j^{(p)} x^{q-j}\right)\\
    & =
    \sum^{p+q}_{k=0}\left(\sum_{i+j=k}\sigma_i^{(p)}\sigma_j^{(q)}\right)x^{p+q-k}.
\end{align*}
