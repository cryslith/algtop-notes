\section{Classifying spaces and bundles}
Let $\pi:Y\to X$ be a map of spaces. This defines a ``descent category''
$\cCC(\pi)$ whose objects are the points of $Y$, whose morphisms are points of
$Y\times_X Y$, and whose structure morphisms are the obvious maps. Let $cX$
denote the category whose objects and morphisms are both given by points of
$X$, so that the nerve $NcX$ is the constant simplicial object with value $X$.
There is a functor $\cCC(\pi)\to cX$ specified by the map $\pi$. 

Let $\cU$ be a cover of $X$. Let $\cCC(\cU)$ denote the descent category
associated to the obvious map $\epsilon:\coprod_{U\in \cU} U\to X$. It is easy
to see that $\epsilon:B\cCC(\cU) \simeq X$ if $\cU$ is numerable. The morphism
determined by $x\in U\cap V$ is denoted $x_{U,V}$. Suppose $p:P\to X$ is a
principal $G$-bundle.  Then $\cU$ trivializes $p$ if there are homeomorphisms
$t_U:p^{-1}(U)\xrightarrow{\simeq} U\times G$ over $U$. Specifying such
homeomorphisms is the same as a trivialization of the pullback bundle
$\epsilon^\ast P$.

This, in turn, is the same as a functor $\theta_P:\cCC(\cU)\to G$. To
see this, we note that the $G$-equivariant composite $t_V\circ t_U^{-1}:(U\cap
V)\times G\to (U\cap V)\times G$ is determined by the value of $(x,1)\in (U\cap
V)\times G$. The map $U\cap V\to G$ is denoted $f_{U,V}$. Then, the functor
$\theta_P:\cCC(\cU)\to G$ sends every object of $\cCC(\cU)$ to the point, and
$x_{U,V}$ to $f_{U,V}(x)$.

On classifying spaces, we therefore get a map $X\xleftarrow{\simeq}
B\cCC(\cU)\xrightarrow{\theta_P} BG$, where the map on the left is given by
$\epsilon$.
\begin{exercise}
    Prove that $\theta_P^\ast EG \simeq \epsilon^\ast P$.
\end{exercise}
This suggests that $BG$ is a classifying space for principal $G$-bundles (in
the sense of \S \ref{grassmannmodel}). To make this precise, we need to prove
that two principal $G$-bundles are isomorphic if and only if the associated
maps $X\to BG$ are homotopic.

To prove this, we will need to vary the open cover. Say that $\cV$
\emph{refines} $\cU$ if for any $V\in \cU$, there exists $U\in\cU$ such that
$V\subseteq U$. A \emph{refinement} is a function $p:\cV\to\cU$ such that
$V\subseteq p(V)$. A refinement $p$ defines a map $\coprod_{V\in \cV} V\to
\coprod_{U\in\cU} U$, denoted $\rho$.

As both $\coprod_{V\in \cV} V$ and $\coprod_{U\in\cU} U$ cover $X$, we get a
map $\cCC(\cV)\to \cCC(\cU)$ over $cX$. Taking classifying spaces begets a
diagram:
$$
\xymatrix{
    B\cCC(\cV)\ar[r]\ar[dr] & B\cCC(\cU)\ar[d]\\
    & X
}
$$
Let $t$ be trivialization of $P$ for the open cover $\cU$. The construction
described above begets a functor $B\cCC(\cU)\to BG$, so we get a trivialization
$s$ for $\cV$. This is a homeomorphism $s_V:p^{-1}(V)\to V\times G$ which fits
into the following diagram:
\begin{equation*}
    \xymatrix{
	p^{-1}(V)\ar[r]^{s_V}_\sim\ar[d] & V\times G\ar[d]\\
	p^{-1}(\rho(V))\ar[r]_{t_{\rho(V)}}^\sim & \rho(V)\times G
    }
\end{equation*}
By construction, there is a large commutative diagram:
\begin{equation*}
\xymatrix{
    B\cCC(\cV)\ar[r]\ar[dr]^\sim\ar@/^0.135in/[rr] &
    B\cCC(\cU)\ar[d]^\sim\ar[r] & BG\\
    & X.
}
\end{equation*}
This justifies dropping the symbol $\cU$ in the notation for the map
$\theta_P$.

Consider two principal $G$-bundles over $X$:
\begin{equation*}
    \xymatrix{
	P\ar[r]^{\simeq} \ar[dr] & Q\ar[d]\\
	& X,
    }
\end{equation*}
and suppose I have trivializations $(\cU, t)$ of $P$ and $(\cW, s)$ of $Q$.
Let $\cV$ be a common refinement, so that there is a diagram:
\begin{equation*}
    \xymatrix{
	& \cCC(\cU)\ar[dr]^{\theta^\cU_{P}} & \\
	\cCC(\cV)\ar[ur]\ar[dr] \ar@/^/[rr]^{\theta^\cV_P}
	\ar@/_/[rr]_{\theta^\cV_Q} & & G\\
	& \cCC(\cW)\ar[ur]_{\theta^\cW_Q}
    }
\end{equation*}
Included in the diagram is a mysterious natural transformation
$\beta:\theta^\cV_P\to \theta^\cV_Q$, whose construction is left as an exercise
to the reader\todo{Should we describe this? It's rather technical...}. Its
existence combined with Lemma \ref{nat-trans-htpy} implies that the two maps
$\theta_P,\theta_Q:B\cCC(\cV)\simeq X\to BG$ are homotopic, as desired.

\subsection{Topological properties of $BG$}
Before proceeding, let us summarize the constructions discussed so far. Let $G$
be some topological group (assumed to be an absolute neighborhood retract of a
Lie group). We constructed $EG$, which is a contractible space with $G$ acting
freely on the right (this works for any topological group). There is an orbit
projection $EG\to BG$, which is a principal $G$-bundle under our assumption on
$G$. The space $BG$ is universal, in the sense that there is a bijection
$$\Bun_G(X)\xleftarrow{\simeq} [X, BG]$$
given by $f\mapsto [f^\ast EG]$.

Let $E$ be a space such that $G$ acts on $E$ from the left. If $P\to B$ is any
principal $G$-bundle, then $P\times E\to P\times_G E$ is another principal
$G$-bundle. In the case $P = EG$, it follows that if $E$ is a contractible
space on which $G$ acts, then the quotient $EG\times_G E$ is a model for $BG$.
Recall that $EG$ is contractible. Therefore, if $E$ is a contractible space on
which $G$ acts freely, then the quotient $G\backslash E$ is a model for $BG$.
Of course, one can run the same argument in the case that $G$ acts on $E$ from
the right. Although the construction with simplicial sets provided us with a
very concrete description of the classifying space of a group $G$, we could
have chosen any principal action on a contractible space in order to obtain a
model for $BG$.

Suppose $X$ is a pointed path connected space. Remember that $X$ has a
contractible path space $PX = X^I_\ast$. The canonical map $PX\to X$ is a
fibration, with fiber $\Omega X$.

Consider the case when $X = BG$. Then, we can compare the above fibration with
the fiber bundle $EG\to BG$:
\begin{equation*}
    \xymatrix{
	G\ar[r]\ar[d] & \Omega BG\ar[d]\\
	\ast \simeq EG\ar[d]\ar@{-->}[r] & PBG\simeq \ast \ar[d]\\
	BG\ar@{=}[r] & BG
    }
\end{equation*}
The map $EG\to BG$ is nullhomotopic; a choice of a nullhomotopy is exactly a
lift into the path space. Therefore, the dotted map $EG\to PBG$ exists in the
above diagram. As $EG$ and $PBG$ are both contractible, we conclude that
$\Omega BG$ is weakly equivalent to $G$. In fact, this weak equivalence is a
$H$-map, i.e., it commutes up to homotopy with the multiplication on both
sides.
\begin{remark}[Milnor]
    If $X$ is a countable CW-complex, then $\Omega X$ is not a CW-complex, but
    it is \emph{homotopy} equivalent (not just weakly equivalent) to one.
    Moreover, $\Omega X$ is weakly equivalent to a topological group $GX$ such
    that $BGX\simeq X$.
\end{remark}
\subsection{Examples}
We claim that $BU(n)\simeq \Gra_n(\cC^\infty)$. To see this, let
$V_n(\cC^\infty)$ is the contractible space of complex $n$-frames in
$\cC^\infty$, i.e., isometric embeddings of $\cc^n$ into $\cc^\infty$. The
Lie group $U(n)$ acts principally on $V_n(\cC^\infty)$ by precomposition, and the
quotient $V_n(\cC^\infty)/U(n)$ is exactly the Grassmannian
$\Gra_n(\cC^\infty)$. As $\Gra_n(\cC^\infty)$ is the quotient of a principal action
of $U(n)$ on a contractible space, our discussion in the previous section
implies the desired claim.

Let $G$ be a compact Lie group (eg finite).
\begin{theorem}[Peter-Weyl]
    There exists an embedding $G\hookrightarrow U(n)$ for some $n$.
\end{theorem}
Since $U(n)$ acts principally on $V_n(\cC^\infty)$, it follows $G$ also acts
principally on $V_n(\cc^\infty)$. Therefore $V_n(\cc^\infty)/G$ is a model for
$BG$. It is not necessarily that this the most economic description of $BG$.

For instance, in the case of the symmetric group $\Sigma_n$, we have a much
nicer geometric description of the classifying space. Let
$\mathrm{Conf}_n(\RR^k)$ denote embeddings of $\{1,\cdots,n\}\to \RR^k$
(ordered distinct $n$-tuples). This space is definitely \emph{not}
contractible! However, the classifying space $\mathrm{Conf}_n(\RR^\infty)$ is
contractible. The symmetric group obviously acts freely on this (for finite
groups, a principal action is the same as a free action). It follows that
$B\Sigma_n$ is the space of \emph{un}ordered configurations of $n$ distinct
points in $\RR^\infty$. Using Cayley's theorem from classical group theory, we
find that if $G$ is finite, a model for $BG$ is the quotient
$\mathrm{Conf}_n(\RR^\infty)/G$.

We conclude this chapter with a construction of Eilenberg-Maclane spaces via
classifying spaces. If $A$ is a topological abelian group, then the
multiplication $\mu:A\times A\to A$ is a homomorphism. Applying the classifying
space functor begets a map $m:BA\times BA\to BA$. If $G$ is a finite group, then
$BA = K(A, 1)$. The map $m$ above gives a topological abelian group model for
$K(A, 1)$. There is nothing preventing us from iterating this construction: the
space $B^2 A$ sits in a fibration
$$BA \to EBA\simeq \ast \to B^2 A.$$
It follows from the long exact sequence in homotopy that the homotopy groups of
$B^2 A$ are the same as that of $BA$, but shifted up by one. Repeating this
procedure multiple times gives us an explicit model for $K(A,n)$:
$$B^n A = K(A, n).$$
