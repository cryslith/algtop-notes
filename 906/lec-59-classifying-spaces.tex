\section{Classifying spaces}
Let $\cU$ be a cover of $X$.
Let $\cCC\cU$ denote the ...
Let $p:P\to X$ be a principal $G$-bundle.
Say that $\cU$ trivializes $P$ if there are homeomorphisms $p^{-1}(U)\xrightarrow{\simeq} U\times G$ over $U$.
Equivalently, it's a trivialization of the pullback bundle $\epsilon^\ast P$:
\begin{equation*}
    \xymatrix{
	\epsilon^\ast P\ar[r]\ar[d] & P\ar[d]\\
	\coprod_{U\in\cU} U\ar[r]_{\epsilon} & X
    }
\end{equation*}
This is the same thing as a functor $\cCC(\cU)\xrightarrow{\gamma} G$.
On classifying spaces, we therefore get a map $X\xleftarrow{\simeq} \cCC(\cU)\xrightarrow{\gamma} BG$, with the left map given by $\epsilon$.
How unique is $\theta$?

Suppose I have two principal $G$-bundles over $X$:
\begin{equation*}
    \xymatrix{
	P\ar[r]^{\simeq} \ar[dr] & Q\ar[d]\\
	& X
    }
\end{equation*}
Let me put this off for a minute.

We want to vary the open cover.
Say that $\cV$ refines $\cU$ if for any $V\in \cU$, there exists $U\in\cU$ such that $V\subseteq U$.
A \emph{refinement} is a function $p:\cV\to\cU$ such that $V\subseteq p(V)$.
A refinement defines a map $\coprod_{V\in \cV} V\to \coprod_{U\in\cU} U$, also denoted $\rho$.
Both cover $X$.
So I get a map $\cCC(\cV)\to \cCC(\cU)$ that covers $cX$ (constant with only identity morphisms) that commutes strictly.
When I take classifying spaces, I get a diagram:
$$
\xymatrix{
    B\cCC(\cV)\ar[r]\ar[dr] & B\cCC(\cU)\ar[d]\\
    & X
}
$$
Suppose $t$ trivialization of $P$ for $\cU$.
Well, I have a functor $B\cCC(\cU)\to BG$, so I get a trivialization $s$ for $\cV$.
This is a map $s_V:p^{-1}(V)\to V\times G$ that works as:
\begin{equation*}
    \xymatrix{
	p^{-1}(V)\ar[r]^p\ar[d] & V\times G\ar[d]\\
	p^{-1}(\rho(V))\ar[r]_{t_{\rho(V)}} & \rho(V)\times G
    }
\end{equation*}
Just by construction, we have a larger commutative diagram:
\begin{equation*}
\xymatrix{
    B\cCC(\cV)\ar[r]\ar[dr]\ar@/^/[rr] & B\cCC(\cU)\ar[d]\ar[r] & BG\\
    & X
}
\end{equation*}
Suppose I have trivializations $\cU, t$ of $P$ and $\cV, s$ of $Q$.
Let $\cW$ be a common refinement.
I have a diagram:
\begin{equation*}
    \xymatrix{
	& \cCC(\cU)\ar[dr]^{\gamma^\cU_{P}} & \\
	\cCC(\cW)\ar[ur]\ar[dr]\ar@/^/[rr]^{\gamma^\cW_P}\ar@/_/[rr]_{\gamma^\cW_Q} & & G\\
	& \cCC(\cV)\ar[ur]_{\gamma^\cV_Q}
    }
\end{equation*}
I claim that there is a natural transformation $\beta:\gamma^\cW_P\to \gamma^\cW_Q$.
Then, the two maps $\theta_P,\theta_Q:B\cCC(\cW)\simeq X\to BG$ are homotopic.

OK, so what is $\beta$? It assigns to each object in $\cCC(\cW)$ some $x_W$.
At this point I didn't follow what was being said.

Ta-da!
This completes the story that $G$-bundles are represented by $BG$.
Let's see some examples then.
\subsection{Two general principles}
Let $G$ be some topological group.
Here is a summary of what we have done.
I constructed $EG$, which is contractible with $G$ acting freely on the right.
There is an orbit projection $EG\to BG$, and it's a principal $G$-bundle provided $G$ is, like, a Lie group.
For instance a discrete group.
And, it's universal, so that $\Bun_G(X)\xleftarrow{\simeq} [X, BG]$ given by $f\mapsto [f^\ast EG]$.

Suppose $G$ acts on $E$, say from the left.
Then let $E^{op}$ be $E$ as a space, but with $G$ acting on the left.
So $g\cdot e = e\cdot g^{-1}$.
I claim that we then have a fiber bundle:
\begin{equation*}
    E\to EG\times_G E\to BG
\end{equation*}
Note that I have to choose a point in $EG$ to get a map $E\to EG\times_G E$.

I could collapse $EG$ to a point instead, and I have a map $G\backslash E$.
The fiber of the map $EG\times_G E\to G\backslash E$, and the fiber over $G_x$ is $BG_x$, where $G_x = \{g: gx = x\}$.

What I want to say is that if $G$ acts principally on $E$, then $E\to E/G$ is a principal $G$-bundle.
We have a fiber bundle $EG\to EG\times_G E\to G\backslash E$.
Since $EG$ is contractible, then $EG\times_G E$ is weakly equivalent to $G\backslash E$.
If $E\simeq \ast$, then $p$ is also a weak equivalence.
Thus $BG\simeq G\backslash E$.
So if you come up with another principal action on a contractible space, then $BG$ is weakly equivalent to $G\backslash E$.
The construction with simplicial sets was a very concrete way of constructing $BG$, but you could have chosen any principal action on a contractible space.

Here's the second comment.
Suppose $X$ is a pointed path connected.
Remember that $X$ has a path space $PX = X^I_\ast$.
This is contractible by the spaghetti move.
The map $PX\to X$ is a fibration, with fiber $\Omega X$.

For instance, suppose $X = BG$.
Then, I have:
\begin{equation*}
    \xymatrix{
	G\ar[rr] & & \Omega BG\ar[d]\\
	EG\ast\ar[dr]\ar[rr] & & PBG\simeq \ast \ar[dl]\\
	& BG &
    }
\end{equation*}
A nullhomotopy of $EG\to BG$ (it is nullhomotopic) is exactly a lift into the path space, so I have the map $EG\to PBG$ in the diagram above.
As $EG$ and $PBG$ are both contractible, we find that $\Omega BG$ is weakly equivalent to $G$.
This weak equivalence is a $H$-map (it commutes up to homotopy with the multiplication on both sides).
\begin{remark}[Milnor]
    If $X$ is a countable CW-complex, then $\Omega X$ is homotopy equivalent to a CW-complex.
    It's not a CW-complex (any space is weakly equivalent to a CW-complex, but these are homotopy equivalent to a CW-complex).
\end{remark}
\begin{remark}[another Milnor remark]
    $\Omega X$ is weakly equivalent to a topological group $GX$ such that $BGX\simeq X$.
\end{remark}
\subsection{Examples}
I want to say that $BU(n)\simeq \Gra_n(\cc^\infty)$.
The reason is:
\begin{equation*}
    \xymatrix{
	U(n)\ar[d] & U(n)\ar[d]\\
	EU(n) \ar[d] \ar[r]^{\simeq} & V_n(\cc^\infty)\simeq \ast\ar[d]\\
	BU(n) \ar[r]_{\simeq} & \Gra_n(\cc^\infty)
    }
\end{equation*}
Here $V_n(\cc^\infty)$ is the contractible space of complex $n$-frames in $\cc^\infty$.
It's the same as isometric embeddings of $\cc^n$ into $\cc^\infty$.
$U(n)$ acts by precomposition.

Let $G$ be a compact Lie group (eg finite).
\begin{theorem}[Peter-Weyl]
    There exists an embedding $G\hookrightarrow U(n)$ for some $n$.
\end{theorem}
Thus, $G$ acts principally on $V_n(\cc^\infty)$, since $U(n)$ does.
Thus $V_n(\cc^\infty)/G$ is a model for $BG$!

For instance, what about $\Sigma_n$, acting on $n$-points?
I can imitate that linear model like this.
Let $\mathrm{Conf}_n(\RR^k)$ denote embeddings of $\{1,\cdots,n\}\to \RR^k$ (ordered distinct $n$-tuples).
These things are definitely not contractible.
The low-dimensional homotopy groups are zero, though.
So, $\mathrm{Conf}_n(\RR^\infty)\simeq \ast$.
The symmetric group obviously acts freely on this (same as principally for finite groups).
Thus, $B\Sigma_n$ is th space of unordered configurations of $n$ distinct points in $\RR^\infty$.
But, you know, any group acts freely on itself, and it embeds into a symmetric group.
So, like above, if $G$ is finite, a model for $BG$ is $\mathrm{Conf}_n(\RR^\infty)/G$.

One more thing.
If $A$ is a topological abelian group, then $\mu:A\times A\to A$ is a homomorphism.
So, I get a map $BA\times BA\to BA$ which is another abelian group structure.
We know what it is!
$BA = K(A, 1)$.
Its universal cover is contractible (it's $EA$).
I have a topological abelian group model for $K(A, 1)$.
Well, $B^2 A$ sits in a fibration $BA \to EBA\simeq \ast \to B^2 A$.
So the homotopy groups of $B^2 A$ are the same as that of $BA$, but shifted.
Doing this multiple times gives us a model:
$$B^n A = K(A, n)$$
It's one of the coolest things to do with the classifying space construction.
