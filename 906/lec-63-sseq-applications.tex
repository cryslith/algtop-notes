\section[Applications]{The homology of $\Omega S^n$, and the Serre exact sequence}
%Leray originally invented his spectral sequences for locally compact things,
%but $\Omega S^n$ for instance isn't locally compact.
The goal of this section is to describe a computation of the homology of
$\Omega S^n$ via the Serre spectral sequence, as well as describe a
``degenerate'' case of the Serre spectral sequence.

\subsection*{The homology of $\Omega S^n$}\label{loops-sn}
Let us first consider the case $n=1$. The space $\Omega S^1$ is the base of a
fibration $\Omega S^1 \to PS^1 \to S^1$. Comparing this to the fibration $\Z\to
\RR\to S^1$, we find that $\Omega S^1 \simeq \Z$. Equivalently, this follows
from the discussion in \S \ref{loops-bg} and the observation that $S^1 \simeq
K(\Z,1)$.

Having settled that case, let us now consider the case $n>1$. Again, there is a
fibration $\Omega S^n\to PS^n \to S^n$. In general, if $F\to E\to B$ is a
fibration and the space $F$ has torsion-free homology, we can (via the
universal coefficients theorem) rewrite the $E^2$-page:
$$E^2_{s,t} = H_s(B;H_t(F)) \simeq H_s(B)\otimes H_t(F).$$
Since $S^n$ has torsion-free homology, the Serre spectral sequence (see \S
\ref{serre-sseq}) runs:
$$E^2_{s,t} = H_s(S^n) \otimes H_t(\Omega S^n) \Rightarrow H_{\ast}(PS^n) =
\Z.$$
Since $H_s(S^n)$ is concentrated in degrees $0$ and $n$, we learn that
$E^2$-page is concentrated in columns $s=0,n$. For instance, if $n=4$, then the
$E^2$-page (without the differentials drawn in) looks like:
\sseqset{classes={draw=none}}
\begin{sseqdata}[name = {loops-Sn}, x range={0}{5}, y range={0}{5},
    homological Serre grading, differentials={->}, y axis gap=0.8cm, x label =
    {$H_\ast(S^4)$}, y label = {$H_\ast(\Omega S^4)$}]
    \foreach \y in {0,...,6} {\class["H_{\y}(\Omega S^n)"](0,\y)}
    \foreach \y in {0,...,6} {\class["H_{\y}(\Omega S^n)"](4,\y)}
    \foreach \y in {0,...,2} {\d["d^4"]4(4,\y)}
\end{sseqdata}
\begin{center}
    \printpage[name={loops-Sn},page=0,no differentials]
\end{center}
We know that $H_0(\Omega S^n) = \Z$. Since the target has homology concentrated
in degree $0$, we know that $E^2_{n,0}$ has to be killed. The only possibility
is that it is hit by a differential, or that it supports a nonzero
differential.

There are not very many possibilities for differentials in this spectral
sequence. In fact, up until the $E^n$-page, there are no differentials (either
the target or source of the differential is zero), so $E^2\simeq E^3 \simeq
\cdots \simeq E^n$. On the $E^n$-page, there is only one possibility for a
differential: $d^n:E^2_{n,0} \to E^n_{0,n-1}$. This differential has to be a
monomorphism because if it had anything in its kernel, that will be left over
in the position. In our example above (with $n=4$), we have
\begin{center}
    \printpage[name={loops-Sn},page=4]
\end{center}

However, we still do not know the group $E^n_{0,n-1}$. If it is bigger than
$\Z$, then $d^n$ is not surjective. There can be no other differentials on the
$E^r$-page for $r\geq n+1$ (because of sparsity), so the $d^n$ differential is
our last hope in killing everything in degree $(0,n-1)$.  This means that $d^n$
is an epimorphism. We find that $E^n_{0,n-1} = H_{n-1}(\Omega S^n) \simeq \Z$,
and that $d^n$ is an isomorphism.

We have now discovered that $H_{n-1}(\Omega S^n) \simeq \Z$ --- but there is a
lot more left in the $E^2$-page! For instance, we still have a $\Z$ in
$E^n_{n,n-1}$. Because $H^\ast(PS^n)$ is concentrated in degree $0$, this, too,
must die! We are in exactly the same situation as before, so the same arguments
show that the differential $d^n:E^n_{n,n-1}\to E^n_{0,2(n-1)}$ has to be an
isomorphism. Iterating this argument, we find:
\begin{equation*}
    H_q(\Omega S^n) \simeq \begin{cases}
        \Z & \text{if }(n-1)|q\geq 0\\
        0 & \text{else}
    \end{cases}
\end{equation*}
This is a great example of how useful spectral sequences can be.
\begin{remark}
    The loops $\Omega X$ is an associative $H$-space. Thus, as is the case for
    any $H$-space, the homology $H_\ast(\Omega X; R)$ is a graded associative
    algebra. Recall that the suspension functor $\Sigma$ is the left adjoint to
    the loops functor $\Omega$, so there is a unit map $A\to \Omega \Sigma A$.
    This in turn begets a map $\widetilde{H}_\ast(A)\to H_\ast(\Omega \Sigma A)$.
    
    Recall that the universal tensor algebra
    $\mathrm{Tens}(\widetilde{H}_\ast(A))$ is the free associative algebra on
    $\widetilde{H}_\ast(A)$. Explicitly:
    $$\mathrm{Tens}(\widetilde{H}_\ast(A)) = \bigoplus_{n\geq
    0}\widetilde{H}_\ast(A)^{\otimes n}.$$
    In particular, by the universal property of
    $\mathrm{Tens}(\widetilde{H}_\ast(A))$, we get a map
    $\alpha:\mathrm{Tens}(\widetilde{H}_\ast(A))\to H_\ast(\Omega \Sigma A)$.
    \begin{theorem}[Bott-Samelson]\label{bott-samelson}
	The map $\alpha$ is an isomorphism if $R$ is a PID and $H_\ast(A)$ is
	torsion-free.
    \end{theorem}
    For instance, if $A = S^{n-1}$ then $\Omega S^n = \Omega \Sigma A$. Theorem
    \ref{bott-samelson} then shows that
    $$
    H_\ast(\Omega S^n) = \mathrm{Tens}(\widetilde{H}_\ast(S^{n-1})) = \langle
    1, x, x^2, x^3, \cdots\rangle,
    $$
    where $|x| = n-1$. It is a mistake to call this ``polynomial'', since if
    $n$ is even, $x$ is an odd class (in particular, $x$ squares to zero by the
    Koszul sign rule).
\end{remark}
%Note that $H_\ast(\Omega S^n)$ acts on this spectral sequence.
%Then the $d^r$'s are linear (module homomorphisms).

Theorem \ref{bott-samelson} suggests thinking of $\Omega \Sigma A$ as the
``free associative algebra'' on $A$. Let us make this idea more precise.
\begin{remark}
    The space $\Omega A$ is homotopy equivalent to a topological monoid
    $\Omega_M A$, called the \emph{Moore loops} on $A$. This means that
    $\Omega_M A$ has a \emph{strict} unit and is \emph{strictly} associative
    (i.e., not just up to homotopy). Concretely,
    $$\Omega_M A := \{(\ell,\omega):\ell\in\RR_{\geq 0}, \omega:[0,\ell]\to A,
    \omega(0) = \ast = \omega(\ell)\},$$
    topologized as a subspace of the product. There is an identity class
    $1\in\Omega_M A$, given by $1 = (0,c_\ast)$ where $c_\ast$ is the constant
    loop at the basepoint $\ast$. The addition on this space is just given by
    concatenatation. In particular, the lengths get added; this overcomes the
    obstruction to $\Omega A$ not being strictly associative, so the Moore
    loops $\Omega_M A$ are indeed strictly associative.
    If the basepoint is nondegenerate, it is not hard to see that the inclusion
    $\Omega A\hookrightarrow \Omega_M A$ is a homotopy equivalence.

    Given the space $A$, we can form the free monoid $\mathrm{FreeMon}(A)$. The
    elements of this space are just formal sequences of elements of $A$
    (with topology coming from the product topology), and the multiplication is
    given by juxtaposition. Let us adjoin the element $1 = \ast$. As with all
    free constructions, there is a map $A\to\mathrm{FreeMon}(A)$ which is
    universal in the sense that any map $A\to M$ to a monoid factors through
    $\mathrm{FreeMon}(A)$.
    
    The unit $A\to\Omega\Sigma A$ is a map from $A$ to a monoid, so we get a
    monoid map $\beta:\mathrm{FreeMon}(A)\to\Omega\Sigma A$.
    \begin{theorem}[James]
	The map $\beta:\mathrm{FreeMon}(A)\to \Omega\Sigma A$ is a weak
	equivalence if $A$ is path-connected.
    \end{theorem}
    The free monoid looks very much like the tensor product, as the following
    theorem of James shows.
    \begin{theorem}[James]\label{james-splitting}
	Let $J(A) = \mathrm{FreeMon}(A)$. There is a splitting:
	$$
	\Sigma J(A) \simeq_w \Sigma \left(\bigvee_{n\geq 0}A^{\wedge n}\right).
	$$
    \end{theorem}
    Applying homology to the splitting of Theorem \ref{james-splitting} shows
    that:
    $$
    \wt{H}_\ast(J(A)) \simeq \bigoplus_{n\geq 0} \wt{H}_\ast(A^{\wedge n}).
    $$
    Assume that our coefficients are in a PID, and that $\wt{H}_\ast(A)$ is
    torsion-free; then this is just $\bigoplus_{n\geq 0}\wt{H}_\ast(A)^{\otimes
    n}$. In particular, we recover our computation of $H_\ast(\Omega S^n)$ from
    these general facts.
\end{remark}
\subsection{The Serre exact sequence}
Suppose $\pi:E\to B$ is a fibration over a path-connected base. Assume that
$\widetilde{H}_s(B) = 0$ for $s<p$ where $p\geq 1$. Let $\ast\in B$ be a chosen
basepoint. Denote by $F$ the fiber $\pi^{-1}(\ast)$. Assume $\widetilde{H}_t(F)
= 0$ for $t<q$, where $q\geq 1$.  We would like to use the Serre spectral
sequence to understand $H_\ast(E)$. As always, we will assume that $\pi_1(B)$
acts trivially on $H_\ast(F)$. 

Recall that the Serre spectral sequence runs
$$
E^2_{s,t} = H_s(B;H_t(F)) \Rightarrow H_{s+t}(E).
$$
Our assumptions imply that $E^2_{0,0}=\Z$, and $E^2_{0,t} = 0$ for $t<q$.
Moreover, $E^2_{s,0} = 0$ for $s<p$. In particular, $E^2_{0,q+t} = H_{q+t}(F)$
and $E^2_{p+k,0} = H_{p+k}(B)$ --- the rest of the spectral sequence is
mysterious.

By sparsity, the first possible differential is $d^{p}:H_p(B) \to H_{p-1}(F)$, and
$d^{p+q}:H_{p+1}(B)\to H_{p}(F)$. In the mysterious zone, there are
differentials that hit $E^2_{p,q}$.

Again by sparsity, the only differential is $d^s:E^s_{s,0}\to E^s_{0,s-1}$ for
$s<p+q-1$. This is called a \emph{transgression}. It is the last possible
differential which has a chance at being nonzero. This means that the cokernel
of $d^s$ is $E^\infty_{0,s-1}$. There is also a map $E^\infty_{s,0}\to
E^s_{s,0}$. We obtain a mysterious composite 
\begin{equation}\label{part1-serre-lexseq}
    0\to E^\infty_{s,0}\to E^s_{s,0} \simeq H_s(B) \xrightarrow{d^s}
    E^s_{0,s-1}\simeq H_{s-1}(F) \to E^\infty_{0,s-1}\to 0.
\end{equation}

Let $n<p+q-1$. Recall that $F_s H_n(E) =
\img(H_\ast(\pi^{-1}(\mathrm{sk}_s(B)))\to H_\ast(E))$, so $F_0 H_n(E) =
E^\infty_{0,n}$. Here, we are using the fact that $F_{-1} H_\ast(E) = 0$. In
particular, there is a map $E^\infty_{0,n}\to H_n(E)$. By our hypotheses, there
is only one other potentially nonzero filtration in this range of dimensions,
so we have a short exact sequence:
\begin{equation}\label{part2-serre-lexseq}
    0\to F_0 H_n(E) = E^\infty_{0,n} \to H_n(E) \to E^\infty_{n,0} \to 0
\end{equation}
Splicing the short exact sequences \eqref{part1-serre-lexseq} and
\eqref{part2-serre-lexseq}, we obtain a long exact sequence:
$$
H_{p+q-1}(F)\to \cdots\to H_n(F)\to H_n(E)\to H_n(B) \xrightarrow{\text{transgression}} H_{n-1}(F) \to H_{n-1}(E)\to \cdots
$$
This is called the \emph{Serre exact sequence}. In this range of dimensions,
homology behaves like homotopy.
