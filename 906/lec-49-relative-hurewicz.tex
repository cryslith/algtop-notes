\section{Relative Hurewicz and J.~H.~C.~Whitehead}
Here is an ``alternative definition'' of connectedness:
\begin{definition}
    Let $n\geq 0$.
    The space $X$ is said to be \emph{$(n-1)$-connected} if, for all $0\leq k\leq n$, any map $f:S^{k-1}\to X$ extends:
    \begin{equation*}
	\xymatrix{
	S^{k-1}\ar[d]\ar[r] & X\\
	D^k\ar@{-->}[ur]_\exists & 
	}
    \end{equation*}
\end{definition}
When $n=0$, we know that $S^{-1} = \emptyset$, and $D^0 = \ast$.
Thus being $(-1)$-connected is equivalent to being nonempty.
When $n=1$, this is equivalent to path connectedness. You can check that this is exactly the same as what we said before, using homotopy groups.

As is usual in homotopy theory, there is a relative version of this definition.
\begin{definition}
    Let $n\geq 0$. Say that a pair $(X,A)$ is \emph{$n$-connected} if, for all $0\leq k\leq n$, any map $f:(D^k,S^{k-1}) \to (X,A)$ extends:
    \begin{equation*}
	\xymatrix{
	    (D^k,S^{k-1})\ar[r]^f\ar@{-->}[d] & (X,A)\\
	    (A,A)\ar[ur] & 
	    }
    \end{equation*}
    up to homotopy.
    In other words, there is a homotopy between $f$ and a map with image in $A$, such that $f|_{S^{k-1}}$ remains unchanged.
\end{definition}
$0$-connectedness implies that $A$ meets every path component of $X$.
Equivalently:
    \begin{definition}
	$(X,A)$ is $n$-connected if:
	\begin{itemize}
	    \item when $n=0$, the map $\pi_0(A)\to \pi_0(X)$ surjects.
	    \item when $n>0$, the canonical map $\pi_0(A)\xrightarrow{\simeq}\pi_0(X)$ is an isomorphism,
		and for all $a\in A$, the group $\pi_k(X,A,a)$ vanishes for $1\leq k\leq n$.
		(Equivalently, $\pi_0(A)\xrightarrow{\simeq}\pi_0(X)$ and $\pi_k(A,a)\to\pi_k(X,A)$ is an isomorphism for $1\leq k<n$ and is onto for $k=n$.)
	\end{itemize}
    \end{definition}
\subsection{The relative Hurewicz theorem}
    Assume that $\pi_0(A) = \ast = \pi_0(X)$, and pick $a\in A$.
    Then, we have a comparison of long exact sequences, arising from the classical (i.e., non-relative) Hurewicz map:
    \begin{equation*}
	\xymatrix{
	    \cdots\ar[r] & \pi_1(A)\ar[r]\ar[d]^h & \pi_1(X)\ar[r]\ar[d]^h & \pi_1(X,A)\ar[r]\ar[d]^h & \pi_0(A)\ar[r]\ar[d]^h & \pi_0(X)\ar[d]^h & \\
	    \cdots\ar[r] & H_1(A)\ar[r] & H_1(X)\ar[r] & H_1(X,A)\ar[r] & H_0(A)\ar[r] & H_0(X)\ar[r] & H_0(X,A)
	    }
    \end{equation*}
To define the relative Hurewicz map, let $\alpha\in \pi_n(X,A)$, so that $\alpha:(D^n,S^{n-1})\to (X,A)$;
pick a generator of $H_n(D^n,S^{n-1})$, and send it to an element of $H_n(X,A)$ via the induced map
$\alpha_\ast:H_n(D^n,S^{n-1})\to H_n(X,A)$.

Because $H_n(X,A)$ is abelian, the group $\pi_1(A)$ acts trivially on $H_n(X,A)$; in other words,
$h(\omega(\alpha)) = h(\alpha)$.
Consequently, the relative Hurewicz map factors through the group $\pi_n^\dagger(X,A)$, defined to be
the quotient of $\pi_n(X,A)$ by the normal subgroup generated by $(\omega\alpha)\alpha^{-1}$,
where $\omega\in\pi_1(A)$ and $\alpha\in \pi_n(X,A)$.
This begets a map $\pi_n^\dagger(X,A)\to H_n(X,A)$.
\begin{theorem}[Relative Hurewicz]
    Let $n\geq 1$, and assume $(X,A)$ is $n$-connected.
    Then $H_k(X,A) = 0$ for $0\leq k\leq n$, and the map $\pi_{n+1}^\dagger(X,A)\to H_{n+1}(X,A)$ constructed above
    is an isomorphism.
\end{theorem}
We will prove this later using the Serre spectral sequence.
\subsection{The Whitehead theorems}
J.~H.~C.~Whitehead was a rather interesting character. He raised pigs.

Whitehead was interested in determining when a continuous map $f:X\to Y$ that is an isomorphism in homology or homotopy
is a homotopy equivalence.
\begin{definition}
    Let $f:X\to Y$ and $n\geq 0$. Say that $f$ is a \emph{$n$-equivalence}\footnote{Some sources sometimes use ``$n$-connected''.}
    if, for every $\ast\in Y$, the homotopy fiber $F(f,\ast)$ is $(n-1)$-connected.
\end{definition}
For instance, $f$ being a $0$-equivalence simply means that $\pi_0(X)$ surjects onto $\pi_0(Y)$ via $f$.
For $n>0$, this says that $f:\pi_0(X)\to \pi_0(Y)$ is a bijection, and that for every $\ast\in X$:
\begin{equation*}
    \pi_k(X,\ast)\to\pi_k(Y,f(\ast)) \text{ is }\begin{cases}
	\text{an isomorphism } & 1\leq k<n\\
	\text{onto } & k = n.
    \end{cases}
\end{equation*}
Using the ``mapping cylinder'' construction (see Exercise \ref{cofibrep}), we can always assume $f:X\to Y$ is a cofibration;
in particular, that $X\hookrightarrow Y$ is a closed inclusion.
Then, $f:X\to Y$ is an $n$-equivalence if and only if $(Y,X)$ is $n$-connected.
\begin{theorem}[Whitehead]
    Suppose $n\geq 0$, and $f:X\to Y$ is $n$-connected. Then:
    \begin{equation*}
	H_k(X)\xar{f} H_k(Y) \text{ is }\begin{cases}
	    \text{an isomorphism } & 1\leq k<n\\
	    \text{onto } & k = n.
	\end{cases}
    \end{equation*}
\end{theorem}
\begin{proof}
    When $n=0$, because $\pi_0(X)\to \pi_0(Y)$ is surjective, we learn that
    $H_0(X)\simeq \Z[\pi_0(X)]\to \Z[\pi_0(Y)]\simeq H_0(Y)$ is surjective.
    To conclude, use the relative Hurewicz theorem.
    (Note that the relative Hurewicz dealt with $\pi_n^\dagger(X,A)$, but the map $\pi_n(X,A)\to\pi_n^\dagger(X,A)$ is surjective.)
\end{proof}
The case $n=\infty$ is special.
\begin{definition}
    $f$ is a \emph{weak equivalence} (or an $\infty$-equivalence, to make it sound more impressive) if it's an $n$-equivalence for all $n$, i.e., it's a $\pi_\ast$-isomorphism.
\end{definition}
Putting everything together, we obtain:
\begin{corollary}
    A weak equivalence induces an isomorphism in integral homology.
\end{corollary}
How about the converse?

If $H_0(X)\to H_0(Y)$ surjects, then the map $\pi_0(X)\to \pi_0(Y)$ also surjects.
Now, assume $X$ and $Y$ path connected,  and that $H_1(X)$ surjects onto $H_1(Y)$.
We would like to conclude that $\pi_1(X)\to\pi_1(Y)$ surjects.
Unfortunately, this is hard, because $H_1(X)$ is the abelianization of $\pi_1(X)$.
To forge onward, we will simply give up, and assume that $\pi_1(X)\to \pi_1(Y)$ is surjective.

Suppose $H_2(X)\to H_2(Y)$ surjects, and that $f_\ast:H_1(X)\xrightarrow{\simeq}H_1(Y)$.
We know that $H_2(Y,X) = 0$.
On the level of the Hurewicz maps, we are still stuck, because we only obtain information about $\pi_2^\dagger$.
Let us assume that $\pi_1(X)$ is trivial\footnote{This is a pretty radical assumption; for the following argument to work,
it would technically be enough to ask that $\pi_1(X)$ acts trivially on $\pi_2(Y,X)$: but this is basically impossible to check.}.
Under this assumption, we find that $\pi_1(Y) = 0$.
This implies $\pi_2(Y,X)$ is trivial.
Arguing similarly, we can go up the ladder.
\begin{theorem}[Whitehead]
    Let $n\geq 2$, and assume that $\pi_1(X) = 0 = \pi_1(Y)$.
    Suppose $f:X\to Y$ such that:
    \begin{equation*}
	H_k(X)\to H_k(Y) \text{ is }\begin{cases}
	    \text{an isomorphism } & 1\leq k<n\\
	    \text{onto } & k = n;
	\end{cases}
    \end{equation*}
    then $f$ is an $n$-equivalence.
\end{theorem}
Setting $n=\infty$, we obtain:
\begin{corollary}
    Let $X$ and $Y$ be simply-connected.
    If $f$ induces an isomorphism in homology, then $f$ is a weak equivalence.
\end{corollary}
This is incredibly useful, since homology is actually computable!
To wrap up the story, we will state the following result, which we will prove in a later section.
\begin{theorem}\label{weakhtpyequiv}
    Let $Y$ be a CW-complex.
    Then a weak equivalence $f:X\to Y$ is in fact a homotopy equivalence.
\end{theorem}
