\section{Exact couples}
Let us begin with a conceptual discussion of exact couples. As a special case,
we will recover the construction of the spectral sequence associated to a
filtered chain complex (Theorem-Definition \ref{filtered-sseq}).
\begin{definition}
    An \emph{exact couple} is a diagram of (possilby (bi)graded) abelian
    groups
    \begin{equation*}
	\xymatrix{
	    A\ar[rr]^i & & A\ar[dl]_j\\
	    & E\ar[ul]_k & 
	    }
    \end{equation*}
    which is exact at each joint.
\end{definition}
As $jkjk = 0$, the map $E\xar{jk} E$ is a differential, denoted $d$. An exact
couple determines a ``derived couple'':
\begin{equation}\label{derived-couple}
    \xymatrix{
	A^\prime\ar[rr]^{i^\prime = i|_{\img i}} & &
	A^\prime\ar[dl]_{j^\prime}\\
	& E^\prime\ar[ul]_{k^\prime} & 
        }
\end{equation}
where $A^\prime = \img(i)$ and $E^\prime = H_\ast(E,d)$. Iterating this
procedure, we get exact sequences 
\begin{equation*}
    \xymatrix{
	A^r\ar[rr]^{i_r} & & A^r\ar[dl]_{j_r}\\
	& E^r\ar[ul]_{k_r} & 
        }
\end{equation*}
where the next exact couple is the derived couple of the preceding exact
couple.

It remains to define the maps in the above diagram. Define $j^\prime(ia) = ja$.
\emph{A priori}, it is not clear that this well-defined. For one, we need $[ja]
\in E^\prime$; for this, we must check that $dja = 0$, but $d=jk$, and $jkja =
0$ so this follows. We also need to check that $j^\prime$ is well-defined
modulo boundaries. To see this, suppose $ia = 0$. We then need to know that
$ja$ is a boundary. But if $ia = 0$, then $a = ke$ for some $e$, so $ja = jke =
de$, as desired.

Define $k^\prime:H(E,d) \to \img i$ via $k^\prime([e])\mapsto ke$. As before,
we need to check that this is well-defined. For instance, we have to check that
$ke\in\img i$. Since $de = 0$ and $d = jk$, we learn that $jke = 0$. Thus $ke$
is killed by $j$, and therefore, by exactness, is in the image of $i$. We also
need to check that $k^\prime$ is independent of the choice of representative of
the homology class. Say $e = de^\prime$. Then $kd = kd e^\prime = kjke^\prime =
0$.
\begin{exercise}
    Check that these maps indeed make diagram \eqref{derived-couple} into an
    exact couple.
\end{exercise}
It follows that we obtain a spectral sequence, in the sense of
Theorem-Definition \ref{filtered-sseq}.
\begin{exercise}\label{explicit-er}
    By construction,
    $$A^r = \img(i^{r}|_{A}) = i^{r} A.$$
    Show, by induction, that
    $$E^r = \frac{k^{-1}(i^{r}A)}{j(\ker i^{r})}$$
    and that
    \begin{equation*}
	i_r(a) = ia,\ j_r(i^r a) = [ja],\ k_r(e) = ke.
    \end{equation*}
\end{exercise}
Intuitively: an element of $E^1$ will survive to $E^r$ if its image in $A^1$
can be pulled back under $i^{r-1}$. The differential $d^r$ is obtained by
the homology class of the pushforward of this preimage via $j$ to $E^1$.
\begin{remark}\label{bigraded}
    In general, the groups in consideration will be bigraded. It is clear by
    construction that $\deg(i') = \deg(i)$, $\deg(k') = \deg(k)$, and $\deg(j')
    = \deg(j)-\deg(i)$. It follows by an easy inductive argument that
    $$\deg(d^r) = \deg(j)+\deg(k)-(r-1)\deg(i).$$
\end{remark}
The canonical example of an exact couple is that of a filtered complex; the
resulting spectral sequence is precisely the spectral sequence of
Theorem-Definition \ref{filtered-sseq}. If $C_\ast$ is a filtered chain
complex, we let $A_{s,t} = H_{s+t}(F_s C_\ast)$, and $E^1_{s,t} = E_{s,t} =
H_{s+t}(\gr_s C_\ast)$. The exact couple is precisely that which arises from
the long exact sequence in homology associated to the short exact sequence of
chain complexes
$$0\to F_{s-1} C_\ast\to F_s C_\ast \to \gr_s C_\ast\to 0.$$
Note that in this case, the exact couple is one of bigraded groups, so Remark
\ref{bigraded} dictates the bidegrees of the differentials.

We will conclude this section with a brief discussion of the convergence of the
spectral sequence constructed above.  Assume that $i:A\to A$ satisfies the
property that
$$\ker(i)\cap\bigcap i^r A = 0.$$
Let $\wt{A}$ be the colimit of the directed system
$$A\xar{i} A\xar{i} A\to\cdots$$
There is a natural filtration on $\wt{A}$. Let $I$ denote the image of the map
$A\to \wt{A}$; the kernel of this map is $\bigcup \ker(i^r)$. The groups $i^r
I$ give an exhaustive filtration of $\wt{A}$, and the quotients $i^r I/i^{r+1}
I$ are all isomorphic to $I/i I$ (since $i$ is an isomorphism on $\wt{A}$).
Then we have an isomorphism
\begin{equation}\label{einfty}
    E^\infty \simeq I/iI.
\end{equation}
Indeed, we know from Exercise \ref{explicit-er} that
$$E^\infty \simeq \frac{k^{-1}\left(\bigcap i^r A\right)}{j\left(\bigcup \ker
i^{r}\right)};$$
by our assumption on $i$, this is
$$\frac{\ker(k)}{j\left(\bigcup \ker i^{r}\right)} \simeq
\frac{j(A)}{j\left(\bigcup \ker i^r\right)}.$$
But there is an isomorphism $A/iA\to j(A)$ which clearly sends $iA + \bigcup
\ker i^r$ to $j\left(\bigcup \ker i^r\right)$. By our discussion above,
$A/\bigcup \ker i^r \simeq I$, and $iA/\bigcup \ker i^r \simeq iI$. Modding out
by $iI$ on both sides, we get \eqref{einfty}.

%Our starting point is that you have a chain complex $F_\ast C$.
%We have a short exact sequence $0\to F_{s-1} C\to F_s C \to \gr_s C \to 0$, giving a lexseq in homology:
%$$
%\cdots\to H_{s+t}(F_{s-1}) \to H_{s+t}(F_s) \to H_{s+t}(\gr_s) \to H_{s+t-1}(F_{s-1}) \to \cdots
%$$
%We let $A^1_{s,t} = H_{s+t}(F_{s})$ and $E^1_{s,t} = H_{s+t}(\gr_s)$.
%\textbf{Note: }Weibel uses this notation for something else.
%His $D$ is my $A$, but with different indexing.
%
%Anyway, our diagram can now be rewritten as:
%$$
%\cdots\to A^1_{s-1,t+1} \xar{i} A^1_{s,t} \xar{j} E^1_{s,t} \xar{k} A_{s-1, t} \to \cdots
%$$
%Define:
%$$
%A^r_{s,t} := \img(A^1_{s,t} \xar{i^{r-1}} A^1_{s+r - 1, t-r + 1}) = A^1_{s,t}/\ker(i^{r-1})
%$$
%One thing you can do is look at:
%$$
%\cdots \xar{i} \to A_{s-1} \xar{i} A_s \xar{i} A_{s+1} \xar{i} A_{s+2} \xar{i} \cdots
%$$
%We also have a surjection $\img(i^{r-1}) \to \img(i^r)$ and an injection $i:\img(i^r) \to \img(i^{r-1})$, giving a map $\img(i^{r-1}) \to \img(i^{r-1})$, given by $i$.
%There's another thing you can do: you also have a map $\img(i^r) \to \img(i^{r-1}) \to \img(i^r)$, whose composite is $i$ itself.
%Explicitly, we can write:
%\begin{equation*}
%    \xymatrix{
%	& A^r_{s,t} \ar@{->>}[dr]\ar[rr]^i & & A^r_{s+1,t-1}\ar@{->>}[dr] & \\
%	A^{r+1}_{s-1,t+1}\ar@{>->}[ur]\ar[rr]_i & & A^{r+1}_{s,t}\ar@{>->}[ur]\ar[rr]_i & & A^{r+1}_{s+1,t-1}
%    }
%\end{equation*}
%Great, so in our lexseq, we now have:
%\begin{equation*}
%    \xymatrix{
%	\cdots\ar[r] & A^1_{s-1,t+1}\ar@{->>}[d] \ar[r]^i & A^1_{s,t} \ar@{->>}[d] \ar[r]^j & E^1_{s,t} \ar[r]^k & A^1_{s-1,t}\ar[r]^i & A^1_{s,t-1} \ar[r] & \cdots\\
%	& A^2_{s-1,t+1}\ar@{->>}[d]\ar[r]^i & A^2_{s,t}\ar@{->>}[d] & & A^2_{s-2,t+1}\ar[r]^i \ar@{>->}[u] & A^2_{s-1,t}\ar@{>->}[u] & \\
%	& A^3_{s-1,t+1}\ar[r]^i\ar@{->>}[d] & A^3_{s,t}\ar@{->>}[d] & & A^3_{s-3,t+2}\ar@{>->}[u]\ar[r]^i & A^3_{s-2,t+1}\ar@{>->}[u] &\\
%	& \vdots & \vdots & & \vdots \ar@{>->}[u] & \vdots\ar@{>->}[u] &
%    }
%\end{equation*}
%Recall that for a filtered complex, we really have a surjection:
%$$
%A^r_{s,t} = \img(H_{s+t}(F_s) \to H_{s+t}(F_{s+r-1}) \to \img(H_{s+t}(F_s) \to H_{s+1}(C)) = F_s H_{s+t}(C)
%$$
%In particular, all the vertical surjections in our big diagram maps down surjectively to the filtration.
%More precisely:
%\begin{equation*}
%    \xymatrix{
%	\cdots\ar[r] & A^1_{s-1,t+1}\ar@{->>}[d] \ar[r]^i & A^1_{s,t} \ar@{->>}[d] \ar[r]^j & E^1_{s,t} \ar[r]^k & A^1_{s-1,t}\ar[r]^i & A^1_{s,t-1} \ar[r] & \cdots\\
%	& A^2_{s-1,t+1}\ar@{->>}[d]\ar[r]^i & A^2_{s,t}\ar@{->>}[d] & & A^2_{s-2,t+1}\ar[r]^i \ar@{>->}[u] & A^2_{s-1,t}\ar@{>->}[u] & \\
%	& A^3_{s-1,t+1}\ar[r]^i\ar@{->>}[d] & A^3_{s,t}\ar@{->>}[d] & & A^3_{s-3,t+2}\ar@{>->}[u]\ar[r]^i & A^3_{s-2,t+1}\ar@{>->}[u] &\\
%	& \vdots\ar@{->>}[d] & \vdots\ar@{->>}[d] & & \vdots \ar@{>->}[u] & \vdots\ar@{>->}[u] &\\
%	0\ar[r] & F_{s-1} H_{s+t}(C) \ar[r]^i & F_s H_{s+t}(C) \ar[r] & \gr_s H_{s+t}(C)\ar[r] & 0 & & &
%    }
%\end{equation*}
%Note that if $F_\ast C$ is exhaustive, then this filtration on homology is exhaustive, and, well, what it says is that $F_s H_{s+t}(C) = \colim A^r_{s,t}$.
%Also, if $F_\ast C$ is bounded below ($F_{-1}(C) = 0$), then $A^1_{s,t} = 0$ for $s\leq -1$.
%Eventually, the groups in the vertical injections will be $0$, so that's good.
%So, see, all I need to do now is fill in the missing column beneath $E^1_{s,t}$.
%In particular:
%\begin{equation*}
%    \xymatrix{
%	\cdots\ar[r] & A^1_{s-1,t+1}\ar@{->>}[d] \ar[r]^i & A^1_{s,t} \ar@{->>}[d] \ar[r]^j & E^1_{s,t} \ar[r]^k & A^1_{s-1,t}\ar[r]^i & A^1_{s,t-1} \ar[r] & \cdots\\
%	& A^2_{s-1,t+1}\ar@{->>}[d]\ar[r]^i & A^2_{s,t}\ar@{->>}[d]\ar[r] & E^2_{s,t}\ar[r] & A^2_{s-2,t+1}\ar[r]^i \ar@{>->}[u] & A^2_{s-1,t}\ar@{>->}[u] & \\
%	& A^3_{s-1,t+1}\ar[r]^i\ar@{->>}[d] & A^3_{s,t}\ar@{->>}[d] \ar[r] & E^3_{s,t}\ar[r] & A^3_{s-3,t+2}\ar@{>->}[u]\ar[r]^i & A^3_{s-2,t+1}\ar@{>->}[u] &\\
%	& \vdots\ar@{->>}[d] & \vdots\ar@{->>}[d] & \vdots & \vdots \ar@{>->}[u] & \vdots\ar@{>->}[u] &\\
%	0\ar[r] & F_{s-1} H_{s+t}(C) \ar[r]^i & F_s H_{s+t}(C) \ar[r] & \gr_s H_{s+t}(C)\ar[r] & 0 & 0 & 0 &
%    }
%\end{equation*}
%We want to construct in the $E^r$ groups.
%One way to do this is by exact couples, which is the easiest approach.
%This is what I'll do.
%
%On Wednesday, we'll do examples.
%Today was the guts, and the applications will come next week.
