\section{Examples of CW-complexes}
\subsection{Bringing you up-to-speed on CW-complexes}
\begin{definition}
    A \emph{relative CW-complex} is a pair $(X,A)$, together with a filtration
    $$A=X_{-1}\subseteq X_0\subseteq X_1\subseteq\cdots\subseteq X,$$
    such that for all $n$, the space $X_n$ sits in a pushout square:
    $$
    \xymatrix{
	\coprod_{\alpha\in \Sigma_n}S^{n-1}\ar[r]\ar[d]_{\text{attaching maps}} & \coprod_{\alpha\in \Sigma_n}D^n\ar[d]^{\text{characteristic maps}}\\
	X_{n-1}\ar[r] & X_n,
    }
    $$
    and $X=\varinjlim X_n$.
\end{definition}
If $A=\emptyset$, this is just the definition of a CW-complex.
In this case, $X$ is also compactly generated.
(This is one of the reasons for defining compactly generated spaces.)
Often, $X$ will be a CW-complex, and $A$ will be a subcomplex.
If $A$ is Hausdorff, then so is $X$.

If $X$ and $Y$ are both CW-complexes, define
$$(X\times^k Y)_n = \bigcup_{i+j = n}X_i\times Y_j;$$
this gives a CW-structure on the product $X\times^k Y$.
Any closed smooth manifold admits a CW-structure. 

\begin{example}[Complex projective space]
    The complex projective $n$-space $\CP^n$ is a CW-complex, with skeleta $\CP^0\subseteq\CP^1\subseteq\cdots\subseteq \CP^n$.
    Indeed, any complex line through the origin meets the hemisphere defined by
    $\begin{pmatrix}z_0\\\vdots\\z_n\end{pmatrix}$ with $||z||=1$, $\Im(z_n) = 0$, and $\Re(z_n)\geq 0$.
	Such a line meets this hemisphere (which is just $D^{2n}$) at one point --- unless it's on the equator;
	this gives the desired pushout diagram:
    \begin{equation*}
	\xymatrix{
	    S^{2n-1}\ar[r]\ar[d] & D^{2n}\ar[d]\\
	    \CP^{n-1}\ar[r] & \CP^n.
	    }
    \end{equation*}
\end{example}
\begin{example}[Grassmannians]
    Let $V=\RR^n$ or $\cC^n$ or $\mathbf{H}^n$, for some fixed $n$.
    Define the Grassmannian $\Gra_k(\RR^n)$ to be the collection of $k$-dimensional subspaces of $V$.
    This is equivalent to specifying a $k\times n$ rank $k$ matrix.

    %just linalg here, nothing to see
    %The span of rows of $A$ is the row space $V_A$. This is the span of the rows of the reduced reduced echelon form of $A$. An entry is a \emph{pivot} if its leftmost nonzero is in its row. A column is \emph{pivotal} if it contains a pivot. Any matrix is reduced echelon if the $i$th pivotal column is $e_i$.

    For instance, $\Gra_2(\RR^4)$ is, as a set, the disjoint union of:
    \begin{equation*}
	\begin{pmatrix}& 1 & \\ & & 1\end{pmatrix},\begin{pmatrix}&1&\ast\\&&1\end{pmatrix},\begin{pmatrix}1&\ast&\ast\\&&1\end{pmatrix},\begin{pmatrix}&1&\ast\\&1&\ast\end{pmatrix},\begin{pmatrix}1&\ast&\ast\\&1&\ast\end{pmatrix},\begin{pmatrix}1&\ast&\ast\\1&\ast&\ast\end{pmatrix}.
    \end{equation*}
    Motivated by this, define:
    \begin{definition}
	The $j$-skeleton of $\Gra(V)$ is
	$$\mathrm{sk}_j\Gra_k(V) = \{A:\text{row echelon representation with at most $j$ free entries}\}.$$
    \end{definition}
    For a proof that this is indeed a CW-structure, see \cite[\S 6]{milnorstasheff}.
    %They don't know it in 18.06, but they're constructing a CW-structure for the Grassmannian.
\end{example}
The top-dimensional cell tells us that
$$\dim\Gra_k(\RR^n) = k(n-k).$$
The complex Grassmannian has cells in only even dimensions.
We know the homology of Grassmannians: Poincar\'e duality is visible if we count the number of cells.
(Consider, for instance, in $\Gra_2(\RR^4)$).
