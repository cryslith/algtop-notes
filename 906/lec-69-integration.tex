\section{Integration, Gysin, Euler, Thom}\label{gysin-sequence}
Today
there's a talk by

the
one

the
only

JEAN-PIERRE SERRE

OK let's begin.
\subsection{Umkehr}
Let $\pi:E \to B$ be a fibration and suppose $B$ is path-connected.
Suppose the fiber has no cohomology above some dimension $d$.
The Serre sseq has nothing above row $d$.

Let's look at $H^n(E)$.
This happens along total degree $n$.
We have this neat increasing filtration that I was talking about on Monday whose associated quotients are the rows in this thing.
So I can divide out by it (i.e I divide out by $F_{d-1} H^n(E)$).
Then I get 
$$H^n(E) \fib H^n(E) / F_{d-1} H^n(E) = E^{n-d,d}_\infty \cofib E_2^{n-d,d} = H^{n-d}(B;\underline{H^d(F)})$$
That's because on the $E_2$ page, at that spot, there's nothing hitting it, but there might be a differential hitting it.
There it is; here's another edge homomorphism.

\begin{remark}
This is a \emph{wrong-way map}, also known as an ``umkehr'' map.
It's also called a \emph{pushforward map}, or the \emph{Gysin map}.
\end{remark}

We know from the incomprehensible discussion that I was giving on Monday that this was a filtration of modules over $H^\ast(B)$, so that this map $H^n(E) \to H^{n-d}(B;\underline{H^d(F)})$ is a $H^\ast(B)$-module map.

\begin{example}
    $F$ is a compact connected $d$-manifold with a given $R$-orientation.
    Thus $H^d(F) \simeq R$, given by $x\mapsto \langle x, [F]\rangle$.
    There might some local cohomology there, but I do get a map $H^n(E;R) \to H^{n-d}(B;\overline{R})$.
    This is such a map, and it has a name: it's written $\pi_!$ or $\pi_\ast$.
    I'll write $\pi_\ast$.
    
    Of course, if $\pi_1(B)$ fixes $[F]\in H_d(F;R)$, then $\underline{R}$-cohomology is $R$-cohomology.
    Thus our map is now $H^n(E;R) \to H^{n-d}(B;R)$.
    Sometimes it's also called a pushforward map.
    Note that we also get a projection formula
    $$
    \pi_\ast(\pi^\ast(b)\cup e) = b\cup \pi_\ast(e)
    $$
    where $\pi^\ast$ is the pushforward, $e\in H^n(E)$ and $b\in H^s(B)$.
    Others call this Frobenius reciprocity.
\end{example}

\subsection{Gysin}

Suppose $H^\ast(F) = H^\ast(S^{n-1})$.
In practice, $F\cong S^{n-1}$, or even $F\simeq S^{n-1}$.
In that case, $\pi:E\to B$ is called a \emph{spherical fibration}
Then the spectral sequence is \emph{even simpler}!
It has only two nonzero rows!

Let's pick an orientation for $S^{n-1}$, to get an isomorphism $H^{n-1}(S^{n-1})$.
Well the spectral sequence degenerates, and you get a long exact sequence
$$
\cdots \to H^s(B) \xar{\pi^\ast} H^s(E) \xar{\pi_\ast} H^{s-n+1}(B;\underline{R}) \xar{d_n} H^{s+1}(B) \xar{\pi^\ast} H^{s+1}(E) \to \cdots
$$
That's called the \emph{Gysin sequence}\footnote{pronounced Gee-sin}.
Because everything is a module over $H^\ast(B)$, this is a lexseq of $H^\ast(B)$-modules.

Let me be a little more explicit.
Suppose we have an orientation.
We now have a differential $H^0(B) \to H^n(B)$.
We have the constant function $1\in H^0(B)$, and this maps to something in $B$.
This is called the \emph{Euler class}, and is denoted $e$.

Since $d_n$ is a module homomorphism, we have $d_n(x) = d_n(1\cdot x) = d_n(1) \cdot x = e \cdot x$ where $x$ is in the cohomology of $B$.
Thus our lexseq is of the form
$$
\cdots \to H^s(B) \xar{\pi^\ast} H^s(E) \xar{\pi_\ast} H^{s-n+1}(B;\underline{R}) \xar{e\cdot - } H^{s+1}(B) \xar{\pi^\ast} H^{s+1}(E) \to \cdots
$$
\subsection{Some facts about the Euler class}
Suppose $E\to B$ has a section $\sigma:B\to E$ (so that $\pi\sigma = 1_B$).
So, if it came from a vector bundle, I'm asking that there's a nowhere vanishing cross-section of that vector bundle.
Let's apply cohomology, so that you get $\sigma^\ast \pi^\ast = 1_{H^\ast(B)}$.
Thus $\pi^\ast$ is monomorphic.
In terms of the Gysin sequence, this means that $H^{s-n}(B)\xar{e\cdot -} H^s(B)$ is zero.
But this implies that
$$\boxed{e = 0}$$
Thus, if you don't have a nonzero Euler class then you cannot have a section!
If your Euler class is zero sometimes you can conclude that your bundle has a section, but that's a different story.

The Euler class of the tangent bundle of a manifold when paired with the fundamental class is the Euler characteristic.
More precisely, if $M$ is oriented connected compact $n$-manifold, then
$$
\langle e(\tau_M), [M] \rangle = \chi(M)
$$
That's why it's called the Euler class.
(He didn't know about spectral sequences or cohomology.)
\subsection{Time for Thom}
This was done by Rene Thom.
Let $\xi$ be a $n$-plane bundle over $X$.
I can look at $H^\ast(E(\xi), E(\xi) - \text{section})$.
If I pick a metric, this is $H^\ast(D(\xi), S(\xi))$, where $D(\xi)$ is the disk bundle\footnote{$D(\xi) = \{v\in E(\xi):||v||\leq 1\}$.} and $S(\xi)$ is the sphere bundle.
If there's no point-set annoyance, this is $\widetilde{H}^\ast(D(\xi)/S(\xi))$.

If $X$ is a compact Hausdorff space, then ...
The open disk bundle $D^0(\xi) \simeq E(\xi)$.
This quotient $D(\xi)/S(\xi) = E(\xi)^+$ since you get the one-point compactification by embedding into a compact Hausdorff space ($D(\xi)$ here) and then quotienting by the complement (which is $S(\xi)$ here).
This is called the \emph{Thom space} of $\xi$.
There are two notations: some people write $\mathrm{Th}(\xi)$, and some people (Atiyah started this) write $X^\xi$.

\begin{example}[Dumb]
    Suppose $\xi$ is the zero vector bundle.
    Then your fibration is $\pi:X \to X$.
    What's the Thom space?
    The disk bundle is $X$, and the boundary of a disk is empty, so $\mathrm{Th}(0) = X^0 = X\sqcup \ast$.
\end{example}

The Thom space is a pointed space (corresponding to $\infty$ or the point which $S(\xi)$ is collapsed to).

I'd like to study its cohomology, because it's interesting.
There's no other justification.
Maybe I'll think of it as the relative cohomology.

So, guess what?
We've developed sseqs and done cohomology.
Anything else we'd like to do to groups and functors and things?

Let's make the spectral sequence relative!

I have a path connected $B$, and I'll study:
\begin{equation*}
    \xymatrix{
	F_0 \ar[d] \ar@{^(->}[r] & F\ar[d]\\
	E_0\ar[d] \ar@{^(->}[r] & E\ar[d]\\
	B\ar@{=}[r] & B
    }
\end{equation*}
Then if you sit patiently and work through things, we get
$$
E^{s,t}_2 = H^s(B; H^t(F, F_0)) \Rightarrow_s H^{s+t}(E, E_0)
$$
Note that $\Rightarrow_s$ means that $s$ determines the filtration.

Let's do this with the Thom space.
We have $D(\xi) \xar{\simeq} X$.
That isn't very interesting.
In our case, we have an incredibly simple spectral sequence, where everything on the $E_2$-page is concentrated in row $n$.
Thus the $E_2$ page \emph{is} the cohomology of
$$\widetilde{H}^{s+n}(\Th(\xi)) = H^{s+n}(D(\xi),S(\xi)) \simeq H^s(B; \underline{R})$$
where $\underline{R} = \underline{H^n(D^n,S^{n-1})}$.
This is a canonical isomorphism of $H^\ast(B)$-modules.

Suppose your vector bundle $\xi$ is oriented, so that $\underline{R} = R$.
Now, if $s=0$, then I have $1\in H^0(B)$.
This gives $u\in H^n(\Th(\xi))$, which is called the \emph{Thom class}.

The cohomology of $B$ is a free module of rank one over $H^\ast(B)$, so that $H^\ast(\Th(\xi))$ is also a $H^\ast(B)$-module that is free of rank $1$, generated by $u$.

Let me finish by saying one more thing.
This is why the Thom space is interesting.
Notice one more thing: there's a lexseq of a pair
$$
\cdots\to \widetilde{H}^s(\Th(\xi)) \to H^s(D(\xi)) \to H^s(S(\xi)) \to \widetilde{H}^{s+1}(\Th(\xi)) \to \cdots
$$
We have synonyms for these things:
$$
\cdots \to H^{s-n}(X) \to H^s(X) \to H^s(S(\xi)) \to H^{s-n+1}(X) \to \cdots
$$
And aha, this is is exactly the same form as the Gysin sequence.
Except, oh my god, what have I done here?

Yeah, right!
In the Gysin sequence, the map $H^{s-n}(X) \to H^s(X)$ was multiplication by the Euler class.
The Thom class $u$ maps to some $e^\prime\in H^n(X)$ via $\widetilde{H}^n(D(\xi),S(\xi)) \to H^n(D(\xi)) \simeq H^n(X)$.
And the map $H^{s-n}(X) \to H^s(X)$ is multiplication by $e^\prime$.
Guess what?
This is the Gysin sequence.

You'll explore more in homework.

I'll talk about characteristic classes on Friday.
