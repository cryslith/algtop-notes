\section{Classifying spaces: the Grassmann model}\label{grassmannmodel}
We will now shift our focus somewhat and talk about classifying spaces for
principal bundles and for vector bundles. We will do this in two ways: the
first will be via the Grassmann model and the second via simplicial methods.
\begin{lemma}\label{embedding}
    Over a compact Hausdorff space, any $n$-plane bundle embeds in a trivial
    bundle.
\end{lemma}
\begin{proof}
    Let $\cU$ be a trivializing open cover of the base $B$; since $B$ is
    compact, we may assume that $\cU$ is finite with $k$ elements. There is no
    issue with numerability, so there is a subordinate partition of unity $\phi_i$.
    Consider an $n$-plane bundle $E\to B$. By trivialization, there is a
    fiberwise isomorphism $p^{-1}(U_i)\xrightarrow{f_i}\RR^n$ where the $U_i\in
    \cU$. A map to a trivial bundle is the same thing as a bundle map in the
    following diagram:
    \begin{equation*}
	\xymatrix{
	    E\ar[r]\ar[d] & \RR^N\ar[d]\\
	    B\ar[r] & \ast
	    }
    \end{equation*}
    We therefore define $E\to (\RR^n)^k$ via
    $$e\mapsto (\phi_i(p(e))f_i(e))_{i=1,\cdots,k}.$$
    This is a fiberwise linear embedding, generally called a ``Gauss map''.
    Indeed, observe that this map has no kernel on every fiber, so it is an
    embedding.
\end{proof}
The trivial bundle has a metric on it, so choosing the orthogonal complement of
the embedding of Lemma \ref{embedding}, we obtain:
\begin{corollary}
    Over a compact Hausdorff space, any $n$-plane bundle has a complement (i.e.
    a $\xi^\perp$ such that $\xi\oplus\xi^\perp$ is trivial).
\end{corollary}
Another way to say this is that if $B$ is a compact Hausdorff space with an
$n$-plane bundle $\xi$, there is a map $f:X\to\Gra_n(\RR^{kn})$; this is
exactly the Gauss map. It has the property that taking the pullback
$f^\ast\gamma^n$ of the tautologous bundle over $\Gra_n(\RR^{kn})$ gives back
$\xi$. 

In general, we do not have control over the number $k$. There is an easy fix to
this problem: consider the tautologous bundle $\gamma^n$ over
$\Gra_n(\RR^\infty)$, defined as the union of $\Gra_n(\RR^m)$ and given the
limit topology. This is a CW-complex of finite type (i.e. finitely many cells
in each dimension). Note that $\Gra_n(\RR^m)$ are not the $m$-skeleta of
$\Gra_n(\RR^\infty)$!

The space $\Gra_n(\RR^\infty)$ is ``more universal'':
\begin{lemma}\label{universal}
    Any (numerable) $n$-plane bundle is pulled back from $\gamma^n\downarrow
    \Gra_n(\RR^\infty)$ via the Gauss map.
\end{lemma}
Lemma \ref{universal} is a little bit tricky, since the covering can be wildly
uncountable; but this is remedied by the following bit of point-set topology.
\begin{lemma}\label{sublemma}
    Let $\cU$ be a numerable cover of $X$. Then there's another numerable cover
    $\cU^\prime$ such that:
    \begin{enumerate}
	\item the number of open sets in $\cU^\prime$ is countable, and
	\item each element of $\cU^\prime$ is a disjoint union of elements of
	    $\cU$.
    \end{enumerate}
\end{lemma}
If $\cU$ is a trivializing cover, then $\cU^\prime$ is also a trivializing
cover.
\begin{proof}
    See \cite[Proposition 3.5.4]{husemoller}.
\end{proof}
It is now an exercise to deduce Lemma \ref{universal}. The main result of this
section is the following.
\begin{theorem}
    The map $[X,\Gra_n(\RR^\infty)]\to \Vect_n(X)$ defined by $[f]\mapsto
    [f^\ast\gamma^n]$ is bijective, where $[f]$ is the homotopy class of $f$
    and $[f^\ast\gamma^n]$ is the isomorphism class of the bundle
    $f^\ast\gamma^n$.
\end{theorem}
This is why $\Gra_n(\RR^\infty)$ is also called the \emph{classifying space}
for $n$-plane bundles. The Grassmannian provides a very explicit geometric
description for the classifying space of $n$-plane bundles. There is a more
abstract way to produce a classifying space for principal $G$-bundles, which
we will describe in the next section; the Grassmannian is the special case when
$G = \GL_n(\RR)$.
\begin{proof}
    We have already shown surjectivity, so it remains to prove injectivity.
    Suppose $f_0,f_1:X\to \Gra_n(\RR^\infty)$ such that $f_0^\ast\gamma^n$ and
    $f_1^\ast\gamma^n$ are isomorphic over $X$. We need to construct a homotopy
    $f_0\simeq f_1$. For ease of notation, let us identify $f_0^\ast\gamma^n$
    and $f_1^\ast\gamma_n$ with each other; call it $\xi:E\downarrow X$.

    The maps $f_i$ are the same thing as Gauss maps $g_i:E\to\RR^\infty$, i.e.,
    maps which are fiberwise linear embeddings. The homotopy $f_0\simeq f_1$ is
    created by saying that we have a homotopy from $g_0$ to $g_1$ through Gauss
    maps, i.e., through other fiberwise linear embeddings.
    
    In fact, we will prove a much stronger statement: \emph{any two} Gauss maps
    $g_0,g_1:E\to \RR^\infty$ are homotopic through Gauss maps. This is very far from
    true if I didn't have a $\RR^\infty$ on the RHS there.

    Let us attempt (and fail!) to construct an affine homotopy between $g_0$
    and $g_1$. Consider the map $tg_0 + (1-t)g_1$ for $0\leq t\leq 1$. In order
    for these maps to define a homotopy via Gauss maps, we need the following
    statement to be true: for all $t$, if $tg_0(v) + (1-t)g_1(v) =
    0\in\RR^\infty$, then $v=0$.  In other words, we need $tg_0+(1-t)g_1$ to be
    injective. Of course, this is not guaranteed from the injectivity of $g_0$
    and $g_1$!

    Instead, we will construct a composite of affine homotopies between $g_0$
    and $g_1$ using the fact that $\RR^\infty$ is an infinite-dimensional
    Euclidean space. Consider the following two linear isometries:
    \begin{equation*}
	\xymatrix{
	    & \RR^\infty = \langle
	    e_0,e_1,\cdots\rangle\ar[dl]^\alpha_{e_i\mapsto
	    e_{2i}}\ar[dr]_\beta^{e_i\mapsto e_{2i+1}} & \\
	    \RR^\infty & & \RR^\infty
	    }
    \end{equation*}
    Then, we have four Gauss maps: $g_0$, $\alpha\circ g_0$, $\beta\circ g_1$,
    and $g_1$. There are affine homotopies through Gauss maps:
    $$g_0\simeq \alpha \circ g_0\simeq \beta\circ g_1\simeq g_1.$$
    We will only show that there is an affine homotopy through Gauss maps $g_0
    \simeq \alpha \circ g_0$; the others are left as an exercise. Let $t$ and
    $v$ be such that $tg_0(v) + (1-t)\alpha g_0(v) = 0$. Since $g_0$ and
    $\alpha g_0$ are Gauss maps, we may suppose that $0<t<1$. Since $\alpha
    g_0(v)_i$ has only even coordinates, it follows by definition of the map
    $\alpha$ that $g_0(v)$ only had nonzero coordinates only in dimensions
    congruent to $0$ mod $4$. Repeating this argument proves the desired
    result.
\end{proof}
