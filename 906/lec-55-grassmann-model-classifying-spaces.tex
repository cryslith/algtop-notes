\section{Classifying spaces: the Grassmann model}
Office hours:Hood from 12 in 2-390, and me from 4-5 on Tuesday in 2-478.

I want to talk about classifying spaces for principal bundles and for vector bundles. I'm going to do this in two ways, one via the Grassmann model and via simplicial methods.
\begin{lemma}
    Over a compact Hausdorff space, any $n$-plane bundle embeds in a trivial bundle.
\end{lemma}
\begin{proof}
    Let $\cU$ be a trivializing open cover; since it's compact, we might as well assume that it's finite with $k$ elements.
    There's no numerability issue, so there's a subordinate partition of unity $\phi_i$.
    My bundle is $E\to X$. By trivialization, there is a fiberwise isomorphism $p^{-1}(U_i)\xrightarrow{f_i}\RR^n$.
    Note that a map to a trivial bundle is the same thing as a bundle map:
    \begin{equation*}
	\xymatrix{
	    E\ar[r]\ar[d] & \RR^?\ar[d]\\
	    B\ar[r] & \ast
	    }
    \end{equation*}
    Define $E\to (\RR^n)^k$ via $e\mapsto (\phi_i(p(e))f_i(e))_{i=1,\cdots,k}$. This is a fiberwise linear embedding, generally called a ``Gauss map''.
    You have to observe that this thing has no kernel on every fiber, so this is an embedding.
\end{proof}
\begin{corollary}
    Over a compact Hausdorff space, any $n$-plane bundle has a complement (i.e. a $\xi^\perp$ such that $\xi\oplus\xi^\perp = \mathrm{trivial}$), because the trivial bundle has a metric on it, so I just choose the orthogonal complement of the embedding.
\end{corollary}
Another way to say this is that if $X$ is compact Hausdorff (henceforth called CHS) with an $n$-plane bundle $\xi$, I've constructed a map $f:X\to\Gra_n(\RR^{kn})$; this is exactly the Gauss map.
It has the property that I can take the pullback $f^\ast\gamma^n$ of the tautologous bundle over $\Gra_n(\RR^{kn})$, then I get back $\xi$.
I don't have control over $k$, but in fact I can just consider $\gamma^n$ over $\Gra_n(\RR^\infty)$, topologized as the union of $\Gra_n(\RR^m)$, where $\RR^\infty = \bigoplus^\infty\RR$. This is a CW-complex of finite type (i.e. finitely many cells in each dimension)! Note that $\Gra_n(\RR^m)$ are not the $m$-skeleta of $\Gra_n(\RR^\infty)$.
It's more universal -- any $n$-plane bundle $\xi$ over $X$ can be obtained as the pull back $f^\ast\gamma^n$ for the Gauss map $f:X\to\Gra_n(\RR^{kn})\to\Gra_n(\RR^\infty)$.
\begin{lemma}\label{universal}
    Any (numerable) $n$-plane bundle is pulled back from $\gamma^n\downarrow \Gra_n(\RR^\infty)$.
\end{lemma}
This is a little bit tricky, since the covering can be wildly uncountable. To show this, let's use:
\begin{lemma}\label{sublemma}
    Let $\cU$ be a numerable cover of $X$. Then there's another numerable cover $\cU^\prime$ such that:
    \begin{enumerate}
	\item the number of open sets in $\cU^\prime$ is countable.
	\item Each element of $\cU^\prime$ is a disjoint union of elements of $\cU$.
    \end{enumerate}
\end{lemma}
If $\cU$ is a trivializing cover, then $\cU^\prime$ is also a trivializing cover.
\begin{proof}
    If you don't want to work it out you can refer to Husemoller's book, p.28. I'll put a link on the website.
\end{proof}
I think that basically proves Lemma \ref{universal}. Don't be deceived! Beware that you can't necessarily expect a complementary bundle.
\begin{theorem}
    The map $[X,\Gra_n(\RR^\infty)]\to \Vect_n(X)$ via $[f]\mapsto [f^\ast\gamma^n]$ is bijective, where $[f]$ is the homotopy class and $[f^\ast\gamma^n]$ is the iso class.
\end{theorem}
That's why $\Gra_n(\RR^\infty)$ is called the \emph{classifying space} for $n$-plane bundles.

This is a very explicit geometric description of classifying spaces. There's an abstract way to produce a classifying space for principal $G$-bundles, which we'll talk about on Wednesday. This is the special case of $\GL_n(\RR)$.
\begin{proof}
    I've already shown surjectivity; we need to show injectivity. Suppose $f_0,f_1:X\to \Gra_n(\RR^\infty)$ such that $f_0^\ast\gamma^n\simeq f_1^\ast\gamma^n$ over $X$.
    I'd like to see that $f_0\simeq f_1$.
    I might as well identify $f_0^\ast\gamma^n$ and $f_1^\ast\gamma_n$ with each other; call it $\xi:E\downarrow X$. What do I have?

    Giving $f_i$ is the same thing as giving a map $g_i:E\to\RR^\infty$ that's a fiberwise linear embedding.
    The $g$ is for Gauss.
    This homotopy $f_0\simeq f_1$ is created by saying that we have a homotopy from $g_0$ to $g_1$ through Gauss maps, i.e., through other fiberwise linear embeddings.
    I'm going to do this without any reference from where $g_0$ came from. I'm saying that \emph{any two} Gauss maps are homotopic through Gauss maps -- that's what I'm claiming! I mean, from this given vector bundle $E$.
    This is very far from true if I didn't have a $\RR^\infty$ on the RHS there.

    In fact, let's try for an affine homotopy. Think about $tg_0 + (1-t)g_1$ for $0\leq t\leq 1$. Now, wouldn't it be great if that works? I guess it can't.
    We need that for all $t$, if $tg_0(v) + (1-t)g_1(v) = 0\in\RR^\infty$, then $v=0$ (where $v\in E$). I.e., on each fiber, this map is injective, i.e., has zero kernel. Of course, I can't guarantee that just from $g_0$ and $g_1$ are injective!
    So we fail.

    We should use the fact that this is an infinite-dimensional Euclidean space. I'll give you two linear isometries:
    \begin{equation*}
	\xymatrix{
	    & \RR^\infty = \langle e_0,e_1,\cdots\rangle\ar[dl]^\alpha_{e_i\mapsto e_{2i}}\ar[dr]_\beta^{e_i\mapsto e_{2i+1}} & \\
	    \RR^\infty & & \RR^\infty
	    }
    \end{equation*}
    We have four Gauss maps: $g_0$, $\alpha\circ g_0$, $\beta\circ g_1$, and $g_1$.
    I claim that each of these are homotopic, i.e., $g_0\simeq \alpha \circ g_0\simeq \beta\circ g_1\simeq g_1$ via affine homotopies through Gauss maps.

    Shall we do just one case of that? I'll do the middle case. In the middle case, you'll never have cancellation.

    Let's try $g_0\simeq\alpha\circ g_0$. Suppose $t,v$ are such that $tg_0(v) + (1-t)\alpha g_0(v) = 0$. What's the argument I'll make here? So $\alpha g_0(v)$ has only even coordinates. So I might as well suppose that $0<t<1$. Anyway, since $\alpha g_0(v)$ has only even coordinates, $g_0(v)$ itself must have only even coordinates. But then that means that the nonzero coordinates only for $0\mod 4$. But this equation says that $g_0(v)\neq 0$ only for $i\equiv 0\mod 4$. So I repeat this and find that the coordinates are zero for powers of two.
\end{proof}
Maybe that's all I want to say today. I'll just stop today unless people have questions.
