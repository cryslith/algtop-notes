\section{Vector bundles, principal bundles}
Let $X$ be a topological space. A point in $X$ can be viewed as a map $\ast\to
X$; this is a cross section of the canonical map $X\to \ast$. Motivated by
this, we will define a vector space over $B$ to be a space $E\to B$ over $B$
with the following extra data:
\begin{itemize}
    \item a multiplication $\mu:E\times_B E\to E$, compatible with the maps
	down to $B$;
    \item a ``zero'' section $s:B\to E$ such that the composite $B\xar{s}E\to
	B$ is the identity;
    \item an inverse $\chi:E\to E$, compatible with the map down to $B$; and
    \item an action of $\RR$:
	\begin{equation*}
	    \xymatrix{
		\RR\times E\ar[dr]_{p\circ \pr_2}\ar@{=}[r] &
		(B\times\RR)\times_B E\ar[r]\ar[d] & E\ar[dl]^p\\
		& B &
	    }
	\end{equation*}
\end{itemize}
Because $\RR$ is a field, the last piece of data shows that $p^{-1}(b)$ is a
$\RR$-vector space for any point $b\in B$.
\begin{example}\label{trivialvectorbundle}
    A rather silly example of a vector space over $B$ is the projection
    $B\times V\to B$ where $V$ is a (real) vector space, which we will always
    assume to be finite-dimensional.
\end{example}
\begin{example}
    Consider the map
    $$\RR\times\RR\xrightarrow{(s,t)\mapsto(s,st)}\RR\times\RR,$$
    over $\RR$ (the structure maps are given by projecting onto the first
    factor). It is an isomorphism on all fibers, but is zero everywhere else.
    The kernel is therefore $0$ everywhere, except over the point $0\in\RR$.
    This the ``skyscraper'' vector bundle over $B$.
\end{example}
Sheaf theory accommodates examples like this.

One can only go so far you can go with this simplistic notion of a ``vector
space'' over $B$. Most interesting and naturally arising examples have a little
more structure, which is exemplified in the following definition.
\begin{definition}
    A \emph{vector bundle} over $B$ is a vector space over $B$ that is locally
    trivial (in the sense of Definition \ref{fiberbundle}).
\end{definition}
\begin{remark}
    We will always assume that the space $B$ admits a numerable open cover (see
    Definition \ref{numerable}) which trivializes the vector bundle. Moreover,
    the dimension of the fiber will always be finite.
\end{remark}
If $p:E\to B$ is a vector bundle, then $E$ is called the \emph{total space},
$p$ is called the \emph{projection map}, and $B$ is called the \emph{base
space}. We will always use a Greek letter like $\xi$ or $\zeta$ to denote a
vector bundle, and $E(\xi)\to B(\xi)$ denotes the actual projection map from
the total space to the base space. The phrase ``$\xi$ is a vector bundle over
$B$'' will also be shortened to $\xi\downarrow B$.
\begin{example}\label{grassmannianvb}
\begin{enumerate}
    \item Following Example \ref{trivialvectorbundle}, one example of a vector
	bundle is the trivial bundle $B\times\RR^n\to B$, denoted by
	$n\epsilon$.
    \item In contrast to this silly example, one gets extremely interesting
	examples from the Grassmannians $\Gra_k(\RR^n)$, $\Gra_k(\cC^n)$, and
	$\Gra_k(\HH^n)$. For simplicity, let $K$ denote $\RR,\cC$, or $\HH$.
	Over $\Gra_k(K^n)$ lies the \emph{tautological bundle} $\gamma$. This
	is a sub-bundle of $n\epsilon$ (i.e., the fiber over any point $x\in
	\Gra_k(K^n)$ is a subspace of the fiber of $n\epsilon$ over $x$). The
	total space of $\gamma$ is defined as:
	\begin{equation*}
	    E(\gamma) = \{(V,x)\in\Gra_k(K^n)\times K^n:x\in V\}
	\end{equation*}
	This projection map down to $\Gra_k(K^n)$ is the literal projection map
	$$(V,x)\mapsto V.$$
	\begin{exercise}
	    Prove that $\gamma$, as defined above, is locally trivial; so
	    $\gamma$ defines a vector bundle over $\Gra_k(K^n)$.
	\end{exercise}
	    For instance, when $k=1$, we have $\Gra_1(\RR^n) = \RP^{n-1}$. In
	    this case, $\gamma$ is one-dimensional (i.e., the fibers are all of
	    dimension $1$); this is called a \emph{line bundle}. In fact, it is
	    the ``canonical line bundle'' over $\RP^{n-1}$.
    \item Let $M$ be a smooth manifold. Define $\tau_M$ to be the tangent
	bundle $TM\to M$ over $M$. For example, if $M = S^{n-1}$, then
	$$TS^{n-1} = \{(x,v)\in S^{n-1}\times\RR^n:v\cdot x = 0\}.$$
\end{enumerate}
\end{example}
\subsection{Constructions with vector bundles}
One cannot take the kernels of a map of vector bundles; but just about
anything which can be done for vector spaces can also be done for vector
bundles:
\begin{enumerate}
    \item Pullbacks are legal: if $p^\prime:E^\prime\to B^\prime$, then the
	leftmost map in the diagram below is also a vector bundle.
	\begin{equation*}
	    \xymatrix{
		E\ar[r]\ar[d] & E^\prime\ar[d]^{p^\prime}\\
		B\ar[r]_f & B^\prime
		}
	\end{equation*}
	For instance, if $B=\ast$, the pullback is just the fiber of $E^\prime$
	over the point $\ast\to B^\prime$. If $\xi$ is the bundle $E^\prime\to
	B^\prime$, we denote the pullback $E\to B$ as $f^\ast \xi$.
    \item If $p:E\to B$ and $p^\prime:E^\prime\to B^\prime$, then we can take
	the product $E\times E^\prime\xrightarrow{p\times p^\prime}B\times
	B^\prime$.
    \item If $B=B^\prime$, we can form the pullback:
	\begin{equation*}
	    \xymatrix{
		E\oplus E^\prime\ar[r]\ar[d] & E\times E^\prime\ar[d]\\
		B\ar[r]_{\Delta} & B\times B
		}
	\end{equation*}
	The bundle $E\oplus E^\prime$ is called the \emph{Whitney sum}. For
	instance, it is an easy exercise to see that
	$$n\epsilon = \epsilon\oplus\cdots\oplus\epsilon.$$
    \item If $E,E^\prime\to B$ are two vector bundles over $B$, we can form
	another vector bundle $E\otimes_\RR E^\prime\to B$ by taking the
	fiberwise tensor product. Likewise, taking the fiberwise Hom begets a
	vector bundle $\Hom_\RR(E,E^\prime)\to B$.
\end{enumerate}
\begin{example}
    Recall from Example \ref{grassmannianvb}(2) that the tautological bundle
    $\gamma$ lives over $\RP^{n-1}$; we will write $L = E(\gamma)$. The tangent
    bundle $\tau_{\RP^{n-1}}$ also lives over $\RP^{n-1}$. As this is the first
    explicit pair of vector bundles over the same space, it is natural to
    wonder what is the relationship between these two bundles.
    
    At first glance, one might guess that $\tau_{\RP^{n-1}} = \gamma^\perp$;
    but this is false! Instead,
    $$\tau_{\RP^{n-1}} = \Hom(\gamma,\gamma^\perp).$$
    To see this, note that we have a $2$-fold covering map $S^{n-1}\to
    \RP^{n-1}$; therefore, $T_x(\RP^{n-1})$ is a quotient of $T(S^n)$ by the
    map sending $(x,v)\mapsto (-x,-v)$, where $v\in T_x(S^n)$. Therefore,
    $$T_x\RP^{n-1} = \{(x,v)\in S^{n-1}\times\RR^n:v\cdot x =
    0\}/((x,v)\sim(-x,-v)).$$
    This is exactly the fiber of $\Hom(\gamma,\gamma^\perp)$ over $x\in
    \RP^{n-1}$, since the line through $x$ can be mapped to the line through
    $\pm v$.
\end{example}
\begin{exercise}
    Prove that if $\gamma$ is the tautological vector bundle over
    $\Gra_k(K^n)$, for $K=\RR,\cC,\HH$, then
    $$\tau_{\Gra_k(K^n)} = \Hom(\gamma,\gamma^\perp).$$
\end{exercise}
\subsection{Metrics and splitting exact sequences}
A \emph{metric} on a vector bundle is a continuous choice of inner products on
fibers.
\begin{lemma}
    Any vector bundle $\xi$ over $X$ admits a metric.
\end{lemma}
Intuitively speaking, this is true because if $g,g^\prime$ are both inner
products on $V$, then $tg+(1-t)g^\prime$ is another. Said differently, the
space of metrics forms a real affine space.
\begin{proof}
    Pick a trivializing open cover of $X$, and a subordinate partition of
    unity. This means that we have a map $\phi_U:U\to [0,1]$, such that the
    preimage of the complement of $0$ is $U$. Moreover,
    $$\sum_{x\in U} \phi_U(x) = 1.$$
    Over each one of these trivial pieces, pick a metric $g_U$ on $E|_{U}$.
    Let
    $$g \coloneqq \sum_{U}\phi_U g_U;$$
    this is the desired metric on $\xi$.
\end{proof}
We remark that, in general, one cannot pick metrics for vector bundles. For
instance, this is the case for vector bundles which arise in algebraic
geometry.
\begin{definition}
    Suppose $E,E^\prime\to B$ are vector bundles over $B$. An
    \emph{isomorphism} is a map $\alpha:E\to E^\prime$ over $B$ that is a
    linear isomorphism on each fiber.
\end{definition}
In particular, the map $\alpha$ admits an inverse (over $B$).
\begin{corollary}
    Any exact\footnote{This is the obvious definition.} sequence $0\to
    E^\prime\to E\to E^{\prime\prime}\to 0$ of vector bundles (over the same
    base) splits.
\end{corollary}
\begin{proof}[Proof sketch]
    Pick a metric for $E$. Consider the composite
    $${E^\prime}^\perp\subseteq E\to E^{\prime\prime}.$$
    This is an isomorphism: the dimensions of the fibers are the same. It
    follows that
    $$E\cong E^\prime\oplus {E^\prime}^{\perp}\cong E^\prime\oplus
    E^{\prime\prime},$$
    as desired.
\end{proof}
Note that this splitting is not natural.
