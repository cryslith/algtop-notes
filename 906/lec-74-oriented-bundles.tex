\section{Oriented bundles, Pontryagin classes, Signature theorem}
Missed the first part.

\todo{stuff that I missed}
A map $\xi:X\to BO(n)$ is orientable iff $w_1(\xi) = 0$, i.e., the map $X\xar{\xi} BO(n)\to B\{\pm 1\}$ is nullhomotopic, since the map $BO(n) \to B\Z/2\Z\simeq\RP^\infty$ is an element of $H^1(BO(n);\FF_2)$.
This means that it factors as
$$
\xymatrix{
    & BSO(n)\ar[r]\ar[d]^{\text{double cover}} & S^\infty\ar[d]\\
    X\ar@{-->}[ur]\ar[r]_{\xi} & BO(n)\ar[r]_{w_1} & B\Z/2\Z
}
$$
Well, we have a story on the Lie group side, where we have $SO(n)\to O(n)\to\Z/2\Z$, which gives our $BSO(n)\to BO(n)\to \RP^\infty$ on the topological side.
For $n\geq 3$, I can kill the $\pi_1$ of $BSO(n)$ to get $Spin(n)$, which gives a double cover $Spin(n)\to SO(n)$.
I can take its classifying space to get $BSpin(n)$, and, well, we have a cofiber sequence $BSO(n)\xar{w_2} K(\Z/2\Z,2)$.
So, if $w_2(\xi) = 0$, I get a further lift, called a \emph{spin structure}.

I can do this again.
This gives you the \emph{string group}.
Note that this is just the Whitehead tower.
$$
\xymatrix{
    & BString(n)\ar[d] & \\
    & BSpin(n)\ar[d] \ar[r]^{p_1/2} & K(\Z,4)\\
    & BSO(n)\ar[r]\ar[d]^{\text{double cover}} & K(\Z/2\Z,2)\\
    X\ar@{-->}[ur]^{\text{orientation}}\ar@{-->}[uur]^{\text{spin structure}}\ar@{-->}[uuur]^{\text{string structure}}\ar[r]_{\xi} & BO(n)\ar[r]_{w_1} & B\Z/2\Z
}
$$
Note that $String(n)$ is no longer a Lie group since you have an infinite dimensional summand $K(\Z,2)$.

Let's compute the cohomology of $BSO(n)$.
Like I said, $BSO(n)\to BO(n)$ is a double cover, with fiber $S^0$.
It therefore has a Gysin sequence:
$$
0\to H^q(BO(n)) \xar{w_1} H^{q+1}(BO(n)) \xar{\pi^\ast} H^{q+1}(BSO(n)) \to 0
$$
since $w_1$ is a nonzero divisor.
This means that
$$
H^\ast(BSO(n)) = \FF_2[w_2,\cdots,w_n]
$$
That was easy.
But it's not easy to compute $H^\ast(BSpin(n))$ and $H^\ast(BString(n))$.

We want to try to understand integral characteristic classes for oriented bundles.
The trick is to use Chern classes.
Suppose $C_X$ is a complex vector space.
I can get an oriented real vector space, with choice of ordered basis given by $e_1,ie_1,\cdots,e_2,ie_2$.
This has an involution $-$, given by the opposite orientation.
I can also tensor up to $\cc$ to get a complex vector space from a real vector space.

How does $\overline{V}_\RR$ relate to $V_\RR$?
The orientation has changed, although it's the same vector space.
In fact, you have $\overline{V}_\RR = (-1)^{\dim V} V_\RR$.

I can also consider $(V\otimes_\RR\cc)_\RR$.
This is just $V\oplus V$, and we get that $(V\otimes_\RR\cc)_\RR = (-1)^{\dim V\cdot(\dim V - 1)/2} V\oplus V$.

Another one.
We know that $\overline{(V\otimes_\RR\cc)}\simeq V\otimes_\RR\cc$.

We're going to use those identities.

Let $\xi$ be a complex $n$-plane bundle.
What is $c(\overline{\xi})$?
They occur as coefficients in this identity $\sum c_i(\overline{\xi}) e(\lambda_{\overline{\xi}})^{n-i} = 0$, where $\lambda_{\overline{\xi}}\downarrow \PP(\overline{\xi}) = \PP(\xi)$.
Now, well, $\lambda_{\overline{\xi}} = \overline{\lambda_\xi}$.
In particular, we get that $e(\lambda_{\overline{\xi}})^{n-i} = (-1)^{n-i}e(\lambda_\xi)^{n-i}$, because $\lambda_\xi$ is one-dimensional.
It follows that
$$
0 = \sum^n_{i=0}c_i(\overline{\xi})e(\overline{\lambda_\xi})^{n-i} = \sum^n_{i=0}c_i(\overline{\xi}) (-1)^{n-i}e(\lambda_\xi)^{n-i} = (-1)^n e(\lambda_\xi)^n + \cdots
$$
This is \emph{not} monic, and hence doesn't define the Chern classes of $\overline{\xi}$.
The monic identity is going to be (multiply that identity by $(-1)^n$)
$$
\sum^n_{i=0}(-1)^ic_i(\overline{\xi})e(\lambda_\xi)^{n-i} = 0
$$
This means that
$$
\boxed{c_i(\overline{\xi}) = (-1)^ic_i(\xi)}
$$
Now, the idea is, that I don't know what to do for characteristic classes real bundles.
We can always tensor it up with $\cc$, though, i.e., I can look at $c_i(\xi\otimes\cc)$.
Let's do an experiment.
We know that $c_i(V\otimes\cc) = c_i(\overline{V\otimes\cc}) = (-1)^ic_i(V\otimes\cc)$, by the identities above.
So, we have a problem if $i$ is odd, namely that $2c_{odd}(V\otimes\cc) = 0$.
That's fine, but I want integral classes.
So the thing to do is to just ignore the prime $2$.

We know what the cohomology of $BSO(n)$ is in $\FF_2$, and now I'm going to tell you what $H^\ast(BSO(n))$ is with coefficients in a $\Z[1/2]$-algebra (so $1/2$ exists).
Then those odd Chern classes are just zero.
\begin{definition}[Pontryagin classes]
    Let $\xi$ be a real $n$-plane vector bundle.
    Then $p_k(\xi) = (-1)^kc_{2k}(\xi\otimes\cc)\in H^{4k}(X;R)$.
\end{definition}
Notice that this is $0$ if $2k>n$, since $\xi\otimes\cc$ is of complex dimension $n$.
The Whitney sum formula now says that:
$$
(-1)^k p_k(\xi\oplus\eta) = \sum_{i+j = k}(-1)^i p_i(\xi) (-1)^j p_j(\eta) = (-1)^k\sum_{i+j=k}p_i(\xi)p_j(\eta)
$$
So I can cross out the $(-1)^k$.

If $\xi$ is an oriented real $2k$-plane bundle, then you can calculate that
$$
p_k(\xi) = e(\xi)^2\in H^{4k}(X;R)
$$

Here's the table:
\begin{center}
    \begin{tabular}{ c|c c c c c c} 
	\hline
	$\ast = $ & 2 & 4 & 6 & 8 & 10 & 12\\
	\hline
	$H^\ast(BSO(2))$ & $e_2$ & $(e_2^2)$ & & & &\\
	$H^\ast(BSO(3))$ & & $p_1\uparrow$ & & & &\\
	$H^\ast(BSO(4))$ & & $p_1,e_4$ & & $(e_4^2)$ & &\\
	$H^\ast(BSO(5))$ & & $p_1$ & & $p_2\uparrow$ & &\\
	$H^\ast(BSO(6))$ & & $p_1$ & $e_6$ & $p_2$ & & $(e_6^2)$\\
	$H^\ast(BSO(7))$ & & $p_1$ & & $p_2$ & & $p_3\uparrow$
    \end{tabular}
\end{center}
In the limiting case, you get a polynomial algebra.
\subsection{Applications}
\begin{theorem}[Wall]
    Let $M^n,N^n$ be oriented manifolds.
    Then I can ask for \emph{oriented cobordisms}, i.e., $\partial W = M\sqcup -N$.
    If all Stiefel-Whitney numbers and Pontryagin numbers coincide, then $M$ is oriented cobordant to $N$.
\end{theorem}
The real excitement comes from the Signature theorem.
Let $M^{4k}$ be an oriented $4k$-manifold.
I have a pairing $H^{2k}(M)/\mathrm{torsion}\otimes H^{2k}(M)/\mathrm{tors}\to \Z$ given by $x\otimes y\mapsto\langle x\cup y,[M]\rangle$.
Poincar\'e duality says that this is a perfect pairing.
This, I have a nonsingular symmetric bilinear form.
Well, this is some large-dimensional free abelian group, which I can tensor with the reals to get a nonsingular symmetric bilinear form on a real vector space.
All you need to know is that you can diagonalize this, so that the diagonal entries are $\pm 1$.
The number of $1$s minus the number of $-1$s is called the \emph{signature} of that bilinear form.
That is called the signature of the manifold, if the form comes from the perfect pairing.
\begin{lemma}[Thom]
    The signature is an oriented bordism invariant.
\end{lemma}
This is an easy thing to prove using Lefschetz duality, which is a deep theorem.
On general principles, it follows that it's given by some ... of characteristic numbers.
\begin{theorem}[Signature theorem, due to Hirzebruch]
    There exists a rational polynomial $L_k(p_1,\cdots,p_k)$ of degree $4k$ such that $\langle L(p_1(\tau_M),\cdots,p_1(\tau_M)),[M]\rangle = \mathrm{signature}(M)$.
\end{theorem}
The $L(p_1(\tau_M),\cdots,p_1(\tau_M))$ is defined in terms of the tangent bundle of the manifold, while the signature relates purely to the topology of the manifold.
This is the main relationship between the manifold and the tangent bundle.
\begin{example}
    $L_1(p_1) = p_1/3$.
    So, what this means is that $\langle p_1(\tau),[M^4]\rangle$ is divisible by $3$.
    This is very cool.
\end{example}
\begin{example}
    $L_2(p_1,p_2) = (7p_2 - p_1^2)/45$.
    That puts very cool divisibility constraints on these char. classes of a tangent bundle of an $8$-manifold.
    This example was used by Milnor to produce manifolds that are homeomorphic to $S^7$ but not diffeomorphic to it.
    These are the \emph{exotic spheres}.
\end{example}
That's an ad for 18.919, by the way, in the fall.
That's the Kan seminar.
You guys are all prepared to take it now.
