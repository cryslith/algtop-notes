\section{Principal bundles, associated bundles}
\subsection{$I$-invariance}
We will denote by $\Vect(B)$ the set of isomorphism classes of vector bundles
over $B$. (Justify the use of the word ``set''!)

Consider a vector bundle $\xi\downarrow B$. If $f:B^\prime\to B$, taking the
pullback gives a vector bundle denoted $f^\ast\xi$. This operation descends to
a map $f^\ast:\Vect(B)\to \Vect(B^\prime)$; we therefore obtain a functor
$\Vect:\Top^{op}\to \Set$. One might expect this functor to give some
interesting invariants of topological spaces.
%\begin{warning}
%    In order to say anything meaningful about vector bundles, you've to assume
%    that the vector bundles are numerable. If $B$ is paracompact (eg a
%    CW-complex), this is automatic.
%\end{warning}
\begin{theorem}\label{Iinvariance}
    Let $I = \Delta^1$. Then $\Vect$ is $I$-invariant. In other words, the
    projection $X\times I\to X$ induces an isomorphism $\Vect(X)\to
    \Vect(X\times I)$.
\end{theorem}
One important corollary of this result is:
\begin{corollary}
    $\Vect$ is a homotopy functor.
\end{corollary}
\begin{proof}
    Consider two homotopic maps $f,g:B\to B^\prime$, so there exists a homotopy
    $H:B^\prime\times I\to B$. If $\xi\downarrow B$, we need to prove that
    $f^\ast_0\xi\simeq f_1^\ast\xi$. This is far from obvious.

    Consider the following diagram.
    \begin{equation*}
	\xymatrix{
	    B^\prime\times I\ar[r]^H\ar[d]_{\pr} & B\\
	    B^\prime & 
	    }
    \end{equation*}
    The leftmost map is an isomorphism under $\Vect$, by Theorem
    \ref{Iinvariance}. Let $\eta\downarrow B$ be a vector bundle such that
    $\pr^\ast\eta \simeq f^\ast\xi$. For any $t\in I$, define a map
    $\in_t:B^\prime\to B^\prime\times I$ sends $x\mapsto(x,t)$. We then have
    isomorphisms:
    $$f_t^\ast\xi \simeq \in_t^\ast f^\ast\xi \simeq \in_t^\ast \pr^\ast\eta
    \simeq (\pr\circ \in_t)^\ast\eta \simeq \eta,$$
    as desired.
\end{proof}
It is easy to see that $\Vect(X)\to \Vect(X\times I)$ is injective. In the next
lecture, we will prove surjectivity, allowing us to conclude Theorem
\ref{Iinvariance}.
\subsection{Principal bundles}
%There's a famous video of J.-P.~Serre talking about mathematics. He says you have to know the difference between ``principle'' and ``principal''. He contemplated ``bundles of principles'' (like politics, or society, or something).
\begin{definition}\label{principaldefn}
    Let $G$ be a topological group\footnote{We will only care about discrete
    groups and Lie groups.}. A \emph{principal $G$-bundle} is a right action of
    $G$ on $P$ such that:
    \begin{itemize}
	\item $G$ acts freely.
	\item The orbit projection $P\to P/G$ is a fiber bundle.
    \end{itemize}
\end{definition}
These are not unfamiliar objects, as the next example shows.
\begin{example}\label{principalcovering}
    Suppose $G$ is discrete. Then the fibers of the orbit projection $P\to P/G$
    are all discrete. Therefore, the condition that $P\to P/G$ is a fiber
    bundle is simply that it's a covering projection (the action is ``properly
    discontinuous'').

    As a special case, let $X$ be a space with universal cover
    $\widetilde{X}\downarrow X$. Then $\pi_1(X)$ acts freely on
    $\widetilde{X}$, and $\widetilde{X}\downarrow X$ is the orbit projection.
    It follows from our discussion above that this is a principal bundle.
    Explicit examples include the principal $\Z/2$-bundle
    $S^{n-1}\downarrow\RP^{n-1}$, and the Hopf fibration $S^{2n-1}\downarrow
    \CP^{n-1}$, whcih is a principle $S^1$-bundle.
\end{example}
By looking at the universal cover, we can classify covering spaces of $X$.
Remember how that goes: if $F$ is a set with left $\pi_1(X)$-action, the dotted
map in the diagram below is the desired covering space.
\begin{equation*}
    \xymatrix{
	\widetilde{X}\times F\ar[r]\ar[d]_{p\circ \pr_1} & \widetilde{X}\times
	F/\sim\ar[dl]^q\\
	X & 
    }
\end{equation*}
Here, we say that $(y,gz)\sim (yg,z)$, for elements $y\in\widetilde{X}$, $z\in
F$, and $g\in\pi_1(X)$.

Fix $y_0\in\widetilde{X}$ over $\ast\in X$. Then it is easy to see that
$F\xrightarrow{\sim}q^{-1}(\ast)$, via the map $z\mapsto (y_0,z)$. This is all
neatly summarized in the following theorem from point-set topology.
\begin{theorem}[Covering space theory]
    There is an equivalence of categories:
    $$\{\text{Left $\pi_1(X)$-sets}\}\xrightarrow{\simeq}\{\text{Covering
    spaces of }X\},$$
    with inverse functor given by taking the fiber over the basepoint and
    lifting a loop in $X$ to get a map from the fiber to itself.
\end{theorem}
Example \ref{principalcovering} shows that covering spaces are special examples
of principal bundles. The above theorem therefore motivates finding a more
general picture.
\begin{construction}
    Let $P\downarrow B$ is a principal $G$-bundle. If $F$ is a left $G$-space,
    we can define a new fiber bundle, exactly as above:
    \begin{equation*}
	\xymatrix{
	    P\times F\ar[r]\ar[d] & P\times F/\sim\ar[dl]^q\\
	    B & 
	    }
    \end{equation*}
    This is called an \emph{associated bundle}, and is denoted $P\times_G F$.
\end{construction}
We must still justify that the resulting space over $B$ is indeed a new fiber
bundle with fiber $F$. Let $x\in B$, and let $y\in P$ over $x$. As above, we
have a map $F\to q^{-1}(\ast)$ via the map $z\mapsto[y,z]$. We claim that this
is a homeomorphism. Indeed, define a map $q^{-1}(\ast)\to F$ via
$$[y^\prime,z^\prime]=[y,gz^\prime]\mapsto gz^\prime,$$
where $y^\prime = yg$ for some $g$ (which is necessarily unique).
\begin{exercise}
    Check that these two maps are inverse homeomorphisms.
\end{exercise}
\begin{definition}
    A vector bundle $\xi\downarrow B$ is said to be an $n$-plane bundle if the
    dimensions of all the fibers are $n$.
\end{definition}
Let $\xi\downarrow B$ be an $n$-plane bundle. Construct a principal
$\GL_n(\RR)$-bundle $P(\xi)$ by defining
$$P(\xi)_b = \{\text{bases for }E(\xi)_b = \mathrm{Iso}(\RR^n, E(\xi)_b)\}.$$
To define the topology, note that (topologically) we have
$$P(B\times \RR^n) = B\times \mathrm{Iso}(\RR^n,\RR^n),$$
where $\mathrm{Iso}(\RR^n,\RR^n) = \GL_n(\RR)$ is given the usual topology as a
subspace of $\RR^{n^2}$.

There is a right action of $\GL_n(\RR)$ on $P(\xi)\downarrow B$, given by
precomposition. It is easy to see that this action is free and simply
transitive. One therefore has a {principal action} of $\GL_n(\RR)$ on $P(\xi)$.
The bundle $P(\xi)$ is called the \emph{principalization} of $\xi$. 

Given the principalization $P(\xi)$, we can recover the total space $E(\xi)$.
Consider the associated bundle $P(\xi)\times_{\GL_n(\RR)}\RR^n$ with fiber $F =
\RR^n$, with $\GL_n(\RR)$ acting on $\RR^n$ from the left. Because this is a
linear action, $P(\xi)\times_{\GL_n(\RR)}\RR^n$ is a vector bundle. One can
show that
$$P(\xi)\times_{\GL_n(\RR)}\RR^n\simeq E(\xi).$$

Fix a topological group $G$. Define $\Bun_G(B)$ as the set of isomorphism
classes of $G$-bundles over $B$. An isomorphism is a $G$-equivariant
homeomorphism over the base. Again, arguing as above, this begets a functor
$\Bun_G:\Top\to\Set$. The above discussion gives a natural isomorphism of
functors:
$$\Bun_{\GL_n(\RR)}(B) \simeq \Vect(B).$$
The $I$-invariance theorem will therefore follow immediately from:
\begin{theorem}
    $\Bun_G$ is $I$-invariant.
\end{theorem}
\begin{remark}
    Principal bundles allow a description of ``geometric structures on $\xi$''.
    Suppose, for instance, that we have a metric on $\xi$. Instead of looking
    at all ordered bases, we can attempt to understand all ordered orthonormal
    bases in each fiber. This give the \emph{frame bundle}
    $$\mathrm{Fr}(B) = \{\text{ordered orthonormal bases of }E(\xi)_b\};$$
    these are isometric isomorphisms $\RR^n\to E(\xi)_b$. Again, there is an
    action of the orthogonal group on $\mathrm{Fr}(B)$: in fact, this begets a
    principal $O(n)$-bundle. Such examples are in abundance: consistent
    orientations give an $SO(n)$-bundle. Trivializations of the vector bundle
    also give principal bundles. This is called ``reduction of the structure
    group''.
\end{remark}
% This wasn't proven in the lectures, but is extremely important.
One useful fact about principal $G$-bundles (which should not be too
surprising) is the following statement.
\begin{theorem}\label{morphismiso}
    Every morphism of principal $G$-bundles is an isomorphism.
\end{theorem}
\begin{proof}
    Let $p:P\to B$ and $p^\prime:P^\prime\to B$ be two principal $G$-bundles
    over $B$, and let $f:P\to P^\prime$ be a morphism of principal $G$-bundles.
    For surjectivity of $f$, let $y\in P^\prime$. Consider $x\in P$ such that
    $p(x) = p^\prime(y)$. Since $p(x) = p^\prime f(x)$ we conclude that $y =
    f(x)g$ for some $g\in G$. But $f(x)g = f(xg)$, so $xg$ maps to $y$, as
    desired.  To see that $f$ is injective, suppose $f(x) = f(y)$. Now $p(x) =
    p^\prime f(x) = p(y)$, so there is some $g\in G$ such that $xg = y$. But
    $f(y) = f(xg) = f(x)g$, so $g=1$, as desired.  We will leave the continuity
    of $f^{-1}$ as an exercise to the reader.
\end{proof}
Theorem \ref{morphismiso} says that if we view $\Bun_G(B)$ as a category where
the morphisms are given by morphisms of principal $G$-bundles, then it is a
groupoid.
