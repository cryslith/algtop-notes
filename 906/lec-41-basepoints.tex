\section{``Cartesian closed'', Hausdorff, Basepoints}\label{basepoints}
Pushouts are colimits, so the quotient space $X/A=X\cup_A \ast$ is an example of a colimit.
Let $Y$ be a topological space, and consider the functor $Y\times -:\Top\to\Top$.
Applying this to the pushout square, we find that $(Y\times X)\cup_{Y\times A}\ast\simeq (Y\times X)/(Y\times A)$.
As we discussed in \S \ref{CGWHspaces}, this product is \emph{not} the same as $Y\times (X/A)$!
There is a bijective map $Y\times X/Y\times A\to Y\times(X/A)$,
but it is not, in general, a homeomorphism.
From a categorical point of view (see Theorem \ref{adjointslimits}), the reason for this failure 
stems from $Y\times -$ not being a left adjoint.

The discussion in \S \ref{mappingspaces} implies that, when working with $k$-spaces,
that functor is indeed a left adjoint
(in fancy language, the category $k\Top$ is Cartesian closed),
which means that --- in $k\Top$ --- there is a homeomorphism $Y\times X/Y\times A\to Y\times(X/A)$.
This addresses the issues raised in \S \ref{CGWHspaces}.
The ancients had come up with a good definition of a topology ---
but $k$-spaces are better!
Sometimes, though, we can be greedy and ask for even more:
for instance, we can demand that points be closed.
This leads to a further refinement of $k$-spaces. 

\todo[inline]{I don't like point-set topology, so I'll return to editing this lecture at the end.}
\subsection{``Hausdorff''}
\begin{definition}
    A space is ``weakly Hausdorff'' if the image of every map $K\to X$ from a compact Hausdorff space $K$ is closed.
\end{definition}
Another way to say this is that the map itself if closed. Clearly Hausdorff implies weakly Hausdorff. Another thing this means is that every point in $X$ is closed (eg $K=\ast$). 
\begin{prop}
    Let $X$ be a $k$-space.
    \begin{enumerate}
	\item $X$ is weakly Hausdorff iff $\Delta:X\to X\times^k X$ is closed. In algebraic geometry such a condition is called separated.
	\item Let $R\subseteq X\times X$ be an equivalence relation. If $R$ is closed, then $X/R$ is weakly Hausdorff.
    \end{enumerate}
\end{prop}
\begin{definition}
    A space is compactly generated if it's a weakly Hausdorff $k$-space. The category of such spaces is called $\CG$.
\end{definition}
We have a pair of adjoint functors $(i,k):\Top\to k\Top$. It's possible to define a functor $k\Top\to \CG$ given by $X\mapsto X/\bigcap\text{all closed equivalence relations}$. It is easy to check that if $Z$ is weakly Hausdorff, then $Z^X$ is weakly Hausdorff (where $X$ is a $k$-space). What this implies is that $\CG$ is also Cartesian closed!

I'm getting a little tired of point set stuff. Let's start talking about homotopy and all that stuff today for a bit. You know what a homotopy is. I will not worry about point-set topology anymore. So when I say $\Top$, I probably mean $\CG$. A homotopy between $f,g:X\to Y$ is a map $h:I\times X\to Y$ such that the following diagram commutes:
\begin{equation*}
    \xymatrix{
	X\ar[dr]_{i_0}\ar[drr]^f & &\\
	& I\times X\ar[r]^h & Y\\
	X\ar[ur]^{i_1}\ar[urr]_g & &
    }
\end{equation*}
We write $f\sim g$. We define $[X,Y]=\Top(X,Y)/\sim$. Well, a map $I\times X\to Y$ is the same as a map $X\to Y^I$ but also $I\to Y^X$. The latter is my favorite! It's a path of maps from $f$ to $g$. So $[X,Y]=\pi_0Y^X$.

To talk about higher homotopy groups and induct etc. we need to talk about basepoints.
\subsection{Basepoints}
A pointed space is $(X,\ast)$ with $\ast\in X$. This gives a category $\Top_\ast$ where the morphisms respect the basepoint. This has products because $(X,\ast)\times (Y,\ast)=(X\times Y,(\ast,\ast))$. How about coproducts? It has coproducts as well. This is the wedge product, defined as $X\sqcup Y/\ast_X\sim \ast_Y=:X\vee Y$. This is \verb|\vee|, not \verb|\wedge|. Is this category also Cartesian closed?

Define the space of pointed maps $Z^X_\ast\subseteq Z^X$ topologized as a subspace. Does the functor $Z\mapsto Z^X_\ast$ have a left adjoint? Well $\Top(W,Z^X)=\Top(X\times W,Z)$. What about $\Top(W,Z^X_\ast)$? This is $\{f:X\times W\to Z:f(\ast,w)=\ast\forall w\in W\}$. That's not quite what I wanted either! Thus $\Top_\ast(W,Z^X_\ast)=\{f:X\times W\to Z:f(\ast,w)=\ast=f(x,\ast)\forall x\in X, w\in W\}$. These send both ``axes'' to the basepoint. Thus, $\Top_\ast(W,Z^X_\ast)=\Top_\ast(X\wedge W,Z)$ where $X\wedge W=X\times W/X\vee W$ because $X\vee W$ are the ``axes''.

So $\Top_\ast$ is not Cartesian closed, but admits something called the smash product\footnote{Remark by Sanath: this is like the tensor product.}. What properties would you like? Here's a good property: $(X\wedge Y)\wedge Z$ and $X\wedge(Y\wedge Z)$ are bijective in pointed spaces. If you work in $k\Top$ or $\CG$, then they are homeomorphic! It also has a unit.

Oh yeah, some more things about basepoints! So there's a canonical forgetful functor $i:\Top_\ast\to \Top$. Let's see. If I have $\Top(X,iY)=\Top_\ast(??,Y)$? This is $X_+=X\sqcup \ast$. Thus we have a left adjoint $(-)_+$. It is clear that $(X\sqcup Y)_+ = X_+ \vee Y_+$. The unit for the smash product is $\ast_+ = S^0$.

On Friday I'll talk about fibrations and fiber bundles.
