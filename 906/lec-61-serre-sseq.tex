\begin{quote}
    Spectral sequences are one of those things for which anybody who is
    anybody must suffer through. Once you've done that, it's like linear
    algebra. You stop thinking so much about the `inner workings' later.

    -- Haynes Miller
\end{quote}
\section{The spectral sequence of a filtered complex}
Our goal will be to describe a method for computing the homology of a chain
complex. We will approach this problem by assuming that our chain complex is
equipped with a filtration; then we will discuss how to compute the associated
graded of an induced filtration on the homology, given the homology of the
associated graded of the filtration on our chain complex.

We will start off with a definition.
\begin{definition}
    A \emph{filtered chain complex} is a chain complex $C_\ast$ along with a
    sequence of subcomplexes $F_s C_\ast$ such that the group $C_n$ has a
    filtration by
    $$F_0 C_n\subset F_1 C_n\subseteq \cdots,$$
    such that $\bigcup F_s C_n = C_n$.
\end{definition}
The differential on $C_\ast$ begets the structure of a chain
complex on the associated graded $\gr_s C_n = F_s C_n/F_{s-1} C_n$; in other
words, the differential on $C_\ast$ respects the filtration, hence begets a
differential $d:\gr_s C_n\to \gr_s C_{n-1}$.

The canonical example of a filtered chain complex to keep in mind is the
homology of a filtered space (such as a CW-complex). Let $X$ be a filtered
space, i.e., a space equipped with a filtration $X_0\subseteq X_1\subseteq
\cdots$ such that $\bigcup X_n = X$. We then have a filtration of the chain
complex $C_\ast(X)$ by the subcomplexes $C_\ast(X_n)$.

For ease of notation, let us write
$$E^0_{s,t} = \gr_s C_{s+t} = F_s C_{s+t}/F_{s-1} C_{s+t},$$
so the differential on $C_\ast$ gives a differential $d^0:E^0_{s,t}\to
E^0_{s,t-1}$. A first approximation to the homology of $C_\ast$ might therefore
be the homology $H_{s+t}(\gr_s C_\ast)$. We will denote this group by
$E^1_{s,t}$. This is the homology of the associated graded of the filtration
$F_\ast C_\ast$.

We can get an even better approximation to $H_\ast C_\ast$ by noticing that
there is a differential even on $E^1_{s,t}$. By construction, there is a short
exact sequence of chain complexes
$$0\to F_{s-1} C_\ast\to F_s C_\ast\to \gr_s C_\ast\to 0,$$
so we get a long exact sequence in homology. The differential on $E^1_{s,t}$ is
the composite of the boundary map in this long exact sequence with the natural
map $H_\ast(F_{s-1} C_\ast)\to H_\ast(\gr_{s-1} C_\ast)$; more precisely, it is
the composite
$$d^1:E^1_{s,t} = H_{s+t}(\gr_s C_\ast)\xar{\partial} H_{s+t-1}(F_{s-1}
C_\ast)\to H_{s+t-1}(\gr_{s-1}C_\ast) = E^1_{s-1,t}.$$
It is easy to check that $(d^1)^2 = 0$.

This construction is already familiar from cellular chains: in this case,
$E^1_{s,t}$ is exactly $H_{s+t}(X_{s},X_{s-1})$, which is exactly the cellular
$s$-chains when $t=0$ (and is $0$ if $t\neq 0$). The $d^1$ differential is
constructed in exactly the same way as the differential on cellular chains.

In light of this, we define $E^2_{s,t}$ to be the homology of the chain complex
$(E^1_{\ast,\ast}, d^1)$; explicitly, we let
$$E^2_{s,t} = \ker(d^1:E^1_{s,t}\to E^1_{s-1,t})/\img(d^1:E^1_{s+1,t}\to
E^1_{s,t}).$$
Does this also have a differential $d^2$? The answer is yes. We will
inductively define $E^r_{s,t}$ via a similar formula: if $E^{r-1}_{\ast,\ast}$
and the differential $d^{r-1}:E^{r-1}_{s,t}\to E^{r-1}_{s-r+1,t+r-2}$ are both
defined, we set
$$E^{r}_{s,t} = \ker(d^{r-1}:E^{r-1}_{s,t}\to
E^{r-1}_{s-r+1,t+r-2})/\img(d^{r-1}:E^{r-1}_{s+r-1,t-r+2}\to E^{r-1}_{s,t}).$$
The differential $d^{r}:E^{r}_{s,t}\to E^{r}_{s-r,t+r-1}$ is defined as
follows. Let $[x]\in E^r_{s,t}$ be represented by an element of $x\in
E^1_{s,t}$, i.e., an element of $H_{s+t}(\gr_s C_\ast)$. As above, the boundary
map induces natural maps $\partial:H_{s+t}(\gr_s C_\ast)\to H_{s+t-1}(F_{s-1}
C_\ast)$ and $\partial:H_{s+t-1}(F_{s-r} C_\ast)\to H_{s+t-1}(\gr_{s-r}
C_\ast)$. The element $\partial x\in H_{s+t-1}(F_{s-1} C_\ast)$ in fact lifts
to an element of $H_{s+t-1}(F_{s-r} C_\ast)$. The image of this element under
$\partial$ inside $H_{s+t-1}(\gr_{s-r} C_\ast) = E^1_{s-r,t+r-1}$ begets a
class in $E^r_{s-r,t+r-1}$; this is the desired differential.
\begin{exercise}
    Fill in the missing details in this construction of $d^r$, and show that
    $(d^r)^2 = 0$.
\end{exercise}
We have proven most of the statements in the following theorem.
\begin{thm-defn}\label{filtered-sseq}
    Let $F_\ast C$ be a filtered complex. Then there exist natural
    \begin{enumerate}
	\item bigraded groups $(E^r_{s,t})_{s\geq 0, t\in\Z}$ for any $r\geq
	    0$, and
	\item differentials $d^r:E^r_{s,t} \to E^r_{s-r,t+r - 1}$ for any
	    $r\geq 0$.
    \end{enumerate}
    such that $E^{r+1}_{s,t}$ is the homology of $(E^r_{\ast,\ast},d^r)$, and
    $(E^0, d^0)$ and $(E^1, d^1)$ are as above. If $F_\ast C$ is bounded
    below, then this spectral sequence \emph{converges to $\gr_\ast
    H_\ast(C)$}, in the sense that there is an isomorphism:
    \begin{equation}\label{convergence}
	E^\infty_{s,t} \simeq \gr_s H_{s+t}(C).
    \end{equation}
\end{thm-defn}
This is called a \emph{homology spectral sequence}. One should think of each
$E^r_{\ast,\ast}$ as a ``page'', with lattice points $E^r_{s,t}$. We still need
to describe the symbols used in the formula \eqref{convergence}.

There is a filtration $F_s H_n(C):=\img(H_n(F_s C) \to H_n(C))$, and $\gr_s
H_\ast(C)$ is the associated graded of this filtration. Taking formula
\eqref{convergence} literally, we only obtain information about the associated
graded of the homology of $C_\ast$. Over vector spaces, this is sufficient to
determine the homology of $C_\ast$, but in general, one needs to solve an
extension problem.

To define the notation $E^\infty$ used above, let us assume that the filtration
$F_\ast C$ is bounded below (so $F_{-1} C = 0$). It follows that $E^0_{s,t} =
F_s C_{s+t}/F_{s-1} C_{s+t} = 0$ for $s<0$, so the spectral sequence of
Theorem-Definition \ref{filtered-sseq} is a ``right half plane'' spectral
sequence. It follows that in our example, the differentials from the group in
position $(s,t)$ must have vanishing $d^{s+1}$ differential.

In turn, this implies that there is a surjection $E^{s+1}_{s,t} \to
E^{s+2}_{s,t}$. This continues: we get surjections
$$E^{s+1}_{s,t}\to E^{s+2}_{s,t}\to E^{s+3}_{s,t}\to\cdots,$$
and the direct limit of this directed system is defined to be $E^\infty_{s,t}$.

For instance, in the case of cellular chains, we argued above that $E^1_{s,t} =
H_{s+t}(X_s, X_{s-1})$, so that $E^1_{s,t} = 0$ if $t\neq 0$, and the $d^1$
differential is just the differential in the cellular chain complex. It follows
that $E^2_{s,t} = H_s^{cell}(X)$ if $t=0$, and is $0$ if $t\neq 0$. All higher
differentials are therefore zero (because either the target or the source is
zero!), so $E^r_{s,t} = E^2_{s,t}$ for every $r\geq 2$. In particular
$E^\infty_{s,t} = H_s^{cell}(X)$ when $t=0$, and is $0$ if $t\neq 0$. There are
no extension problems either: the filtration on $X$ is bounded below, so
Theorem-Definition \ref{filtered-sseq} implies that $\gr_s H_{s+0}(X) = H_s(X)
\simeq H_s^{cell}(X) = E^\infty_{s,t}$.

%OK, so for our theorem, we have the following picture of $E^\infty_{\ast,\ast}$:
%\begin{equation*}
%    \xymatrix{
%	\gr_0 H_n(C) \simeq E^\infty_{0,n} & & & \\
%	& \gr_1 H_n(C) \simeq E^\infty_{1,n-1} & & \\
%	& & \gr_2 H_n(C) \simeq E^\infty_{2,n-2} & & \\
%	& & & \ddots
%    }
%\end{equation*}
%Note that $\gr_0 H_n(C) = F_0 H_n(C) = \img(H_n(F_0 C) \to H_n(C))$.
%
%Maybe you still don't get the homology of $n$ (like if $F_\ast$ is $0$).
%But if $F_\ast C$ is exhaustive, then $F_\ast H_\ast C$ is also exhaustive (I said this on Monday).
%With this assumption, then we get all the associated graded pieces.
In a very precise sense, the datum of the spectral sequence of a filtered
complex $F_\ast C_\ast$ determines the homology of $C_\ast$:
\begin{corollary}
    Let $C\xrightarrow{f} D$ be a map of filtered complexes.
    Assume that the filtration on $C$ and $D$ are bounded below and exhaustive.
    Assume also that $E^r(f)$ is an isomorphism for some $r$.
    Then $f_\ast : H_\ast(C) \to H_\ast(D)$ is an isomorphism.
\end{corollary}
\begin{proof}
    The map $E^r(f)$ is an isomorphism which is also also a chain map, i.e., it
    is compatible with the differential $d^r$. It follows that $E^{r+1}(f)$ is
    an isomorphism. By induction, we conclude that $E^\infty_{s,t}(f)$ is an
    isomorphism for all $s,t$. Theorem-Definitino \ref{filtered-sseq} implies
    that the map $\gr_s(f_\ast):\gr_s H_\ast(C) \to \gr_s H(D)$ is an
    isomorphism.
    
    We argue by induction using the short exact sequence:
    $$
    0 \to F_s H_\ast(C) \to F_{s+1} H_\ast(C) \to \gr_{s+1} H_\ast(C) \to 0.
    $$
    We have $\gr_0 H_n(C) = F_0 H_n(C) = \img(H_n(F_0 C) \to H_n(C))$, so the
    base case follows from the five lemma. In general, $f$ induces an
    isomorphism an isomorphism on the groups on the left (by the inductive
    hypothesis) and right (by the above discussion), so it follows that $F_s
    f_\ast$ is an isomorphism by the five lemma. Since the filtration $F_\ast
    C_\ast$ was exhaustive, it follows that $f_\ast$ is an isomorphism.
\end{proof}
\subsection{Serre spectral sequence}\label{serre-sseq}
In this book, we will give two constructions of the Serre spectral sequence.
The second will appear later. Fix a fibration $E\xrightarrow{p} B$, with $B$ a
CW-complex. We obtain a filtration on $E$ by taking the preimage of the
$s$-skeleton of $B$, i.e., $E_s = p^{-1} \sk_s B$. It follows that there is a
filtration on $S_\ast(E)$ given by
$$F_s S_\ast (E) = \img(S_\ast(p^{-1} \sk_s(B)) \to S_\ast E).$$
This filtration is bounded below and exhaustive. The resulting spectral
sequence of Theorem-Definition \ref{filtered-sseq} is the Serre spectral
sequence.
%The low terms might depend on the CW-structure, but higher terms won't.

Let us be more explicit. We have a pushout square:
\begin{equation*}
    \xymatrix{
	E_{s-1}\ar[r]\ar[d] & E_s\ar[d]\\
	B_{s-1}\ar[r] & B_s\\
	\coprod_{\alpha\in\Sigma_s}S^{s-1}_\alpha\ar[r]\ar[u] & \coprod_{\alpha\in\Sigma_s} D^s_\alpha\ar[u]
    }
\end{equation*}
%I can pullback $E_s$ via:
%\begin{equation*}
%    \xymatrix{
%	\coprod_{\alpha\in \Sigma_s} D^s_\alpha\times F_\alpha\ar[r]\ar[d] & E_s\ar[d]\\
%	\coprod_{\alpha\in \Sigma_s} D^s_\alpha\ar[r] & B_s
%    }
%\end{equation*}
Let $F_\alpha$ be the preimage of the center of $\alpha$ cell. In particular,
we have a pushout:
\begin{equation*}
    \xymatrix{
	E_{s-1}\ar[r] & E_s\\
	\coprod_{\alpha\in\Sigma_s} S^{s-1}_\alpha\times F_\alpha\ar[r]\ar[u] & \coprod_{\alpha\in\Sigma_s} D^s_\alpha \times F_\alpha\ar[u]
    }
\end{equation*}
We know that
$$E^1_{s,t} = H_{s+t}(E_s,E_{s-1}) = \bigoplus_{\alpha\in \Sigma_s}
H_{s+t}(D^s_\alpha\times F_\alpha, S^{s-1}_\alpha\times F_\alpha).$$
We can suggestively view this as $\bigoplus_{\alpha\in\Sigma_s}
H_{s+1}((D^s_\alpha, S^{s-1}_\alpha)\times F_\alpha)$. By the K\"unneth
formula (at least, if our coefficients are in a field), this is exactly
$\bigoplus_{\alpha\in\Sigma_s} H_t(F_\alpha)$. In analogy with our discussion
above regarding the spectral sequence coming from the cellular chain complex,
one would like to think of this as ``$C_s(B;H_t(F_\alpha))$''. Sadly, there are
many things wrong with writing this.

For instance, suppose $B$ isn't connected. The fibers $F_\alpha$ could have
completely different homotopy types, so the symbol $C_s(B;H_t(F_\alpha))$
does not make any sense. Even if $B$ was path-connected, there would still be
no canonical way to identify the fibers over different points. Instead, we
obtain a functor $H_t(p^{-1}(-)):\Pi_1(B) \to \mathbf{Ab}$, i.e., a ``local
coefficient system'' on $B$. So, the right thing to say is ``$E^2_{s,t} =
H_s(B;\underline{H_t(\mathrm{fiber})})$''.

To define precisely what $H_s(B;\underline{H_t(\mathrm{fiber})})$ means, let us
pick a basepoint in $B$, and build the universal cover $\widetilde{B}\to B$.
This has an action of $\pi_1(B,\ast)$, so we obtain an action of
$\pi_1(B,\ast)$ on the chain complex $S_\ast(\widetilde{B})$. Said differently,
$S_\ast(\widetilde{B})$ is a chain complex of right modules over
$\Z[\pi_1(B)]$. If $B$ is connected, a local coefficient system on $B$ is the
same thing as a (left) action of $\pi_1(B)$ on $H_t(p^{-1}(\ast))$. Then, we
define a chain complex:
$$
S_\ast(B;\underline{H_t(p^{-1}(\ast))}) = S_\ast(\widetilde{B})
\otimes_{\Z[\pi_1(B)]} H_t(p^{-1}(\ast));
$$
the differential is induced by the $\Z[\pi_1(B)]$-equivariant differential on
$S_\ast(\widetilde{B})$. Our discussion above implies that the homology of this
chain complex is the $E^2$-page.

We will always be in the case where that local system is trivial, so that
$H_\ast(B;\underline{H_\ast(p^{-1}(\ast))})$ is just
$H_\ast(B;H_\ast(p^{-1}(\ast)))$. For instance, this is the case if $\pi_1(B)$
acts trivially on the fiber. In particular, this is the case if $B$ is simply
connected. 
