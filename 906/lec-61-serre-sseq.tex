\section{Serre spectral sequence}
Spectral sequences are one of these things for which anybody who is anybody must suffer through.
Once you've done that, it's like linear algebra.
You stop thinking so much about the ``inner workings'' later.
Today will be the surface story, and we'll get into the innards later.

On Monday I was talking about the example of a filtered space.
That gives rise to a filtered complex.
That's a chain complex $C$, with an increasing sequence $F_\ast C$ of subchain complexes.

We saw various things: we've got
\begin{equation*}
    \xymatrix{
	F_{s-1} C \ar[r] & F_s C \ar[d] \ar[r] & F_{s+1} C\ar[d]\\
	& \gr_s C & \gr_{s+1} C
    }
\end{equation*}
We then let $E^0_{st} = \gr_s C_{s+t}$, and we have a differential $d^0:E^0_{st} \to E^0_{s, t-1}$.
We can go even further: I'll define $E^1_{st} = H_{s,t}(E^0_{s,t}, d^0)$.
\begin{equation*}
    \xymatrix{
	HF_{s-1}\ar[r] & HF_s \ar[r] \ar[d] & HF_{s+1}\\
	E^1_{s-1,\ast} & E^1_{s,t}\ar[l]^{d^1} & 
    }
\end{equation*}
We saw this already in cellular chains, where the $E^1$ were the cellular chains.
Now, $d^1:E^1_{s,t} \to E^1_{s-1,t}$ since the differentials decrease total degree by $1$.
I could draw out the spectral sequence like this\todo{use sseq package to redraw this!!}:
\begin{equation*}
    \xymatrix{
	\ast & & \\
	& \ast & \ast\ar[d]^{d^0}\ar[l]_{d^1}\ar[ull]_{d^2} \\
	& & \ast
    }
\end{equation*}
the vertical thing is $t$, called the complementary degree, and the horizontal indexing is the filtration degree $s$.
\begin{theorem}
    Let $F_\ast C$ be a filtered complex.
    There exists natural:
    \begin{enumerate}
	\item for any $r\geq 0$, a bigraded group $(E^r_{s,t})_{s\geq 0, t\in\Z}$.
	\item for any $r\geq 0$, a differential $d^r:E^r_{s,t} \to E^r_{s-r,t+r - 1}$.
	\item $H_{s,t}(E^r_{\ast,\ast}, d^r) = E^{r+1}_{s,t}$.
    \end{enumerate}
    and $(E^0, d^0)$ and $(E^1, d^1)$ are as above.
    Then under certain hypotheses (see below) ($F_\ast C$ is bounded below), there is an isomorphism:
    $$
    E^\infty_{s,t} \simeq \gr_s H_{s+t}(C)
    $$
    See below for what $E^\infty$ means.
\end{theorem}
It's supposed to help us compute the homology of the thing we started with.
This is called a \emph{homology spectral sequence}.

There's a filtration $F_s H_n(C):=\img(H_n(F_s C) \to H_n(C))$.
What the spectral sequence will tell you is information about the associated graded.
This isn't so bad if you're over, like, vector spaces.

Let's assume that the filtration $F_\ast C$ is bounded below (so $F_{-1} C = 0$).
Then $E^0_{s,t} = F_s C_{s+t}/F_{s-1} C_{s+t} = 0$ for $s<0$.
It's a ``right half plane'' spectral sequence.
Thus, in our example, the differentials from the group at $(s,t)$ must have $d^{s+1} = 0$.
This tells us that there's a surjection $E^{s+1}_{s,t} \to E^{s+2}_{s,t}$.
And that continues: we get surjections all the way, and take the direct limit of $E^{r}_{s,t}$, which is defined to be $E^\infty_{s,t}$.

OK, so for our theorem, we have the following picture of $E^\infty_{\ast,\ast}$:
\begin{equation*}
    \xymatrix{
	\gr_0 H_n(C) \simeq E^\infty_{0,n} & & & \\
	& \gr_1 H_n(C) \simeq E^\infty_{1,n-1} & & \\
	& & \gr_2 H_n(C) \simeq E^\infty_{2,n-2} & & \\
	& & & \ddots
    }
\end{equation*}
Note that $\gr_0 H_n(C) = F_0 H_n(C) = \img(H_n(F_0 C) \to H_n(C))$.

Maybe you still don't get the homology of $n$ (like if $F_\ast$ is $0$).
But if $F_\ast C$ is exhaustive, then $F_\ast H_\ast C$ is also exhaustive (I said this on Monday).
With this assumption, then we get all the associated graded pieces.
\begin{corollary}
    Let $C\xrightarrow{f} D$ be a map of filtered complexes.
    Assume that the filtration on $C$ and $D$ are bounded below and exhaustive.
    Assume also that $E^r(f)$ is an isomorphism for some $r$.
    Then $f_\ast : H(C) \to H(D)$ is an isomorphism.
\end{corollary}
You're completely recovering the homology of $C$ in this sense.
\begin{proof}
    $E^r(f)$ is an isomorphism, and it's also a chain map (it's compatible with the differential $d^r$).
    Thus $E^{r+1}(f)$ is an isomorphism.
    And so on.
    Thus, $E^\infty_{s,t}(f)$ is an isomorphism for all $s,t$.
    Thus the associated graded map $\gr_s(f_\ast):\gr_s H_\ast(C) \to \gr_s H(D)$ is an isomorphism.
    I have a short exact sequence:
    $$
    0 \to F_0 H_\ast \to F_1 H_\ast \to \gr_1 H_\ast \to 0
    $$
    We have an isomorphism on the left and right nontrivial groups, so by the five lemma and induction, $F_s f_\ast$ is an isomorphism.
    The filtration was exhaustive, so $f_\ast$ is an isomorphism.
\end{proof}
This is convergence following Eilenberg-Moore.

Wow, OK.
Let's talk about the Serre spectral sequence.
\subsection{Serre spectral sequence}
I'll give two constructions.
I have a fibration $E\xrightarrow{p} B$, with $B$ a CW-complex.
I can define a filtration on $E$ by taking preimages.
So, I get a filtration $F_s S_\ast (E) = \img(S_\ast(p^{-1} \sk_s(B)) \to S_\ast E)$.
I'll let $E_s = p^{-1} \sk_s B$.
This is bounded below and exhaustive.
And that's the Serre spectral sequence.
The low terms might depend on the CW-structure, but higher terms won't.

Let's think about how you build $E_{s-1}$.
You have a pushout:
\begin{equation*}
    \xymatrix{
	E_{s-1}\ar[r]\ar[d] & E_s\ar[d]\\
	B_{s-1}\ar[r] & B_s\\
	\coprod_{\alpha\in\Sigma_s}S^{s-1}_\alpha\ar[r]\ar[u] & \coprod_{\alpha\in\Sigma_s} D^s_\alpha\ar[u]
    }
\end{equation*}
I can pullback $E_s$ via:
\begin{equation*}
    \xymatrix{
	\coprod_{\alpha\in \Sigma_s} D^s_\alpha\times F_\alpha\ar[r]\ar[d] & E_s\ar[d]\\
	\coprod_{\alpha\in \Sigma_s} D^s_\alpha\ar[r] & B_s
    }
\end{equation*}
where $F_\alpha$ is $p^{-1}$ of the center of $\alpha$ cell.
In particular, I have a pushout:
\begin{equation*}
    \xymatrix{
	E_{s-1}\ar[r] & E_s\\
	\coprod_{\alpha\in\Sigma_s} S^{s-1}_\alpha\times F_\alpha\ar[r]\ar[u] & \coprod_{\alpha\in\Sigma_s} D^s_\alpha \times F_\alpha\ar[u]
    }
\end{equation*}
This fits inside a cube that I didn't have time to \TeX.
Now, we know that $E^1_{s,t} = H_{s+t}(E_s,E_{s-1})$.
The same argument shows that $E^1_{s,t} = \bigoplus_{\alpha\in \Sigma_s} H_{s+t}(D^s_\alpha\times F_\alpha, S^{s-1}_\alpha\times F_\alpha)$.
I can write this differently as $\bigoplus_{\alpha\in\Sigma_s} H_{s+1}((D^s_\alpha, S^{s-1}_\alpha)\times F_\alpha)$, which is, by K\"unneth, exactly $\bigoplus_{\alpha\in\Sigma_s} H_t(F_\alpha)$.
I guess I can write this as ``$C_s(B;H_t(F_\alpha))$''.
There's some things wrong with writing it like that.

I mean, suppose $B$ isn't connected.
Then the $F_\alpha$ could have completely different homotopy types.
That symbol $C_s(B;H_t(F_\alpha))$ doesn't make any sense then.
Even if you're path-connected, there's still no canonical way to identify the fiber over one point to the fiber over another point.
What you get is that $H_t(p^{-1}(-)):\Pi_1(B) \to \mathbf{Ab}$.
This is the same thing as a ``local coefficient system'' on $B$.
So, the right thing to say is:
$$
E^2_{s,t} = H_s(B;\underline{H_t(\mathrm{fiber})})
$$
Pick a basepoint in $B$, and build the universal cover $\widetilde{B}\to B$, with an action of $\pi_1(B,\ast)$.
Look at $S_\ast(\widetilde{B})$.
There is an action of $\pi_1(B,\ast)$ on this chain complex.
So $S_\ast(\widetilde{B})$ is a chain complex of right modules over $\Z[\pi_1(B)]$.
If $B$ is connected, then giving a local coefficient system on $B$ is the same thing as giving an action of $\pi_1(B)$ on $H_t(p^{-1}(\ast))$.
This is now a left action.
You know what you want to do then:
$$
S_\ast(B;\underline{H_t(p^{-1}(\ast))}) = S_\ast(\widetilde{B}) \otimes_{\Z[\pi_1(B)]} H_t(p^{-1}(\ast))
$$
The homology of this chain complex is the $E^2$-page.
On Friday, I'll try to show you some applications of this.

We'll always be in the case where that local system is trivially (eg if $\pi_1$ acts trivially, like if $B$ is simply connected).
