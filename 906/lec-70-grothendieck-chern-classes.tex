\section{Grothendieck's construction of Chern classes}
%No class Monday because I'll be out!
%We have a few lectures left and I'll use them to talk about characteristic classes, and we'll see some applications.
%Maybe we'll have time to give a construction of Steenrod operations as well.
\subsection{Generalities on characteristic classes}
We would like to apply algebraic techniques to study $G$-bundles on a space.
Let $A$ be an abelian group, and $n\geq 0$ an integer.
\begin{definition}
    A \emph{characteristic class} for principal $G$-bundles (with values in
    $H^n(-;A)$) is a natural transformation of functors $\Top\to \Ab$:
    $$\Bun_G(X) \xar{c} H^n(X;A)$$
    Concretely: if $P\to Y$ is a principal $G$-bundle over a space $X$, and
    $f:X\to Y$ is a continuous map of spaces, then
    $$c(f^\ast P) = f^\ast c(P).$$
\end{definition}
The motivation behind this definition is that $\Bun_G(X)$ is still rather
mysterious, but we have techniques (developed in the last section) to compute
the cohomology groups $H^n(X;A)$. It follows by construction that if two
bundles over $X$ have two different characteristic classes, then they cannot be
isomorphic. Often, we can use characteristic classes to distinguish a given
bundle from the trivial bundle.

\begin{example}
    The Euler class takes an oriented real $n$-plane vector bundle (with a
    chosen orientation) and produces an $n$-dimensional cohomology class
    $e:\Vect^{or}_n(X) = \Bun_{SO(n)}(X)\to H^n(X;\Z)$. This is a
    characteristic class. To see this, we need to argue that if $\xi\downarrow
    X$ is a principal $G$-bundle, we can pull the Euler class back via $f:X \to
    Y$. The bundle $f^\ast\xi\downarrow Y$ has a orientation if $\xi$ does, so
    it makes sense to even talk about the Euler class of $f^\ast\xi$. Since all
    of our constructions were natural, it follows that $e(f^\ast\xi) = f^\ast
    e(\xi)$.

    Similarly, the {mod $2$ Euler class} is $e_2:\Vect_n(X) = \Bun_{O(n)}(X)
    \to H^n(X;\Z/2\Z)$ is another Euler class. Since everything has an
    orientation with respect to $\Z/2\Z$, the mod $2$ Euler class is
    well-defined. 
\end{example}

By our discussion in \S \ref{classifying-g-bundles}, we know that $\Bun_G(X) =
[X,BG]$. Moreover, as we stated in Theorem \ref{brown-rep}, we know that
$H^n(X;A) = [X,K(A,n)]$ (at least if $X$ is a CW-complex). One moral reason for
cohomology to be easier to compute is that the spaces $K(A,n)$ are infinite
loop spaces (i.e., they can be delooped infinitely many times). It follows from
the Yoneda lemma that characteristic classes are simply maps $BG\to K(A,n)$,
i.e., elements of $H^n(BG;A)$.

\begin{example}
    The Euler class $e$ lives in $H^n(BSO(n);\Z)$; in fact, it is $e(\xi)$, the
    Euler class of the universal oriented $n$-plane bundle over $BSO(n)$. A
    similar statement holds for $e_2\in H^n(BO(n);\Z/2\Z)$. For instance, if
    $n=2$, then $SO(2) = S^1$. It follows that
    $$BSO(2) = BS^1 = \CP^\infty.$$
    We know that $H^\ast(\CP^\infty;\Z) = \Z[e]$ --- it's the polynomial
    algebra on the ``universal'' Euler class! Similarly, $O(1) = \Z/2\Z$, so
    $$BO(1) = B\Z/2 = \RP^\infty.$$
    We know that $H^\ast(\RP^\infty;\FF_2) = \FF_2[e_2]$ --- as above, it is
    the polynomial algebra over $\Z/2\Z$ on the ``universal'' mod $2$ Euler
    class.
\end{example}

\subsection{Chern classes}
These are one of the most fundamental example of characteristic classes.
\begin{theorem}[Chern classes]\label{chern-classes}
    There is a unique family of characteristic classes for complex vector
    bundles that assigns to a complex $n$-plane bundle $\xi$ over $X$ the
    \emph{$n$th Chern class} $c^{(n)}_k(\xi)\in H^{2k}(X;\Z)$, such that:
    \begin{enumerate}
	\item $c^{(n)}_0(\xi) = 1$.
	\item If $\xi$ is a line bundle, then $c^{(1)}_1(\xi) = -e(\xi)$.
	\item The \emph{Whitney sum formula} holds: if $\xi$ is a $p$-plane
	    bundle and $\eta$ is a $q$-plane bundle (and if $\xi\oplus\eta$
	    denotes the fiberwise direct sum), then
	    \begin{equation*}
		c^{(p+q)}_k(\xi\oplus \eta) = \sum_{i+j=k} c^{(p)}_i(\xi)\cup
		c^{(q)}_j(\eta) \in H^{2k}(X;\Z).
	    \end{equation*}
    \end{enumerate}
    Moreover, if $\xi_n$ is the universal $n$-plane bundle, then
    $$H^\ast(BU(n);\Z) \simeq \Z[c_1^{(n)}, \cdots, c^{(n)}_n],$$
    where $c^{(n)}_k = c^{(n)}_k(\xi_n)$.
\end{theorem}
This result says that all characteristic classes for complex vector bundles are
given by polynomials in the Chern classes because the cohomology of $BU(n)$
gives all the characteristic classes.  It also says that there are no universal
algebraic relations among the Chern classes: you can specify them
independently.

\begin{remark}
    The $(p+q)$-plane bundle $\xi_p\times \xi_q = \pr_1^\ast \xi_p\oplus
    \pr_2^\ast\xi_q$ over $BU(p)\times BU(q)$ is classified by a map
    $BU(p)\times BU(q) \xar{\mu} BU(p+q)$. The Whitney sum formula computes the
    effect of $\mu$ on cohomology:
    $$\mu^\ast(c^{(n)}_k) = \sum_{i+j = k} c^{(p)}_i \times c^{(q)}_j \in
    H^{2k}(BU(p)\times BU(q)),$$
    where, you'll recall,
    $$x\times y := \pr_1^\ast x \cup \pr_2^\ast y.$$
\end{remark}

The Chern classes are ``stable'', in the following sense. Let $\epsilon$ be the
trivial one-dimensional complex vector bundle, and let $\xi$ be an
$n$-dimensional vector bundle. What is $c^{(n+q)}_k(\xi\oplus\epsilon^q)$? For
this, the Whitney sum formula is valuable.

The trivial bundle is characterized by the pullback:
\begin{equation*}
    \xymatrix{
	X\times \cC^n = n\epsilon \ar[r]\ar[d] & \cC^n\ar[d]\\
	X\ar[r] & \ast
    }
\end{equation*}
By naturality, we find that if $k>0$, then $c^{(n)}_k(n\epsilon) = 0$. The
Whitney sum formula therefore implies that
$$c^{(n+q)}_k(\xi\oplus \epsilon^q) = c^{(n)}_k(\xi).$$
This phenomenon is called stability: the Chern class only depends on the
``stable equivalence class'' of the vector bundle (really, they are only
defined on ``K-theory'', for those in the know). For this reason, we will drop
the superscript on $c^{(n)}_k(\xi)$, and simply write $c_k(\xi)$.

\subsection{Grothendieck's construction}\label{grothendieck-chern}
Let $\xi$ be an $n$-plane bundle. We can consider the vector bundle
$\pi:\PP(\xi)\to X$, the projectivization of $\xi$: an element of the fiber of
$\PP(\xi)$ over $x\in X$ is a line inside $\xi_x$, so the fibers are therefore
all isomorphic to $\CP^{n-1}$.

Let us compute the cohomology of $\PP(\xi)$. For this, the Serre spectral
sequence will come in handy:
$$
E_2^{s,t} = H^s(X; H^t(\CP^{n-1})) \Rightarrow H^{s+t}(\PP(\xi)).
$$
\begin{remark}
    Why is the local coefficient system constant? The space $X$ need not be
    simply connected, but $BU(n)$ is simply connected since $U(n)$ is simply
    connected. Consider the projectivization of the universal bundle
    $\xi_n\downarrow BU(n)$; pulling back via $f:X\to BU(n)$ gives the bundle
    $\pi:\PP(\xi)\to X$. The map on fibers $H^\ast(\PP(\xi_n)_{f(x)}) \to
    H^\ast(\PP(\xi_n)_{x})$ is an isomorphism which is equivariant with respect
    to the action of the fundamental group of $\pi_1(X)$ via the map $\pi_1 (X)
    \to \pi_1(BU(n)) = 0$.
\end{remark}
Because $H^\ast(\CP^{n-1})$ is torsion-free and finitely generated in each
dimension, we know that
$$E_2^{s,t} \simeq H^s(X) \otimes H^t(\CP^{n-1}).$$
The spectral sequence collapses at $E_2$, i.e., that $E_2 \simeq E_\infty$,
i.e., there are no differentials. We know that the $E_2$-page is generated as
an algebra by elements in the cohomology of the fiber and elements in the
cohomology of the base. Thus, it suffices to check that elements in the
cohomology of the fiber survive to $E_\infty$. We know that
$$E_2^{0,2t} = \Z\langle x^t\rangle,\text{ and }E_2^{0,2t+1} = 0,$$
where $x = e(\lambda)$ is the Euler class of the canonical line bundle
$\lambda\downarrow\CP^{n-1}$.

In order for the Euler class to survive the spectral sequence, it suffices to
come up with a two dimensional cohomology class in $\PP(\xi)$ that restricts to
the Euler class over $\CP^{n-1}$. We know that $\lambda$ itself is the
restriction of the tautologous line bundle over $\CP^\infty$. There is a
tautologous line bundle $\lambda_\xi \downarrow \PP(\xi)$, given by the
tautologous line bundle on each fiber. Explicitly:
$$E(\lambda_\xi)=\{(\ell,y)\in \PP(\xi)\times_X E(\xi) | y\in \ell\subseteq
\xi_x\}.$$
Thus, $x$ is the restriction $e(\lambda_\xi)|_\text{fiber}$ of the Euler class
to the fiber. It follows that the class $x$ survives to the $E_\infty$-page.

Using the Leray-Hirsch theorem (Theorem \ref{leray-hirsch}), we conclude that
$$H^\ast(\PP(\xi)) = H^\ast(X)\langle 1, e(\lambda_\xi), e(\lambda_\xi)^2,
\cdots, e(\lambda_\xi)^{n-1}\rangle.$$
For simplicity, let us write $e = e(\lambda_\xi)$. Unforunately, we don't know
what $e^n$ is, although we do know that it is a linear combination of the $e^k$
for $k<n$. In other words, we have a relation
$$e^n + c_1e^{n-1} + \cdots + c_{n-1} e + c_n = 0,$$
where the $c_k$ are elements of $H^{2k}(X)$. These are the Chern classes of
$\xi$. By construction, they are unique!

To prove Theorem \ref{chern-classes}(2), note that when $n=1$ the above
equation reads
$$e+c_1 = 0,$$
as desired.

%\begin{question}
%    Why do we know that they are the Chern classes?
%    My margin is too narrow to provide a proof.
%\end{question}
%We'll prove this on Wednesday.
